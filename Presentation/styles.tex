% Obshchie stili oformleniya.
% Vozmozhnye varianty znacheniy ishchite v opisanii biblioteki beamer
\usetheme{Pittsburgh}
\usecolortheme{whale}

% \usetheme[secheader]{Boadilla}
% \usecolortheme{seahorse}

% Razmer poley slaydov
\setbeamersize{text margin left=1cm,%
               text margin right=1cm}

% vyklyuchenie knopok navigatsii
\beamertemplatenavigationsymbolsempty

% Razmery shriftov
\setbeamerfont{title}{size=\large}
\setbeamerfont{subtitle}{size=\small}
\setbeamerfont{author}{size=\normalsize}
\setbeamerfont{institute}{size=\small}
\setbeamerfont{date}{size=\normalsize}
\setbeamerfont{bibliography item}{size=\small}
\setbeamerfont{bibliography entry author}{size=\small}
\setbeamerfont{bibliography entry title}{size=\small}
\setbeamerfont{bibliography entry location}{size=\small}
\setbeamerfont{bibliography entry note}{size=\small}
% Analogichno mozhno nastroit i drugie razmery.
% Nazvaniya klassov elementov mozhno nayti zdes
% http://www.cpt.univ-mrs.fr/~masson/latex/Beamer-appearance-cheat-sheet.pdf

% Tsvet elementov
\setbeamercolor{footline}{fg=blue}
\setbeamercolor{bibliography item}{fg=black}
\setbeamercolor{bibliography entry author}{fg=black}
\setbeamercolor{bibliography entry title}{fg=black}
\setbeamercolor{bibliography entry location}{fg=black}
\setbeamercolor{bibliography entry note}{fg=black}
% Analogichno mozhno nastroit i drugie tsveta.
% Nazvaniya klassov elementov mozhno nayti zdes
% http://www.cpt.univ-mrs.fr/~masson/latex/Beamer-appearance-cheat-sheet.pdf

% Numerovat spisok statey
% https://tex.stackexchange.com/a/419506/104425
\setbeamertemplate{bibliography item}{\insertbiblabel}
% ili ubrat nomera
% \setbeamertemplate{bibliography item}{}

% Ispolzovat shrift s zasechkami dlya formul
% https://tex.stackexchange.com/a/34267/104425
\usefonttheme[onlymath]{serif}

% https://tex.stackexchange.com/a/291545/104425
\makeatletter
\def\beamer@framenotesbegin{% at beginning of slide
    \usebeamercolor[fg]{normal text}
    \gdef\beamer@noteitems{}%
    \gdef\beamer@notes{}%
}
\makeatother

% footer prezentatsii
\setbeamertemplate{footline}{
    \leavevmode%
    \hbox{%
        \begin{beamercolorbox}[wd=.333333\paperwidth,ht=2.25ex,dp=1ex,center]{}%
            % I. O. Familiya, Organizatsiya kratko
            \thesisAuthorShort, \thesisOrganizationShort
        \end{beamercolorbox}%
        \begin{beamercolorbox}[wd=.333333\paperwidth,ht=2.25ex,dp=1ex,center]{}%
            % Gorod, 20XX
            \thesisCity, \thesisYear
        \end{beamercolorbox}%
        \begin{beamercolorbox}[wd=.333333\paperwidth,ht=2.25ex,dp=1ex,right]{}%
            Str. \insertframenumber{} iz \inserttotalframenumber \hspace*{2ex}
        \end{beamercolorbox}}%
    \vskip0pt%
}

% vyvod na ekran zametok k prezentatsii
\ifnumequal{\value{presnotes}}{0}{}{
    \setbeameroption{show notes}
    \ifnumequal{\value{presnotes}}{2}{
        \setbeameroption{show notes on second screen=\presposition}
    }{}
}
