\section{Spiski}
\begin{frame}[plain, noframenumbering]
    \begin{center}
        \Huge
        Spiski
    \end{center}
\end{frame}

\subsection{Numerovannye}

\begin{frame}
    \frametitle{Numerovannye spiski}
    \begin{enumerate}
        \item odin
        \item dva
        \item tri
    \end{enumerate}
\end{frame}
\note{
    Etot tekst budet viden tolko esli ego otobrazhenie vklyucheno
    v~fayle \textbf{Presentation/setup}.
    Dlya razdelnogo vyvoda prezentatsii i zametok na~raznye ekrany (kak
    v~impress ili powerpoint) mozhno ispolzovat programmu
    \textit{pdf-presenter-console}.
}

\subsection{Ne numerovannye}


\begin{frame}
    \frametitle{Perechisleniya}
    \begin{itemize}
        \item Problema 1
        \item Problema 2
        \item Problema 3
    \end{itemize}
\end{frame}
\note[itemize]{
    \item Tezis 1
    \item Tezis 2
    \item Tezis 3
}

\subsection{Kombinirovannye}

\begin{frame}
    \frametitle{Kombinatsiya spiskov}
    \begin{enumerate}
        \item \textbf{Zadacha 1}
              \begin{itemize}
                  \item Podzadacha 1-1
                  \item Podzadacha 1-2
              \end{itemize}
        \item \textbf{Zadacha 2}
              \begin{itemize}
                  \item Podzadacha 2-1
                  \item Podzadacha 2-2
                  \item Podzadacha 2-3
              \end{itemize}
        \item \textbf{Zadacha 3}
              \begin{itemize}
                  \item Podzadacha 3-1
                  \item Podzadacha 3-2
                  \item Podzadacha 3-3
              \end{itemize}
    \end{enumerate}
\end{frame}
\note[itemize]{
    \item Zadacha 1
    \item Zadacha 2
    \item Zadacha 3
}

\begin{frame}[allowframebreaks]
    \frametitle{Razdelenie slayda}
    Poyasnyayushchiy tekst
    \begin{itemize}
        \item Odin
        \item Dva
        \item Tri
    \end{itemize}
    \framebreak
    Prodolzhenie predydushchego slayda
\end{frame}

\section{Grafika}
\begin{frame}[plain, noframenumbering]
    \begin{center}
        \Huge
        Grafika
    \end{center}
\end{frame}


\begin{frame}
    \frametitle{Odinochnoe izobrazhenie}
    \centering
    \includegraphics[width=0.8\linewidth]{latex} % okruzhenie figure ne trebuetsya
\end{frame}

\begin{frame}
    \frametitle{Vektornaya grafika}
    \begin{figure}
        \centering
        \ifdefmacro{\tikzsetnextfilename}{\tikzsetnextfilename{tikz_presentation}}{}% prisvaivaemoe predkompilirovannomu pdf imya fayla (ne obyazatelno)
        \input{Presentation/images/tikz_plot.tikz}
    \end{figure}
\end{frame}

\subsection{Raspolozhenie}

\begin{frame}
    \frametitle{Izobrazheniya po-vertikali}
    \centering
    \vfill
    \includegraphics[width=0.8\linewidth,height=0.1\textheight]{latex} \\
    \TeX
    \vfill
    \includegraphics[width=0.8\linewidth,height=0.2\textheight]{latex} \\
    \LaTeX
    \vfill
    \includegraphics[scale=0.2]{latex} \\
    \vfill
\end{frame}


\begin{frame}
    \frametitle{Izobrazheniya po-gorizontali}
    \begin{minipage}[t]{0.47\linewidth}
        \textbf{Sostavnaya \\ podpis 1}
        \center{\includegraphics[width=1\linewidth]{knuth1}}
    \end{minipage}
    \hfill
    \begin{minipage}[t]{0.47\linewidth}
        \textbf{Sostavnaya \\ podpis 2}
        \center{\includegraphics[width=1\linewidth]{knuth2}}
    \end{minipage}
\end{frame}

\subsection{Linii}

\begin{frame}
    \frametitle{Razdelyayushchie linii}
    \begin{minipage}[c]{0.47\linewidth}
        \center{\includegraphics[width=1\linewidth]{latex}}
        \bigskip
        \hrule{}
        \bigskip
        \textbf{Sostavnaya \\ podpis 1}
    \end{minipage}
    \hfill
    \vrule{}
    \hfill
    \begin{minipage}[c]{0.47\linewidth}
        \flushright
        \textbf{Sostavnaya \\ podpis 2}
        \center{\includegraphics[width=1\linewidth]{knuth2}}
    \end{minipage}
\end{frame}

\section{Ostalnoe}
\begin{frame}[plain, noframenumbering]
    \begin{center}
        \Huge
        Ostalnoe
    \end{center}
\end{frame}

\subsection{Formuly}

\begin{frame}
    \frametitle{Formuly}
    \[
    \left\{
    \begin{array}{rl}
        \dot x = & \sigma (y-x)  \\
        \dot y = & x (r - z) - y \\
        \dot z = & xy - bz
    \end{array}
    \right.
    \]
\end{frame}

\begin{frame}
    \frametitle{amsmath}
    \centering
    \begin{minipage}[t]{0.5\linewidth}
        \begin{multline*}
            y = 1 x^1 + 2 x^2 + 3 x^3 + \\ + 4 x^4 + 5 x^5 + \dots
        \end{multline*}
    \end{minipage}
\end{frame}

\begin{frame}[allowframebreaks]
    \frametitle{Uravneniya Maksvella}
    \centering{
        \small
        \def\arraystretch{1.8}%
        \begin{tabular}{ll}
            \toprule
            Integralnaya forma                                                                                                                                            & Differentsialnaya forma                                                          \\ \midrule
            \(Q_e(t) = \displaystyle\oiint_S \vec D(t) \cdot d\vec{s} = \displaystyle\iiint_V \rho_v(t) dv\)                                                              & \(\nabla \cdot \vec D(t) = \rho_v(t)\)                                          \\
            \(\displaystyle\oiint_S \vec B(t) \cdot d\vec{s} = 0\)                                                                                                        & \(\nabla \cdot \vec B(t) = 0\)                                                  \\
            \(V_{emf}(t) = \displaystyle\oint_L \vec E(t) \cdot d\vec{l}\) = \(- \displaystyle\iint_S \left[\frac{\partial\vec{B}(t)}{\partial t}\right] \cdot d\vec{s}\) & \(\nabla \times \vec E(t) = - \frac{\partial\vec{B}(t)}{\partial t}\)           \\
            \(I(t) = \displaystyle\oint_L \vec H(t) \cdot d\vec{l} = \displaystyle\iint_S \left[\vec J(t) + \frac{\partial\vec{D}(t)}{\partial t}\right] \cdot d\vec{s}\) & \(\nabla \times \vec H(t) = \vec J(t) + \frac{\partial\vec{D}(t)}{\partial t}\) \\ \midrule
            \(\displaystyle\oiint_S \vec J \cdot d\vec{s} = -\frac{\partial Q_e}{\partial t}\)                                                                            & \(\nabla \cdot \vec J = - \frac{\partial \rho_v}{\partial t}\)                  \\
            \bottomrule
            \multicolumn{2}{c}{\(\vec D(t) = \left[\varepsilon(t)\right] * \vec E(t)\)}                                                                                                                                                                     \\
            \multicolumn{2}{c}{\(\vec B(t) = \left[\mu(t)\right] * \vec H(t)\)}                                                                                                                                                                             \\
        \end{tabular}
    }
    \framebreak

    \hspace{0.05\linewidth}
    \centering{
        \small
        \def\arraystretch{1.8}%
        \begin{tabular}{ll}
            \toprule
            Integralnaya forma                                                                                                                & Differentsialnaya forma                               \\ \midrule
            \(Q_e = \displaystyle\oiint_S \vec D \cdot d\vec{s} = \displaystyle\iiint_V \rho_v dv\)                                           & \(\nabla \cdot \vec D = \rho_v\)                     \\
            \(\displaystyle\oiint_S \vec B \cdot d\vec{s} = 0\)                                                                               & \(\nabla \cdot \vec B = 0\)                          \\
            \(V_{emf} = \displaystyle\oint_L \vec E \cdot d\vec{l}\) = \(- \displaystyle\iint_S \left[j \omega \vec B\right] \cdot d\vec{s}\) & \(\nabla \times \vec E = - j \omega \vec B\)         \\
            \(I = \displaystyle\oint_L \vec H \cdot d\vec{l} = \displaystyle\iint_S \left[\vec J + j \omega \vec D\right] \cdot d\vec{s}\)    & \(\nabla \times \vec H = \vec J + j \omega \vec{D}\) \\ \midrule
            \(\displaystyle\oiint_S \vec J \cdot d\vec{s} = - j \omega Q_e\)                                                                  & \(\nabla \cdot \vec J = - j \omega \rho_v\)          \\
            \bottomrule
            \multicolumn{2}{c}{\(\vec D(t) = \left[\varepsilon\right] \vec E(t)\)}                                                                                                                   \\
            \multicolumn{2}{c}{\(\vec B(t) = \left[\mu\right] \vec H(t)\)}                                                                                                                           \\
        \end{tabular}
    }
\end{frame}

\subsection{Tablitsy}

\begin{frame}
    \frametitle{Tablitsa}
    \centering
    \begin{tabular}{|l|l|}
        \hline
        \textbf{Zagolovok 1} & \textbf{Zagolovok 2} \\
        \hline
        Summa                & \(b+a\)              \\
        \hline
        Raznost             & \(a-b\)              \\
        \hline
        Proizvedenie         & \(a*b\)              \\
        \hline
    \end{tabular}
\end{frame}

\begin{frame}
    \frametitle{Drugaya tablitsa}
    \centering
    \begin{tabular}{lc}
        \toprule
        \multicolumn{1}{c}{\textbf{Zagolovok 1}} & \textbf{Zagolovok 2} \\ \midrule
        Summa                                    & \(b+a\)              \\
        Raznost                                 & \(a-b\)              \\
        Proizvedenie                             & \(a*b\)              \\
        \bottomrule
    \end{tabular}
\end{frame}


\subsection{Raznoe}

\begin{frame}
    \frametitle{Bolshoy mnogourovnevyy spisok}
    \begin{itemize}
        \item \textbf{Punkt 1}
              \begin{itemize}
                  \itemi Podpunkt 1-1
                  \itemi Podpunkt 1-2
              \end{itemize}
        \item \textbf{Punkt 2}
              \begin{itemize}
                  \itemi Podpunkt 2-1
              \end{itemize}
        \item \textbf{Punkt 3}
              \begin{itemize}
                  \itemi Podpunkt 3-1
                  \itemi Podpunkt 3-2
              \end{itemize}
        \item \textbf{Punkt 4}
              \begin{itemize}
                  \itemi Podpunkt 4-1
              \end{itemize}
        \item \textbf{Punkt 5}
              \begin{itemize}
                  \itemi Podpunkt 5-1
                  \itemi Podpunkt 5-2
                  \itemi Podpunkt 5-3
              \end{itemize}
    \end{itemize}
\end{frame}

\begin{frame}
    \frametitle{Chetyre izobrazheniya}
    \centering
    \includegraphics[width=0.35\linewidth,angle=35]{latex}
    \includegraphics[width=0.35\linewidth,angle=135]{latex}\\
    \includegraphics[width=0.35\linewidth,angle=15]{latex}
    \includegraphics[width=0.35\linewidth,angle=-15]{latex}
\end{frame}

