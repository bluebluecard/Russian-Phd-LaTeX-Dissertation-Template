%%% Realizatsiya bibliografii paketami biblatex i biblatex-gost s ispolzovaniem dvizhka biber %%%

\usepackage{csquotes} % biblatex rekomenduet ego podklyuchat. Paket dlya oformleniya slozhnykh blokov tsitirovaniya.
%%% Zagruzka paketa s osnovnymi nastroykami %%%
\makeatletter
\ifnumequal{\value{draft}}{0}{% Chistovik
\usepackage[%
backend=biber,% dvizhok
bibencoding=utf8,% kodirovka bib fayla
sorting=none,% nastroyka sortirovki spiska literatury
style=gost-numeric,% stil tsitirovaniya i bibliografii (po GOST)
language=autobib,% poluchenie yazyka iz babel/polyglossia, default: autobib % esli stavit autocite ili auto, to tsitaty v tekste s ukazaniem stranitsy, poluchat ukazanie stranitsy na yazyke originala
autolang=other,% mnogoyazychnaya bibliografiya
clearlang=true,% vnutrenniy sbros polya language, esli on sovpadaet s yazykom iz babel/polyglossia
defernumbers=true,% numeratsiya prostavlyaetsya posle dvukh kompilyatsiy, zato pozvolyaet vytseplyat bibliografiyu po klyuchevym slovam i numerovat ne iz bolshego spiska
sortcites=true,% sortirovat nomera zatekstovykh ssylok pri tsitirovanii (esli v kvadratnykh skobkakh neskolko ssylok, to otobrazhatsya budut otsortirovanno, a ne aby kak)
doi=false,% Pokazyvat ili net ssylki na DOI
isbn=false,% Pokazyvat ili net ISBN, ISSN, ISRN
]{biblatex}[2016/09/17]
\ltx@iffilelater{biblatex-gost.def}{2017/05/03}%
{\toggletrue{bbx:gostbibliography}%
\renewcommand*{\revsdnamepunct}{\addcomma}}{}
}{%Chernovik
\usepackage[%
backend=biber,% dvizhok
bibencoding=utf8,% kodirovka bib fayla
sorting=none,% nastroyka sortirovki spiska literatury
% defernumbers=true, % otkommentiruyte, esli trebuetsya pravilnaya numeratsiya ssylok na literaturu v rezhime chernovika. Zamedlyaet sborku
]{biblatex}[2016/09/17]%
}
\makeatother

\providebool{blxmc} % biblatex version needs and has MakeCapital workaround
\boolfalse{blxmc} % setting our new boolean flag to default false
\ifxetexorluatex
\else
% Ispravlenie sluchaya nepodderzhki znaka nomera v pdflatex
    \DefineBibliographyStrings{russian}{number={\textnumero}}

% Ispravlenie sluchaya otsutstviya propisnykh bukv v nekotorykh sluchayakh
% https://github.com/plk/biblatex/issues/960#issuecomment-596658282
    \ifdefmacro{\ExplSyntaxOn}{}{\usepackage{expl3}}
    \makeatletter
    \ltx@ifpackagelater{biblatex}{2020/02/23}{
    % Assuming this version of biblatex defines MakeCapital correctly
    }{
        \ltx@ifpackagelater{biblatex}{2019/12/01}{
            % Assuming this version of biblatex defines MakeCapital incorrectly
            \usepackage{expl3}[2020/02/25]
            \@ifpackagelater{expl3}{2020/02/25}{
                \booltrue{blxmc} % setting our new boolean flag to true
            }{}
        }{}
    }
    \makeatother
    \ifblxmc
        \typeout{Assuming this version of biblatex defines MakeCapital
        incorrectly}
        \usepackage{xparse}
        \makeatletter
        \ExplSyntaxOn
        \NewDocumentCommand \blx@maketext@lowercase {m}
          {
            \text_lowercase:n {#1}
          }

        \NewDocumentCommand \blx@maketext@uppercase {m}
          {
            \text_uppercase:n {#1}
          }

        \RenewDocumentCommand \MakeCapital {m}
          {
            \text_titlecase_first:n {#1}
          }
        \ExplSyntaxOff

        \protected\def\blx@biblcstring#1#2#3{%
          \blx@begunit
          \blx@hyphenreset
          \blx@bibstringsimple
          \lowercase{\edef\blx@tempa{#3}}%
          \ifcsundef{#2@\blx@tempa}
            {\blx@warn@nostring\blx@tempa
             \blx@endnounit}
            {#1{\blx@maketext@lowercase{\csuse{#2@\blx@tempa}}}%
             \blx@endunit}}

        \protected\def\blx@bibucstring#1#2#3{%
          \blx@begunit
          \blx@hyphenreset
          \blx@bibstringsimple
          \lowercase{\edef\blx@tempa{#3}}%
          \ifcsundef{#2@\blx@tempa}
            {\blx@warn@nostring\blx@tempa
             \blx@endnounit}
            {#1{\blx@maketext@uppercase{\csuse{#2@\blx@tempa}}}%
             \blx@endunit}}
        \makeatother
    \fi
\fi

\ifsynopsis
\ifnumgreater{\value{usefootcite}}{0}{
    \ExecuteBibliographyOptions{autocite=footnote}
    \newbibmacro*{cite:full}{%
        \printtext[bibhypertarget]{%
            \usedriver{%
                \DeclareNameAlias{sortname}{default}%
            }{%
                \thefield{entrytype}%
            }%
        }%
        \usebibmacro{shorthandintro}%
    }
    \DeclareCiteCommand{\smartcite}[\mkbibfootnote]{%
        \usebibmacro{prenote}%
    }{%
        \usebibmacro{citeindex}%
        \usebibmacro{cite:full}%
    }{%
        \multicitedelim%
    }{%
        \usebibmacro{postnote}%
    }
}{}
\fi

%%% Podklyuchenie faylov bib %%%
\addbibresource[label=bl-external]{biblio/external.bib}
\addbibresource[label=bl-author]{biblio/author.bib}
\addbibresource[label=bl-registered]{biblio/registered.bib}

%http://tex.stackexchange.com/a/141831/79756
%There is a way to automatically map the language field to the langid field. The following lines in the preamble should be enough to do that.
%This command will copy the language field into the langid field and will then delete the contents of the language field. The language field will only be deleted if it was successfully copied into the langid field.
\DeclareSourcemap{ %modifikatsiya bib fayla pered tem, kak im zaymetsya biblatex
    \maps{
        \map{% perekidyvaem znacheniya poley language v polya langid, kotorymi polzuetsya biblatex
            \step[fieldsource=language, fieldset=langid, origfieldval, final]
            \step[fieldset=language, null]
        }
        \map{% perekidyvaem znacheniya poley numpages v polya pagetotal, kotorymi polzuetsya biblatex
            \step[fieldsource=numpages, fieldset=pagetotal, origfieldval, final]
            \step[fieldset=numpages, null]
        }
        \map{% perekidyvaem znacheniya poley pagestotal v polya pagetotal, kotorymi polzuetsya biblatex
            \step[fieldsource=pagestotal, fieldset=pagetotal, origfieldval, final]
            \step[fieldset=pagestotal, null]
        }
        \map[overwrite]{% perekidyvaem znacheniya poley shortjournal, esli oni est, v polya journal, kotorymi polzuetsya biblatex
            \step[fieldsource=shortjournal, final]
            \step[fieldset=journal, origfieldval]
            \step[fieldset=shortjournal, null]
        }
        \map[overwrite]{% perekidyvaem znacheniya poley shortbooktitle, esli oni est, v polya booktitle, kotorymi polzuetsya biblatex
            \step[fieldsource=shortbooktitle, final]
            \step[fieldset=booktitle, origfieldval]
            \step[fieldset=shortbooktitle, null]
        }
        \map{% esli v pole medium napisano "Elektronnyy resurs", to ustanavlivaem pole media, kotorym polzuetsya biblatex, v znachenie eresource.
            \step[fieldsource=medium,
            match=\regexp{Elektronnyy\s+resurs},
            final]
            \step[fieldset=media, fieldvalue=eresource]
            \step[fieldset=medium, null]
        }
        \map[overwrite]{% stiraem znacheniya vsekh poley issn
            \step[fieldset=issn, null]
        }
        \map[overwrite]{% stiraem znacheniya vsekh poley abstract, poskolku imi ne polzuemsya, a tam byvayut "nepriyatnye" latekhu simvoly
            \step[fieldsource=abstract]
            \step[fieldset=abstract,null]
        }
        \map[overwrite]{ % peredelka formata zapisi daty
            \step[fieldsource=urldate,
            match=\regexp{([0-9]{2})\.([0-9]{2})\.([0-9]{4})},
            replace={$3-$2-$1$4}, % $4 vstavlen isklyuchitelno radi normalnoy raboty programm podsvetki sintaksisa, kotorye nekorrektno obrabatyvayut $ v takikh konstruktsiyakh
            final]
        }
        \map[overwrite]{ % stiraem klyuchevye slova
            \step[fieldsource=keywords]
            \step[fieldset=keywords,null]
        }
        % realizatsiya foreach razlichaetsya dlya biblatex v3.12 i v3.13.
        % Dlya versii v3.13 eta konstruktsiya zamenyaet posleduyushchie 7 struktur map
        % \map[overwrite,foreach={authorvak,authorscopus,authorwos,authorconf,authorother,authorparent,authorprogram}]{ % zapisyvaem informatsiyu o tipe publikatsii v klyuchevye slova
        %     \step[fieldsource=$MAPLOOP,final=true]
        %     \step[fieldset=keywords,fieldvalue={,biblio$MAPLOOP},append=true]
        % }
        \map[overwrite]{ % zapisyvaem informatsiyu o tipe publikatsii v klyuchevye slova
            \step[fieldsource=authorvak,final=true]
            \step[fieldset=keywords,fieldvalue={,biblioauthorvak},append=true]
        }
        \map[overwrite]{ % zapisyvaem informatsiyu o tipe publikatsii v klyuchevye slova
            \step[fieldsource=authorscopus,final=true]
            \step[fieldset=keywords,fieldvalue={,biblioauthorscopus},append=true]
        }
        \map[overwrite]{ % zapisyvaem informatsiyu o tipe publikatsii v klyuchevye slova
            \step[fieldsource=authorwos,final=true]
            \step[fieldset=keywords,fieldvalue={,biblioauthorwos},append=true]
        }
        \map[overwrite]{ % zapisyvaem informatsiyu o tipe publikatsii v klyuchevye slova
            \step[fieldsource=authorconf,final=true]
            \step[fieldset=keywords,fieldvalue={,biblioauthorconf},append=true]
        }
        \map[overwrite]{ % zapisyvaem informatsiyu o tipe publikatsii v klyuchevye slova
            \step[fieldsource=authorother,final=true]
            \step[fieldset=keywords,fieldvalue={,biblioauthorother},append=true]
        }
        \map[overwrite]{ % zapisyvaem informatsiyu o tipe publikatsii v klyuchevye slova
            \step[fieldsource=authorpatent,final=true]
            \step[fieldset=keywords,fieldvalue={,biblioauthorpatent},append=true]
        }
        \map[overwrite]{ % zapisyvaem informatsiyu o tipe publikatsii v klyuchevye slova
            \step[fieldsource=authorprogram,final=true]
            \step[fieldset=keywords,fieldvalue={,biblioauthorprogram},append=true]
        }
        \map[overwrite]{ % dobavlyaem klyuchevye slova, chtoby razlichat istochniki
            \perdatasource{biblio/external.bib}
            \step[fieldset=keywords, fieldvalue={,biblioexternal},append=true]
        }
        \map[overwrite]{ % dobavlyaem klyuchevye slova, chtoby razlichat istochniki
            \perdatasource{biblio/author.bib}
            \step[fieldset=keywords, fieldvalue={,biblioauthor},append=true]
        }
        \map[overwrite]{ % dobavlyaem klyuchevye slova, chtoby razlichat istochniki
            \perdatasource{biblio/registered.bib}
            \step[fieldset=keywords, fieldvalue={,biblioregistered},append=true]
        }
        \map[overwrite]{ % dobavlyaem klyuchevye slova, chtoby razlichat istochniki
            \step[fieldset=keywords, fieldvalue={,bibliofull},append=true]
        }
%        \map[overwrite]{% stiraem znacheniya vsekh poley series
%            \step[fieldset=series, null]
%        }
        \map[overwrite]{% perekidyvaem znacheniya poley howpublished v polya organization dlya tipa online
            \step[typesource=online, typetarget=online, final]
            \step[fieldsource=howpublished, fieldset=organization, origfieldval]
            \step[fieldset=howpublished, null]
        }
    }
}

\ifnumequal{\value{mediadisplay}}{1}{
    \DeclareSourcemap{
        \maps{%
            \map{% ispolzovanie media=text po umolchaniyu
                \step[fieldset=media, fieldvalue=text]
            }
        }
    }
}{}
\ifnumequal{\value{mediadisplay}}{2}{
    \DeclareSourcemap{
        \maps{%
            \map[overwrite]{% udalenie vsekh zapisey media
                \step[fieldset=media, null]
            }
        }
    }
}{}
\ifnumequal{\value{mediadisplay}}{3}{
    \DeclareSourcemap{
        \maps{
            \map[overwrite]{% stiraem znacheniya vsekh poley media=text
                \step[fieldsource=media,match={text},final]
                \step[fieldset=media, null]
            }
        }
    }
}{}
\ifnumequal{\value{mediadisplay}}{4}{
    \DeclareSourcemap{
        \maps{
            \map[overwrite]{% stiraem znacheniya vsekh poley media=eresource
                \step[fieldsource=media,match={eresource},final]
                \step[fieldset=media, null]
            }
        }
    }
}{}

\ifsynopsis
\else
\DeclareSourcemap{ %modifikatsiya bib fayla pered tem, kak im zaymetsya biblatex
    \maps{
        \map[overwrite]{% stiraem znacheniya vsekh poley addendum
            \perdatasource{biblio/author.bib}
            \step[fieldset=addendum, null] %chtoby izbavitsya ot informatsii ob obeme avtorskikh statey, v otlichie ot avtoreferata
        }
    }
}
\fi

\ifpresentation
% udalyaem lishnie polya v spiske literatury prezentatsii
% ikh nazvaniya mozhno uznat v fayle presentation.bbl
\DeclareSourcemap{
    \maps{
    \map[overwrite,foreach={%
        % {{{ Spisok lishnikh poley v prezentatsii
        address,%
        chapter,%
        edition,%
        editor,%
        eid,%
        howpublished,%
        institution,%
        key,%
        month,%
        note,%
        number,%
        organization,%
        pages,%
        publisher,%
        school,%
        series,%
        type,%
        media,%
        url,%
        doi,%
        location,%
        volume,%
        % Spisok lishnikh poley v prezentatsii }}}
    }]{
        \perdatasource{biblio/author.bib}
        \step[fieldset=$MAPLOOP,null]
    }
    }
}
\fi

\defbibfilter{vakscopuswos}{%
    keyword=biblioauthorvak or keyword=biblioauthorscopus or keyword=biblioauthorwos
}

\defbibfilter{scopuswos}{%
    keyword=biblioauthorscopus or keyword=biblioauthorwos
}

\defbibfilter{papersregistered}{%
    keyword=biblioauthor or keyword=biblioregistered
}

%%% Ubiraem nerazryvnye probely pered dvoetochiem i tochkoy s zapyatoy %%%
%\makeatletter
%\ifnumequal{\value{draft}}{0}{% Chistovik
%    \renewcommand*{\addcolondelim}{%
%      \begingroup%
%      \def\abx@colon{%
%        \ifdim\lastkern>\z@\unkern\fi%
%        \abx@puncthook{:}\space}%
%      \addcolon%
%      \endgroup}
%
%    \renewcommand*{\addsemicolondelim}{%
%      \begingroup%
%      \def\abx@semicolon{%
%        \ifdim\lastkern>\z@\unkern\fi%
%        \abx@puncthook{;}\space}%
%      \addsemicolon%
%      \endgroup}
%}{}
%\makeatother

%%% Pravka zapisey tipa thesis, chtoby dvazhdy ne pisalsya avtor
%\ifnumequal{\value{draft}}{0}{% Chistovik
%\DeclareBibliographyDriver{thesis}{%
%  \usebibmacro{bibindex}%
%  \usebibmacro{begentry}%
%  \usebibmacro{heading}%
%  \newunit
%  \usebibmacro{author}%
%  \setunit*{\labelnamepunct}%
%  \usebibmacro{thesistitle}%
%  \setunit{\respdelim}%
%  %\printnames[last-first:full]{author}%Vot etu strochku nuzhno ubrat, chtoby avtor dissertatsii ne dublirovalsya
%  \newunit\newblock
%  \printlist[semicolondelim]{specdata}%
%  \newunit
%  \usebibmacro{institution+location+date}%
%  \newunit\newblock
%  \usebibmacro{chapter+pages}%
%  \newunit
%  \printfield{pagetotal}%
%  \newunit\newblock
%  \usebibmacro{doi+eprint+url+note}%
%  \newunit\newblock
%  \usebibmacro{addendum+pubstate}%
%  \setunit{\bibpagerefpunct}\newblock
%  \usebibmacro{pageref}%
%  \newunit\newblock
%  \usebibmacro{related:init}%
%  \usebibmacro{related}%
%  \usebibmacro{finentry}}
%}{}

%\newbibmacro{string+doi}[1]{% novaya makrokomanda na prostanovku ssylki na doi
%    \iffieldundef{doi}{#1}{\href{http://dx.doi.org/\thefield{doi}}{#1}}}

%\ifnumequal{\value{draft}}{0}{% Chistovik
%\renewcommand*{\mkgostheading}[1]{\usebibmacro{string+doi}{#1}} % ssylka na doi s avtorov. stoyashchikh vperedi zapisi
%\renewcommand*{\mkgostheading}[1]{#1} % tolko lish ubiraem kursiv s avtorov
%}{}
%\DeclareFieldFormat{title}{\usebibmacro{string+doi}{#1}} % ssylka na doi s nazvaniya raboty
%\DeclareFieldFormat{journaltitle}{\usebibmacro{string+doi}{#1}} % ssylka na doi s nazvaniya zhurnala
%%% Tire kak razdelitel v bibliografii traditsionnoy ruskoy dliny:
\renewcommand*{\newblockpunct}{\addperiod\addnbspace\cyrdash\space\bibsentence}
%%% Ubrat tire iz razdeliteley elementov v bibliografii:
%\renewcommand*{\newblockpunct}{%
%    \addperiod\space\bibsentence}%block punct.,\bibsentence is for vol,etc.
%%% Izmenenie tochki s zapyatoy na zapyatuyu v perechislenii bibliograficheskikh
%%% ssylok:
%\renewcommand*{\multicitedelim}{\addcomma\space}

%%% Vozvrashchaem zapis «Rezhim dostupa» %%%
%\DefineBibliographyStrings{english}{%
%    urlfrom = {Mode of access}
%}
%\DeclareFieldFormat{url}{\bibstring{urlfrom}\addcolon\space\url{#1}}

%%% V spiske literatury oboznachenie odnoy bukvoy diapazona stranits angloyazychnogo istochnika %%%
\DefineBibliographyStrings{english}{%
    pages = {p\adddot} %zaglavnost bukvy zatem po mestu opredelyaetsya rabotoy samogo biblatex
}

%%% V ssylke na istochnik v osnovnom tekste s ukazaniem konkretnoy stranitsy oboznachenie odnoy bolshoy bukvoy %%%
%\DefineBibliographyStrings{russian}{%
%    page = {C\adddot}
%}

%%% Ispravlenie dliny tire v diapazonakh %%%
% \cyrdash --- tire «russkoy» dliny, \textendash --- en-dash
\DefineBibliographyExtras{russian}{%
  \protected\def\bibrangedash{%
    \cyrdash\penalty\value{abbrvpenalty}}% almost unbreakable dash
  \protected\def\bibdaterangesep{\bibrangedash}%tire dlya dat
}
\DefineBibliographyExtras{english}{%
  \protected\def\bibrangedash{%
    \cyrdash\penalty\value{abbrvpenalty}}% almost unbreakable dash
  \protected\def\bibdaterangesep{\bibrangedash}%tire dlya dat
}

%Set higher penalty for breaking in number, dates and pages ranges
\setcounter{abbrvpenalty}{10000} % default is \hyphenpenalty which is 12

%Set higher penalty for breaking in names
\setcounter{highnamepenalty}{10000} % If you prefer the traditional BibTeX behavior (no linebreaks at highnamepenalty breakpoints), set it to ‘infinite’ (10 000 or higher).
\setcounter{lownamepenalty}{10000}

%%% Set low penalties for breaks at uppercase letters and lowercase letters
%\setcounter{biburllcpenalty}{500} %upravlyaet razryvami ssylok posle malenkikh bukv RTFM biburllcpenalty
%\setcounter{biburlucpenalty}{3000} %upravlyaet razryvami ssylok posle bolshikh bukv, RTFM biburlucpenalty

%%% Spisok literatury s krasnoy stroki (bez visyachego otstupa) %%%
%\defbibenvironment{bibliography} % pereopredelyaem okruzhenie bibliografii iz gost-numeric.bbx paketa biblatex-gost
%  {\list
%     {\printtext[labelnumberwidth]{%
%       \printfield{prefixnumber}%
%       \printfield{labelnumber}}}
%     {%
%      \setlength{\labelwidth}{\labelnumberwidth}%
%      \setlength{\leftmargin}{0pt}% default is \labelwidth
%      \setlength{\labelsep}{\widthof{\ }}% Upravlyaet dlinoy otstupa posle tochki % default is \biblabelsep
%      \setlength{\itemsep}{\bibitemsep}% Upravlenie dopolnitelnym vertikalnym razryvom mezhdu zapisyami. \bibitemsep po umolchaniyu sootvetstvuet \itemsep spiskov v dokumente.
%      \setlength{\itemindent}{\bibhang}% Polzuemsya tem, chto \bibhang po umolchaniyu prinimaet znachenie \parindent (abzatsnogo otstupa), kotoryy perenaznachen v styles.tex
%      \addtolength{\itemindent}{\labelwidth}% Sdvigaem pravee na velichinu nomera s tochkoy
%      \addtolength{\itemindent}{\labelsep}% Sdvigaem eshche pravee na otstup posle tochki
%      \setlength{\parsep}{\bibparsep}%
%     }%
%      \renewcommand*{\makelabel}[1]{\hss##1}%
%  }
%  {\endlist}
%  {\item}

%%% Makrosy avtomaticheskogo podscheta kolichestva avtorskikh publikatsiy.
% Pechatayut nevidimuyu (pustuyu) bibliografiyu, schitaya kolichestvo istochnikov.
% http://tex.stackexchange.com/a/66851/79756
%
\makeatletter
    \newtotcounter{citenum}
    \defbibenvironment{counter}
        {\setcounter{citenum}{0}\renewcommand{\blx@driver}[1]{}} % begin code: ubiraet ves vyvodimyy tekst
        {} % end code
        {\stepcounter{citenum}} % item code: cchitaet "pechataemye v bibliografiyu" istochniki

    \newtotcounter{citeauthorvak}
    \defbibenvironment{countauthorvak}
        {\setcounter{citeauthorvak}{0}\renewcommand{\blx@driver}[1]{}}
        {}
        {\stepcounter{citeauthorvak}}

    \newtotcounter{citeauthorscopus}
    \defbibenvironment{countauthorscopus}
        {\setcounter{citeauthorscopus}{0}\renewcommand{\blx@driver}[1]{}}
        {}
        {\stepcounter{citeauthorscopus}}

    \newtotcounter{citeauthorwos}
    \defbibenvironment{countauthorwos}
        {\setcounter{citeauthorwos}{0}\renewcommand{\blx@driver}[1]{}}
        {}
        {\stepcounter{citeauthorwos}}

    \newtotcounter{citeauthorother}
    \defbibenvironment{countauthorother}
        {\setcounter{citeauthorother}{0}\renewcommand{\blx@driver}[1]{}}
        {}
        {\stepcounter{citeauthorother}}

    \newtotcounter{citeauthorconf}
    \defbibenvironment{countauthorconf}
        {\setcounter{citeauthorconf}{0}\renewcommand{\blx@driver}[1]{}}
        {}
        {\stepcounter{citeauthorconf}}

    \newtotcounter{citeauthor}
    \defbibenvironment{countauthor}
        {\setcounter{citeauthor}{0}\renewcommand{\blx@driver}[1]{}}
        {}
        {\stepcounter{citeauthor}}

    \newtotcounter{citeauthorvakscopuswos}
    \defbibenvironment{countauthorvakscopuswos}
        {\setcounter{citeauthorvakscopuswos}{0}\renewcommand{\blx@driver}[1]{}}
        {}
        {\stepcounter{citeauthorvakscopuswos}}

    \newtotcounter{citeauthorscopuswos}
    \defbibenvironment{countauthorscopuswos}
        {\setcounter{citeauthorscopuswos}{0}\renewcommand{\blx@driver}[1]{}}
        {}
        {\stepcounter{citeauthorscopuswos}}

    \newtotcounter{citeregistered}
    \defbibenvironment{countregistered}
        {\setcounter{citeregistered}{0}\renewcommand{\blx@driver}[1]{}}
        {}
        {\stepcounter{citeregistered}}

    \newtotcounter{citeauthorpatent}
    \defbibenvironment{countauthorpatent}
        {\setcounter{citeauthorpatent}{0}\renewcommand{\blx@driver}[1]{}}
        {}
        {\stepcounter{citeauthorpatent}}

    \newtotcounter{citeauthorprogram}
    \defbibenvironment{countauthorprogram}
        {\setcounter{citeauthorprogram}{0}\renewcommand{\blx@driver}[1]{}}
        {}
        {\stepcounter{citeauthorprogram}}

    \newtotcounter{citeexternal}
    \defbibenvironment{countexternal}
        {\setcounter{citeexternal}{0}\renewcommand{\blx@driver}[1]{}}
        {}
        {\stepcounter{citeexternal}}
\makeatother

\defbibheading{nobibheading}{} % pustoy zagolovok, dlya podscheta publikatsiy s pomoshchyu nevidimoy bibliografii
\defbibheading{pubgroup}{\section*{#1}} % obychnyy stil, zagolovok-sektsiya
\defbibheading{pubsubgroup}{\noindent\textbf{#1}} % dlya podrazdelov "po tipu istochnika"

%%%Sortirovka spiska literatury Russkiy-Angliyskiy (predvaritelno udalit dissertation.bbl) (nachalo)
%%%Istochnik: https://github.com/odomanov/biblatex-gost/wiki/%D0%9A%D0%B0%D0%BA-%D1%81%D0%B4%D0%B5%D0%BB%D0%B0%D1%82%D1%8C,-%D1%87%D1%82%D0%BE%D0%B1%D1%8B-%D1%80%D1%83%D1%81%D1%81%D0%BA%D0%BE%D1%8F%D0%B7%D1%8B%D1%87%D0%BD%D1%8B%D0%B5-%D0%B8%D1%81%D1%82%D0%BE%D1%87%D0%BD%D0%B8%D0%BA%D0%B8-%D0%BF%D1%80%D0%B5%D0%B4%D1%88%D0%B5%D1%81%D1%82%D0%B2%D0%BE%D0%B2%D0%B0%D0%BB%D0%B8-%D0%BE%D1%81%D1%82%D0%B0%D0%BB%D1%8C%D0%BD%D1%8B%D0%BC
%\DeclareSourcemap{
%    \maps[datatype=bibtex]{
%        \map{
%            \step[fieldset=langid, fieldvalue={tempruorder}]
%        }
%        \map[overwrite]{
%            \step[fieldsource=langid, match=russian, final]
%            \step[fieldsource=presort,
%            match=\regexp{(.+)},
%            replace=\regexp{aa$1}]
%        }
%        \map{
%            \step[fieldsource=langid, match=russian, final]
%            \step[fieldset=presort, fieldvalue={az}]
%        }
%        \map[overwrite]{
%            \step[fieldsource=langid, notmatch=russian, final]
%            \step[fieldsource=presort,
%            match=\regexp{(.+)},
%            replace=\regexp{za$1}]
%        }
%        \map{
%            \step[fieldsource=langid, notmatch=russian, final]
%            \step[fieldset=presort, fieldvalue={zz}]
%        }
%        \map{
%            \step[fieldsource=langid, match={tempruorder}, final]
%            \step[fieldset=langid, null]
%        }
%    }
%}
%Sortirovka spiska literatury (konets)

%%% Sozdanie komand dlya vyvoda spiska literatury %%%
\newcommand*{\insertbibliofull}{
    \printbibliography[keyword=bibliofull,section=0,title=\bibtitlefull]
    \ifnumequal{\value{draft}}{0}{
      \printbibliography[heading=nobibheading,env=counter,keyword=bibliofull,section=0]
    }{}
}
\newcommand*{\insertbiblioauthor}{
    \printbibliography[heading=pubgroup, section=0, filter=papersregistered, title=\bibtitleauthor]
}
\newcommand*{\insertbiblioauthorimportant}{
    \printbibliography[heading=pubgroup, section=2, filter=papersregistered, title=\bibtitleauthorimportant]
}

% Variant vyvoda pechatnykh rabot avtora, s gruppirovkoy po tipu istochnika.
% Poryadok komand `\printbibliography` dolzhen sootvetstvovat poryadku v fayle common/characteristic.tex
\newcommand*{\insertbiblioauthorgrouped}{
    \section*{\bibtitleauthor}
    \ifsynopsis
    \printbibliography[heading=pubsubgroup, section=0, keyword=biblioauthorvak,    title=\bibtitleauthorvak,resetnumbers=true] % Raboty avtora iz spiska VAK (sbros numeratsii)
    \else
    \printbibliography[heading=pubsubgroup, section=0, keyword=biblioauthorvak,    title=\bibtitleauthorvak,resetnumbers=false] % Raboty avtora iz spiska VAK (skvoznaya numeratsiya)
    \fi
    \printbibliography[heading=pubsubgroup, section=0, keyword=biblioauthorwos,    title=\bibtitleauthorwos,resetnumbers=false]% Raboty avtora, indeksiruemye Web of Science
    \printbibliography[heading=pubsubgroup, section=0, keyword=biblioauthorscopus, title=\bibtitleauthorscopus,resetnumbers=false]% Raboty avtora, indeksiruemye Scopus
    \printbibliography[heading=pubsubgroup, section=0, keyword=biblioauthorpatent, title=\bibtitleauthorpatent,resetnumbers=false]% Patenty
    \printbibliography[heading=pubsubgroup, section=0, keyword=biblioauthorprogram,title=\bibtitleauthorprogram,resetnumbers=false]% Programmy dlya EVM
    \printbibliography[heading=pubsubgroup, section=0, keyword=biblioauthorconf,   title=\bibtitleauthorconf,resetnumbers=false]% Tezisy konferentsiy
    \printbibliography[heading=pubsubgroup, section=0, keyword=biblioauthorother,  title=\bibtitleauthorother,resetnumbers=false]% Prochie raboty avtora
}

\newcommand*{\insertbiblioexternal}{
    \printbibliography[heading=pubgroup,    section=0, keyword=biblioexternal,     title=\bibtitlefull]
}
