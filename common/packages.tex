%%% Proverka ispolzuemogo TeX-dvizhka %%%
\newif\ifxetexorluatex   % opredelyaem novyy uslovnyy operator (http://tex.stackexchange.com/a/47579)
\ifxetex
    \xetexorluatextrue
\else
    \ifluatex
        \xetexorluatextrue
    \else
        \xetexorluatexfalse
    \fi
\fi

\newif\ifsynopsis           % Uslovie, proveryayushchee, chto dokument --- avtoreferat

\usepackage{etoolbox}[2015/08/02]   % Dlya prodvinutoy proverki raznykh usloviy
\providebool{presentation}

\usepackage{comment}    % Pozvolyaet ubirat bloki teksta (dobavlyaet
                        % okruzhenie comment i komandu \excludecomment)

%%% Polya i razmetka stranitsy %%%
\usepackage{pdflscape}  % Dlya vklyucheniya albomnykh stranits
\usepackage{geometry}   % Dlya posleduyushchego zadaniya poley

%%% Matematicheskie pakety %%%
\usepackage{amsthm,amsmath,amscd}   % Matematicheskie dopolneniya ot AMS
\usepackage{amsfonts,amssymb}       % Matematicheskie dopolneniya ot AMS
\usepackage{mathtools}              % Dobavlyaet okruzhenie multlined
\usepackage{xfrac}                  % Krasivye drobi
\usepackage[
    locale = DE,
    list-separator       = {;\,},
    list-final-separator = {;\,},
    list-pair-separator  = {;\,},
    list-units           = single,
    range-units          = single,
    range-phrase={\text{\ensuremath{-}}},
    % quotient-mode        = fraction, % krasivye drobi mogut ne sootvetstvovat GOST
    fraction-function    = \sfrac,
    separate-uncertainty,
    ]{siunitx}[=v2]                 % Razmernosti SI
\sisetup{inter-unit-product = \ensuremath{{}\cdot{}}}

% Kirillitsa v numeratsii subequations
% Dlya pravilnoy raboty trebuetsya vypolnenie srazu posle zagruzki paketov
\patchcmd{\subequations}{\def\theequation{\theparentequation\alph{equation}}}
{\def\theequation{\theparentequation\asbuk{equation}}}
{\typeout{subequations patched}}{\typeout{subequations not patched}}

%%%% Ustanovki dlya razmera shrifta 14 pt %%%%
%% Formirovanie peremennykh i konstant dlya sravneniya (odin raz dlya vsekh podklyuchaemykh faylov)%%
%% dolzhno raspolagatsya do vyzova paketa fontspec ili polyglossia, potomu chto oni sbivayut ego rabotu
\newlength{\curtextsize}
\newlength{\bigtextsize}
\setlength{\bigtextsize}{13.9pt}

\makeatletter
%\show\f@size    % neplokho dlya otslezhivaniya, no vyzyvaet stoporenie protsessa,
                 % esli dokument kompiliruetsya bez komandy  -interaction=nonstopmode
\setlength{\curtextsize}{\f@size pt}
\makeatother

%%% Kodirovki i shrifty %%%
\ifxetexorluatex
    \ifpresentation
        \providecommand*\autodot{} % quick fix for polyglossia 1.50
    \fi
    \PassOptionsToPackage{no-math}{fontspec}    % https://tex.stackexchange.com/a/26295/104425
    \usepackage{polyglossia}[2014/05/21]        % Podderzhka mnogoyazychnosti
                                        % (fontspec podgruzhaetsya avtomaticheski)
\else
   %%% Reshenie problemy kopirovaniya teksta v bufer krakozyabrami
    \ifnumequal{\value{usealtfont}}{0}{}{
        \input glyphtounicode.tex
        \input glyphtounicode-cmr.tex %from pdfx package
        \pdfgentounicode=1
    }
    \usepackage{cmap}   % Uluchshennyy poisk russkikh slov v poluchennom pdf-fayle
    \ifnumequal{\value{usealtfont}}{2}{}{
        \defaulthyphenchar=127  % Esli stoit do fontenc, to perenosy
                                % ne vpishutsya v vydelyaemyy tekst pri
                                % kopirovanii ego v bufer obmena
    }
    \usepackage{textcomp}
    \usepackage[T1,T2A]{fontenc}                    % Podderzhka russkikh bukv
    \ifnumequal{\value{usealtfont}}{1}{% Ispolzuetsya pscyr, pri nalichii
        \IfFileExists{pscyr.sty}{\usepackage{pscyr}}{}  % Podklyuchenie pscyr
    }{}
    \usepackage[utf8]{inputenc}[2014/04/30]         % Kodirovka utf8
    \usepackage[english, russian]{babel}[2014/03/24]% Yazyki: russkiy, angliyskiy
    \makeatletter\AtBeginDocument{\let\@elt\relax}\makeatother % babel 3.40 fix
    \ifnumequal{\value{usealtfont}}{2}{
        % http://dxdy.ru/post1238763.html#p1238763
        \usepackage[scaled=0.914]{XCharter}[2017/12/19] % Podklyuchenie rusifitsirovannykh shriftov XCharter
        \usepackage[charter, vvarbb, scaled=1.048]{newtxmath}[2017/12/14]
        \ifpresentation
        \else
            \setDisplayskipStretch{-0.078}
        \fi
    }{}
\fi

%%% Oformlenie abzatsev %%%
\ifpresentation
\else
    \indentafterchapter     % Krasnaya stroka posle zagolovkov tipa chapter
    \usepackage{indentfirst}
\fi

%%% Tsveta %%%
\ifpresentation
\else
    \usepackage[dvipsnames, table, hyperref]{xcolor} % Sovmestimo s tikz
\fi

%%% Tablitsy %%%
\usepackage{longtable} % Dlinnye tablitsy
\usepackage{multirow,makecell}   % Uluchshennoe formatirovanie tablits
\usepackage{tabulary,tabularray} % Tablitsy s avtomaticheski podbirayushcheysya
                                 % shirinoy stolbtsov
\UseTblrLibrary{booktabs}
\ExplSyntaxOn% define \IfTokenListEmpty to use \captionof with tabularray
\prg_generate_conditional_variant:Nnn \tl_if_empty:n { e } { TF }
\let \IfTokenListEmpty = \tl_if_empty:eTF
\ExplSyntaxOff

\usepackage{threeparttable}      % avtomaticheskiy podgon shiriny podpisi tablitsy

%%% Obshchee formatirovanie
%\usepackage{soul}% Podderzhka perenosoustoychivykh podcherkivaniy i zacherkivaniy
\usepackage{icomma}  % Zapyataya v desyatichnykh drobyakh

%%% Optimizatsiya rasstanovki perenosov i dliny posledney stroki abzatsa
\IfFileExists{impnattypo.sty}{% proverka ustanovlennosti paketa impnattypo
    \ifluatex
        \ifnumequal{\value{draft}}{1}{% Chernovik
            \usepackage[hyphenation, lastparline, nosingleletter, homeoarchy,
            rivers, draft]{impnattypo}
        }{% Chistovik
            \usepackage[hyphenation, lastparline, nosingleletter]{impnattypo}
        }
    \else
        \usepackage[hyphenation, lastparline]{impnattypo}
    \fi
}{}

%% Vektornaya grafika

\usepackage{tikz}                   % Prodvinutyy paket vektornoy grafiki
\usetikzlibrary{chains}             % Dlya primera tikz risunka
\usetikzlibrary{shapes.geometric}   % Dlya primera tikz risunka
\usetikzlibrary{shapes.symbols}     % Dlya primera tikz risunka
\usetikzlibrary{arrows}             % Dlya primera tikz risunka

\usepackage[european,cuteinductors]{circuitikz} % Elektricheskie skhemy
\usepackage{pgfplots}                           % Grafiki
\pgfplotsset{compat=newest}
\usepgfplotslibrary{groupplots,units}
\pgfkeys{/pgf/number format/.cd,use comma,1000 sep={}} % formatirovanie chisel v grafikakh

%%% Giperssylki %%%
\ifxetexorluatex
    \let\CYRDZE\relax
\fi
\usepackage{hyperref}[2012/11/06]

%%% Izobrazheniya %%%
\usepackage{graphicx}[2014/04/25]   % Podklyuchaem paket raboty s grafikoy
\usepackage{caption}                % Podpisi risunkov i tablits
\usepackage{subcaption}             % Podpisi podrisunkov i podtablits
\usepackage{pdfpages}               % Dobavlenie vneshnikh pdf faylov

%%% Schetchiki %%%
\usepackage{aliascnt}
\usepackage[figure,table]{totalcount}   % Schetchik risunkov i tablits
\usepackage{totcount}   % Paket sozdaniya schetchikov na osnove poslednego nomera
                        % podschityvaemogo elementa (mozhet trebovat dvazhdy
                        % kompilirovat dokument)
\usepackage{totpages}   % Schetchik stranits, sovmestimyy s hyperref (ssylaetsya
                        % na nomer posledney stranitsy). Zhelatelno stavit
                        % poslednim paketom v preambule

%%% Prodvinutoe upravlenie gruppovymi ssylkami (poka tolko formulami) %%%
\ifpresentation
\else
    \usepackage[russian]{cleveref} % cleveref imeet slozhnosti so schityvaniem
    % yazyka iz babel. Takoe reshenie rusifikatsii vyvoda vybrano vmesto
    % opredeleniya v documentclass iz opasnosti chto-to lishnee peredat vo vse
    % ostalnye pakety, vklyuchaya bibliografiyu.

    % Dobavlenie vozmozhnosti ispolzovaniya probelov v \labelcref
    % https://tex.stackexchange.com/a/340502/104425
    \usepackage{kvsetkeys}
    \makeatletter
    \let\org@@cref\@cref
    \renewcommand*{\@cref}[2]{%
        \edef\process@me{%
            \noexpand\org@@cref{#1}{\zap@space#2 \@empty}%
        }\process@me
    }
    \makeatother
\fi

\usepackage{placeins} % dlya \FloatBarrier

\ifnumequal{\value{draft}}{1}{% Chernovik
    \usepackage[firstpage]{draftwatermark}
    \SetWatermarkText{DRAFT}
    \SetWatermarkFontSize{14pt}
    \SetWatermarkScale{15}
    \SetWatermarkAngle{45}
}{}

%%% Tsitata, ne privodimaya v avtoreferate:
% vozmozhno, aktualna tolko dlya biblatex
%\newcommand{\citeinsynopsis}[1]{\ifsynopsis\else ~\cite{#1} \fi}

% esli tekushchiy protsess zapushchen bibliotekoy tikz-external, to prekompilyatsiya dolzhna byt vklyuchena
\ifdefined\tikzexternalrealjob
    \setcounter{imgprecompile}{1}
\fi

\ifnumequal{\value{imgprecompile}}{1}{% Tolko esli u nas vklyuchena predkompilyatsiya
    \usetikzlibrary{external}   % podklyuchenie vozmozhnosti predkompilyatsii
    \tikzexternalize[prefix=images/cache/,optimize command away=\includepdf] % activate! % zdes mozhno ukazat otdelnuyu papku dlya skompilirovannykh faylov
    \ifxetex
        \tikzset{external/up to date check={diff}}
    \fi
}{}
