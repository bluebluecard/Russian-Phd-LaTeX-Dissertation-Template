%%%%%%%%%%%%%%%%%%%%%%%%%%%%%%%%%%%%%%%%%%%%%%%%%%%%%%%%%%%%%%%%%%%%%%%%%%%%%%%%
%%%% Fayl uproshchennykh nastroek shablona, obshchikh dlya dissertatsii i avtoreferata %%%%
%%%%%%%%%%%%%%%%%%%%%%%%%%%%%%%%%%%%%%%%%%%%%%%%%%%%%%%%%%%%%%%%%%%%%%%%%%%%%%%%

%%% Rezhim chernovika %%%
\makeatletter
\@ifundefined{c@draft}{
  \newcounter{draft}
  \setcounter{draft}{0}  % 0 --- chistovik (maksimalnoe soblyudenie GOST)
                         % 1 --- chernovik (otkloneniya ot GOST, no bystraya
                         %       sborka itogovykh PDF)
}{}
\makeatother

%%% Pometki v tekste %%%
\makeatletter
\@ifundefined{c@showmarkup}{
  \newcounter{showmarkup}
  \setcounter{showmarkup}{0}  % 0 --- skryt pometki
                              % 1 --- pokazyvat pometki
}{}
\makeatother

%%% Ispolzovanie v pdflatex shriftov ne po-umolchaniyu %%%
\makeatletter
\@ifundefined{c@usealtfont}{
  \newcounter{usealtfont}
  \setcounter{usealtfont}{1}    % 0 --- shrifty na baze Computer Modern
                                % 1 --- ispolzovat paket pscyr, pri ego
                                %       nalichii
                                % 2 --- ispolzovat paket XCharter, pri nalichii
                                %       podkhodyashchey versii
}{}
\makeatother

%%% Ispolzovanie v xelatex i lualatex semeystv shriftov %%%
\makeatletter
\@ifundefined{c@fontfamily}{
  \newcounter{fontfamily}
  \setcounter{fontfamily}{1}  % 0 --- CMU semeystvo. Ispolzuetsya kak fallback;
                              % 1 --- Shrifty ot MS (Times New Roman i kompaniya)
                              % 2 --- Semeystvo Liberation
}{}
\makeatother

%%% Bibliografiya %%%
\makeatletter
\@ifundefined{c@bibliosel}{
  \newcounter{bibliosel}
  \setcounter{bibliosel}{1}   % 0 --- vstroennaya realizatsiya s zagruzkoy fayla
                              %       cherez dvizhok bibtex8;
                              % 1 --- realizatsiya paketom biblatex cherez dvizhok
                              %       biber
}{}
\makeatother

%%% Vyvod tipov ssylok v bibliografii %%%
\makeatletter
\@ifundefined{c@mediadisplay}{
  \newcounter{mediadisplay}
  \setcounter{mediadisplay}{1}   % 0 --- ne delat nichego; nadpisi [Tekst] i
                                 %       [El. resurs] budut vyvoditsya tolko v ssylkakh s
                                 %       zapolnennym polem `media`;
                                 % 1 --- avtomaticheski dobavlyat nadpis [Tekst] k ssylkam s
                                 %       nezapolnennym polem `media`; takim obrazom, u vsekh
                                 %       istochnikov budet ukazan tip, chto sootvetstvuet
                                 %       trebovaniyam GOST
                                 % 2 --- avtomaticheski udalyat nadpisi [Tekst], [El. Resurs] i dr.;
                                 %       ne sootvetstvuet GOST
                                 % 3 --- avtomaticheski udalyat nadpis [Tekst];
                                 %       ne sootvetstvuet GOST
                                 % 4 --- avtomaticheski udalyat nadpis [El. Resurs];
                                 %       ne sootvetstvuet GOST
}{}
\makeatother

%%% Predkompilyatsiya tikz risunkov dlya uskoreniya raboty %%%
\makeatletter
\@ifundefined{c@imgprecompile}{
  \newcounter{imgprecompile}
  \setcounter{imgprecompile}{0}   % 0 --- bez predkompilyatsii;
                                  % 1 --- polzovatsya predvaritelno
                                  %       skompilirovannymi pdf vmesto generatsii
                                  %       zanovo iz tikz
}{}
\makeatother
