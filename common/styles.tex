%%% Shablon %%%
\DeclareRobustCommand{\fixme}{\textcolor{red}}  % reshaem problemu prevrashcheniya
                                % nazvaniya tsveta v rezultate \MakeUppercase,
                                % http://tex.stackexchange.com/a/187930,
                                % \DeclareRobustCommand protects \fixme
                                % from expanding inside \MakeUppercase
\AtBeginDocument{%
    \setlength{\parindent}{2.5em}                   % Abzatsnyy otstup. Dolzhen byt odinakovym po vsemu tekstu i raven pyati znakam (GOST R 7.0.11-2011, 5.3.7).
}

%%% Tablitsy %%%
\DeclareCaptionLabelSeparator{tabsep}{\tablabelsep} % numeratsiya tablits
\DeclareCaptionFormat{split}{\splitformatlabel#1\par\splitformattext#3}

\captionsetup[table]{
        format=\tabformat,                % format podpisi (plain|hang)
        font=normal,                      % normalnye razmer, tsvet, stil shrifta
        skip=.0pt,                        % otbivka pod podpisyu
        parskip=.0pt,                     % otbivka mezhdu paragrafami podpisi
        position=above,                   % polozhenie podpisi
        justification=\tabjust,           % tsentrovka
        indent=\tabindent,                % smeshchenie strok posle pervoy
        labelsep=tabsep,                  % razdelitel
        singlelinecheck=\tabsinglecenter, % ne vyravnivat po tsentru, esli umeshchaetsya v odnu stroku
}

%%% Risunki %%%
\DeclareCaptionLabelSeparator{figsep}{\figlabelsep} % numeratsiya risunkov

\captionsetup[figure]{
        format=plain,                     % format podpisi (plain|hang)
        font=normal,                      % normalnye razmer, tsvet, stil shrifta
        skip=.0pt,                        % otbivka pod podpisyu
        parskip=.0pt,                     % otbivka mezhdu paragrafami podpisi
        position=below,                   % polozhenie podpisi
        singlelinecheck=true,             % vyravnivanie po tsentru, esli umeshchaetsya v odnu stroku
        justification=centerlast,         % tsentrovka
        labelsep=figsep,                  % razdelitel
}

%%% Podpisi podrisunkov %%%
\DeclareCaptionSubType{figure}
\renewcommand\thesubfigure{\asbuk{subfigure}} % numeratsiya podrisunkov
\ifsynopsis
\DeclareCaptionFont{norm}{\fontsize{10pt}{11pt}\selectfont}
\newcommand{\subfigureskip}{2.pt}
\else
\DeclareCaptionFont{norm}{\fontsize{14pt}{16pt}\selectfont}
\newcommand{\subfigureskip}{0.pt}
\fi

\captionsetup[subfloat]{
        labelfont=norm,                 % normalnyy razmer podpisey podrisunkov
        textfont=norm,                  % normalnyy razmer podpisey podrisunkov
        labelsep=space,                 % razdelitel
        labelformat=brace,              % odna skobka sprava ot nomera
        justification=centering,        % tsentrovka
        singlelinecheck=true,           % vyravnivanie po tsentru, esli umeshchaetsya v odnu stroku
        skip=\subfigureskip,            % otbivka nad podpisyu
        parskip=.0pt,                   % otbivka mezhdu paragrafami podpisi
        position=below,                 % polozhenie podpisi
}

%%% Nastroyki ssylok na risunki, tablitsy i dr. %%%
% Fiks problemy so ssylkami \subcaptionref v neobnovlennom TeXLive 2025 i memoir
\makeatletter
\@ifclasslater{memoir}{2025/09/22}{%
% Schitaem, chto versii memoir novee budut vzaimodeystvovat s amsmath i label korrektnee
}{
  \@ifclasslater{memoir}{2025/03/10}{%
  % Eta versiya memoir nekorrektno vzaimodeystvuet s obnovlennym amsmath
    \renewcommand{\memsubfig@captionpar}[2]{%
      \parbox[t]{#1}{\let\label=\memsub@label\@subcapsize\@contsubcstyle #2}}
  }{}
}
% Fiks problemy so ssylkami paketa subcaption i obnovlennogo amsmath
\@ifpackagelater{amsmath}{2025/01/11}{
% https://tex.stackexchange.com/a/746030
  \patchcmd\caption@subtypehook{\let\label\subcaption@label}
    {\let\label\subcaption@label\let\ltx@label\subcaption@label}{}{\fail}
}{}
\makeatother

% komandy \cref...format otvechayut za formatirovanie pri pomoshchi komandy \cref
% komandy \labelcref...format otvechayut za formatirovanie pri pomoshchi komandy \labelcref

\ifpresentation
\else
    \crefdefaultlabelformat{#2#1#3}

    % Uravnenie
    \crefformat{equation}{(#2#1#3)} % odinochnaya ssylka s pristavkoy
    \labelcrefformat{equation}{(#2#1#3)} % odinochnaya ssylka bez pristavki
    \crefrangeformat{equation}{(#3#1#4) \cyrdash~(#5#2#6)} % diapazon ssylok s pristavkoy
    \labelcrefrangeformat{equation}{(#3#1#4) \cyrdash~(#5#2#6)} % diapazon ssylok bez pristavki
    \crefmultiformat{equation}{(#2#1#3)}{ i~(#2#1#3)}{, (#2#1#3)}{ i~(#2#1#3)} % perechislenie ssylok s pristavkoy
    \labelcrefmultiformat{equation}{(#2#1#3)}{ i~(#2#1#3)}{, (#2#1#3)}{ i~(#2#1#3)} % perechislenie bez pristavki

    % Poduravnenie
    \crefformat{subequation}{(#2#1#3)} % odinochnaya ssylka s pristavkoy
    \labelcrefformat{subequation}{(#2#1#3)} % odinochnaya ssylka bez pristavki
    \crefrangeformat{subequation}{(#3#1#4) \cyrdash~(#5#2#6)} % diapazon ssylok s pristavkoy
    \labelcrefrangeformat{subequation}{(#3#1#4) \cyrdash~(#5#2#6)} % diapazon ssylok bez pristavki
    \crefmultiformat{subequation}{(#2#1#3)}{ i~(#2#1#3)}{, (#2#1#3)}{ i~(#2#1#3)} % perechislenie ssylok s pristavkoy
    \labelcrefmultiformat{subequation}{(#2#1#3)}{ i~(#2#1#3)}{, (#2#1#3)}{ i~(#2#1#3)} % perechislenie bez pristavki

    % Glava
    \crefformat{chapter}{#2#1#3} % odinochnaya ssylka s pristavkoy
    \labelcrefformat{chapter}{#2#1#3} % odinochnaya ssylka bez pristavki
    \crefrangeformat{chapter}{#3#1#4 \cyrdash~#5#2#6} % diapazon ssylok s pristavkoy
    \labelcrefrangeformat{chapter}{#3#1#4 \cyrdash~#5#2#6} % diapazon ssylok bez pristavki
    \crefmultiformat{chapter}{#2#1#3}{ i~#2#1#3}{, #2#1#3}{ i~#2#1#3} % perechislenie ssylok s pristavkoy
    \labelcrefmultiformat{chapter}{#2#1#3}{ i~#2#1#3}{, #2#1#3}{ i~#2#1#3} % perechislenie bez pristavki

    % Paragraf
    \crefformat{section}{#2#1#3} % odinochnaya ssylka s pristavkoy
    \labelcrefformat{section}{#2#1#3} % odinochnaya ssylka bez pristavki
    \crefrangeformat{section}{#3#1#4 \cyrdash~#5#2#6} % diapazon ssylok s pristavkoy
    \labelcrefrangeformat{section}{#3#1#4 \cyrdash~#5#2#6} % diapazon ssylok bez pristavki
    \crefmultiformat{section}{#2#1#3}{ i~#2#1#3}{, #2#1#3}{ i~#2#1#3} % perechislenie ssylok s pristavkoy
    \labelcrefmultiformat{section}{#2#1#3}{ i~#2#1#3}{, #2#1#3}{ i~#2#1#3} % perechislenie bez pristavki

    % Prilozhenie
    \crefformat{appendix}{#2#1#3} % odinochnaya ssylka s pristavkoy
    \labelcrefformat{appendix}{#2#1#3} % odinochnaya ssylka bez pristavki
    \crefrangeformat{appendix}{#3#1#4 \cyrdash~#5#2#6} % diapazon ssylok s pristavkoy
    \labelcrefrangeformat{appendix}{#3#1#4 \cyrdash~#5#2#6} % diapazon ssylok bez pristavki
    \crefmultiformat{appendix}{#2#1#3}{ i~#2#1#3}{, #2#1#3}{ i~#2#1#3} % perechislenie ssylok s pristavkoy
    \labelcrefmultiformat{appendix}{#2#1#3}{ i~#2#1#3}{, #2#1#3}{ i~#2#1#3} % perechislenie bez pristavki

    % Risunok
    \crefformat{figure}{#2#1#3} % odinochnaya ssylka s pristavkoy
    \labelcrefformat{figure}{#2#1#3} % odinochnaya ssylka bez pristavki
    \crefrangeformat{figure}{#3#1#4 \cyrdash~#5#2#6} % diapazon ssylok s pristavkoy
    \labelcrefrangeformat{figure}{#3#1#4 \cyrdash~#5#2#6} % diapazon ssylok bez pristavki
    \crefmultiformat{figure}{#2#1#3}{ i~#2#1#3}{, #2#1#3}{ i~#2#1#3} % perechislenie ssylok s pristavkoy
    \labelcrefmultiformat{figure}{#2#1#3}{ i~#2#1#3}{, #2#1#3}{ i~#2#1#3} % perechislenie bez pristavki

    % Tablitsa
    \crefformat{table}{#2#1#3} % odinochnaya ssylka s pristavkoy
    \labelcrefformat{table}{#2#1#3} % odinochnaya ssylka bez pristavki
    \crefrangeformat{table}{#3#1#4 \cyrdash~#5#2#6} % diapazon ssylok s pristavkoy
    \labelcrefrangeformat{table}{#3#1#4 \cyrdash~#5#2#6} % diapazon ssylok bez pristavki
    \crefmultiformat{table}{#2#1#3}{ i~#2#1#3}{, #2#1#3}{ i~#2#1#3} % perechislenie ssylok s pristavkoy
    \labelcrefmultiformat{table}{#2#1#3}{ i~#2#1#3}{, #2#1#3}{ i~#2#1#3} % perechislenie bez pristavki

    % Listing
    \crefformat{lstlisting}{#2#1#3} % odinochnaya ssylka s pristavkoy
    \labelcrefformat{lstlisting}{#2#1#3} % odinochnaya ssylka bez pristavki
    \crefrangeformat{lstlisting}{#3#1#4 \cyrdash~#5#2#6} % diapazon ssylok s pristavkoy
    \labelcrefrangeformat{lstlisting}{#3#1#4 \cyrdash~#5#2#6} % diapazon ssylok bez pristavki
    \crefmultiformat{lstlisting}{#2#1#3}{ i~#2#1#3}{, #2#1#3}{ i~#2#1#3} % perechislenie ssylok s pristavkoy
    \labelcrefmultiformat{lstlisting}{#2#1#3}{ i~#2#1#3}{, #2#1#3}{ i~#2#1#3} % perechislenie bez pristavki

    % Listing
    \crefformat{ListingEnv}{#2#1#3} % odinochnaya ssylka s pristavkoy
    \labelcrefformat{ListingEnv}{#2#1#3} % odinochnaya ssylka bez pristavki
    \crefrangeformat{ListingEnv}{#3#1#4 \cyrdash~#5#2#6} % diapazon ssylok s pristavkoy
    \labelcrefrangeformat{ListingEnv}{#3#1#4 \cyrdash~#5#2#6} % diapazon ssylok bez pristavki
    \crefmultiformat{ListingEnv}{#2#1#3}{ i~#2#1#3}{, #2#1#3}{ i~#2#1#3} % perechislenie ssylok s pristavkoy
    \labelcrefmultiformat{ListingEnv}{#2#1#3}{ i~#2#1#3}{, #2#1#3}{ i~#2#1#3} % perechislenie bez pristavki
\fi

%%% Nastroyki giperssylok %%%
\ifluatex
    \hypersetup{
        unicode,                % Unicode encoded PDF strings
    }
\fi

\hypersetup{
    linktocpage=true,           % ssylki s nomera stranitsy v oglavlenii, spiske tablits i spiske risunkov
%    linktoc=all,                % both the section and page part are links
%    pdfpagelabels=false,        % set PDF page labels (true|false)
    plainpages=false,           % Forces page anchors to be named by the Arabic form  of the page number, rather than the formatted form
    colorlinks,                 % ssylki otobrazhayutsya raskrashennym tekstom, a ne raskrashennym pryamougolnikom, vokrug teksta
    linkcolor={linkcolor},      % tsvet ssylok tipa ref, eqref i podobnykh
    citecolor={citecolor},      % tsvet ssylok-tsitat
    urlcolor={urlcolor},        % tsvet giperssylok
%    hidelinks,                  % Hide links (removing color and border)
    pdftitle={\thesisTitle},    % Zagolovok
    pdfauthor={\thesisAuthor},  % Avtor
    pdfsubject={\thesisSpecialtyNumber\ \thesisSpecialtyTitle},      % Tema
%    pdfcreator={Sozdatel},     % Sozdatel, Prilozhenie
%    pdfproducer={Proizvoditel},% Proizvoditel, Proizvoditel PDF
    pdfkeywords={\keywords},    % Klyuchevye slova
    pdflang={ru},
}
\ifnumequal{\value{draft}}{1}{% Chernovik
    \hypersetup{
        draft,
    }
}{}

%%% Spiski %%%
% Ispolzuem korotkoe tire (endash) dlya nenumerovannykh spiskov (GOST 2.105-95, punkt 4.1.7, trebuet defisa, no tak luchshe smotritsya)
\renewcommand{\labelitemi}{\normalfont\bfseries{--}}

% Perechislenie strochnymi bukvami latinskogo alfavita (GOST 2.105-95, 4.1.7)
%\renewcommand{\theenumi}{\alph{enumi}}
%\renewcommand{\labelenumi}{\theenumi)}

% Perechislenie strochnymi bukvami russkogo alfavita (GOST 2.105-95, 4.1.7)
\makeatletter
\AddEnumerateCounter{\asbuk}{\russian@alph}{shch}      % Upravlyaem spiskami/perechisleniyami cherez paket enumitem, a on 'ne znaet' pro asbuk, potomu 'uchim' ego
\makeatother
%\renewcommand{\theenumi}{\asbuk{enumi}} %pervyy uroven numeratsii
%\renewcommand{\labelenumi}{\theenumi)} %pervyy uroven numeratsii
\renewcommand{\theenumii}{\asbuk{enumii}} %vtoroy uroven numeratsii
\renewcommand{\labelenumii}{\theenumii)} %vtoroy uroven numeratsii
\renewcommand{\theenumiii}{\arabic{enumiii}} %tretiy uroven numeratsii
\renewcommand{\labelenumiii}{\theenumiii)} %tretiy uroven numeratsii

\setlist{nosep,%                                    % Edinyy stil dlya vsekh spiskov (paket enumitem), bez dopolnitelnykh intervalov.
    labelindent=\parindent,leftmargin=*%            % Kazhdyy punkt, podpunkt i perechislenie zapisyvayut s abzatsnogo otstupa (GOST 2.105-95, 4.1.8)
}

%%% Pravilnaya numeratsiya prilozheniy, risunkov i formul %%%
%% Po GOST 2.105, p. 4.3.8 Prilozheniya oboznachayut zaglavnymi bukvami russkogo alfavita,
%% nachinaya s A, za isklyucheniem bukv E, Z, Y, O, Ch, , Y, .
%% Zdes takzhe peredelany vse numeratsii russkimi bukvami.
\ifxetexorluatex
    \makeatletter
    \def\russian@Alph#1{\ifcase#1\or
       A\or B\or V\or G\or D\or E\or Zh\or
       I\or K\or L\or M\or N\or
       P\or R\or S\or T\or U\or F\or Kh\or
       Ts\or Sh\or Shch\or E\or Yu\or Ya\else\xpg@ill@value{#1}{russian@Alph}\fi}
    \def\russian@alph#1{\ifcase#1\or
       a\or b\or v\or g\or d\or e\or zh\or
       i\or k\or l\or m\or n\or
       p\or r\or s\or t\or u\or f\or kh\or
       ts\or sh\or shch\or e\or yu\or ya\else\xpg@ill@value{#1}{russian@alph}\fi}
    \def\cyr@Alph#1{\ifcase#1\or
        A\or B\or V\or G\or D\or E\or Zh\or
        I\or K\or L\or M\or N\or
        P\or R\or S\or T\or U\or F\or Kh\or
        Ts\or Sh\or Shch\or E\or Yu\or Ya\else\xpg@ill@value{#1}{cyr@Alph}\fi}
    \def\cyr@alph#1{\ifcase#1\or
        a\or b\or v\or g\or d\or e\or zh\or
        i\or k\or l\or m\or n\or
        p\or r\or s\or t\or u\or f\or kh\or
        ts\or sh\or shch\or e\or yu\or ya\else\xpg@ill@value{#1}{cyr@alph}\fi}
    \makeatother
\else
    \makeatletter
    \if@uni@ode
      \def\russian@Alph#1{\ifcase#1\or
        A\or B\or V\or G\or D\or E\or Zh\or
        I\or K\or L\or M\or N\or
        P\or R\or S\or T\or U\or F\or Kh\or
        Ts\or Sh\or Shch\or E\or Yu\or Ya\else\@ctrerr\fi}
    \else
      \def\russian@Alph#1{\ifcase#1\or
        \CYRA\or\CYRB\or\CYRV\or\CYRG\or\CYRD\or\CYRE\or\CYRZH\or
        \CYRI\or\CYRK\or\CYRL\or\CYRM\or\CYRN\or
        \CYRP\or\CYRR\or\CYRS\or\CYRT\or\CYRU\or\CYRF\or\CYRH\or
        \CYRC\or\CYRSH\or\CYRSHCH\or\CYREREV\or\CYRYU\or
        \CYRYA\else\@ctrerr\fi}
    \fi
    \if@uni@ode
      \def\russian@alph#1{\ifcase#1\or
        a\or b\or v\or g\or d\or e\or zh\or
        i\or k\or l\or m\or n\or
        p\or r\or s\or t\or u\or f\or kh\or
        ts\or sh\or shch\or e\or yu\or ya\else\@ctrerr\fi}
    \else
      \def\russian@alph#1{\ifcase#1\or
        \cyra\or\cyrb\or\cyrv\or\cyrg\or\cyrd\or\cyre\or\cyrzh\or
        \cyri\or\cyrk\or\cyrl\or\cyrm\or\cyrn\or
        \cyrp\or\cyrr\or\cyrs\or\cyrt\or\cyru\or\cyrf\or\cyrh\or
        \cyrc\or\cyrsh\or\cyrshch\or\cyrerev\or\cyryu\or
        \cyrya\else\@ctrerr\fi}
    \fi
    \makeatother
\fi


%%http://www.linux.org.ru/forum/general/6993203#comment-6994589 (ispolzuetsya totcount)
\makeatletter
\def\formtotal#1#2#3#4#5{%
    \newcount\@c
    \@c\totvalue{#1}\relax
    \newcount\@last
    \newcount\@pnul
    \@last\@c\relax
    \divide\@last 10
    \@pnul\@last\relax
    \divide\@pnul 10
    \multiply\@pnul-10
    \advance\@pnul\@last
    \multiply\@last-10
    \advance\@last\@c
    #2%
    \ifnum\@pnul=1#5\else%
    \ifcase\@last#5\or#3\or#4\or#4\or#4\else#5\fi
    \fi
}
\makeatother

\newcommand{\formbytotal}[5]{\total{#1}~\formtotal{#1}{#2}{#3}{#4}{#5}}

%%% Komandy retsenzirovaniya %%%
\ifboolexpr{ (test {\ifnumequal{\value{draft}}{1}}) or (test {\ifnumequal{\value{showmarkup}}{1}})}{
        \newrobustcmd{\todo}[1]{\textcolor{red}{#1}}
        \newrobustcmd{\note}[2][]{\ifstrempty{#1}{#2}{\textcolor{#1}{#2}}}
        \newenvironment{commentbox}[1][]%
        {\ifstrempty{#1}{}{\color{#1}}}%
        {}
}{
        \newrobustcmd{\todo}[1]{}
        \newrobustcmd{\note}[2][]{}
        \excludecomment{commentbox}
}
