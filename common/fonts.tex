%%% Kodirovki i shrifty %%%
\ifxetexorluatex
    % Yazyk po-umolchaniyu russkiy s podderzhkoy priyatnykh komand paketa babel
    \setmainlanguage[babelshorthands=true]{russian}
    % Dopolnitelnyy yazyk = angliyskiy (v amerikanskoy variatsii po-umolchaniyu)
    \setotherlanguage{english}

    % Proverka sushchestvovaniya shriftov. Nedostupna v pdflatex
    \ifnumequal{\value{fontfamily}}{1}{
        \IfFontExistsTF{Times New Roman}{}{\setcounter{fontfamily}{0}}
    }{}
    \ifnumequal{\value{fontfamily}}{2}{
        \IfFontExistsTF{LiberationSerif}{}{\setcounter{fontfamily}{0}}
    }{}

    \ifnumequal{\value{fontfamily}}{0}{% Semeystvo shriftov CMU. Ispolzuetsya kak fallback
        \setmonofont[
          BoldFont={cmuntb.otf},
          ItalicFont={cmunit.otf},
          BoldItalicFont={cmuntx.otf}
        ]{cmuntt.otf} % {CMU Typewriter Text} % monoshirinnyy shrift
        \newfontfamily\cyrillicfonttt[
          BoldFont={cmuntb.otf},
          ItalicFont={cmunit.otf},
          BoldItalicFont={cmuntx.otf}
        ]{cmuntt.otf} % {CMU Typewriter Text} % monoshirinnyy shrift dlya kirillitsy
        \defaultfontfeatures{Ligatures=TeX}   % standartnye ligatury TeX, zameny neskolkikh defisov na tire i t. p. Nastroyki monoshirinnogo shrifta dolzhny idti do etoy stroki, chtoby pri vrezkakh koda programm v kode ne primenyalis ligatury i zameny defisov
        \setmainfont[
          SlantedFont={cmunsl.otf},
          BoldSlantedFont={cmunbl.otf},
          BoldFont={cmunbx.otf},
          ItalicFont={cmunti.otf},
          BoldItalicFont={cmunbi.otf}
        ]{cmunrm.otf} % {CMU Serif}           % Shrift s zasechkami
        \newfontfamily\cyrillicfont[
          SlantedFont={cmunsl.otf},
          BoldSlantedFont={cmunbl.otf},
          BoldFont={cmunbx.otf},
          ItalicFont={cmunti.otf},
          BoldItalicFont={cmunbi.otf}
        ]{cmunrm.otf}   % {CMU Serif}         % Shrift s zasechkami dlya kirillitsy
        \setsansfont[
          BoldFont={cmunsx.otf},
          ItalicFont={cmunsi.otf},
          BoldItalicFont={cmunso.otf}
        ]{cmunss.otf} % {CMU Sans Serif}      % Shrift bez zasechek
        \newfontfamily\cyrillicfontsf[
          BoldFont={cmunsx.otf},
          ItalicFont={cmunsi.otf},
          BoldItalicFont={cmunso.otf}
        ]{cmunss.otf} % {CMU Sans Serif}      % Shrift bez zasechek dlya kirillitsy
    }

    \ifnumequal{\value{fontfamily}}{1}{                    % Semeystvo MS shriftov
        \setmonofont{Courier New}                          % monoshirinnyy shrift
        \newfontfamily\cyrillicfonttt{Courier New}         % monoshirinnyy shrift dlya kirillitsy
        \defaultfontfeatures{Ligatures=TeX}                % standartnye ligatury TeX, zameny neskolkikh defisov na tire i t. p. Nastroyki monoshirinnogo shrifta dolzhny idti do etoy stroki, chtoby pri vrezkakh koda programm v kode ne primenyalis ligatury i zameny defisov
        \setmainfont{Times New Roman}                      % Shrift s zasechkami
        \newfontfamily\cyrillicfont{Times New Roman}       % Shrift s zasechkami dlya kirillitsy
        \setsansfont{Arial}                                % Shrift bez zasechek
        \newfontfamily\cyrillicfontsf{Arial}               % Shrift bez zasechek dlya kirillitsy
    }

    \ifnumequal{\value{fontfamily}}{2}{                    % Semeystvo shriftov Liberation (https://pagure.io/liberation-fonts)
        \setmonofont{LiberationMono}[Scale=0.87] % monoshirinnyy shrift
        \newfontfamily\cyrillicfonttt{LiberationMono}[     % monoshirinnyy shrift dlya kirillitsy
            Scale=0.87]
        \defaultfontfeatures{Ligatures=TeX}                % standartnye ligatury TeX, zameny neskolkikh defisov na tire i t. p. Nastroyki monoshirinnogo shrifta dolzhny idti do etoy stroki, chtoby pri vrezkakh koda programm v kode ne primenyalis ligatury i zameny defisov
        \setmainfont{LiberationSerif}                      % Shrift s zasechkami
        \newfontfamily\cyrillicfont{LiberationSerif}       % Shrift s zasechkami dlya kirillitsy
        \setsansfont{LiberationSans}                       % Shrift bez zasechek
        \newfontfamily\cyrillicfontsf{LiberationSans}      % Shrift bez zasechek dlya kirillitsy
    }

\else
    \ifnumequal{\value{usealtfont}}{1}{% Ispolzuetsya pscyr, pri nalichii
        \IfFileExists{pscyr.sty}{\renewcommand{\rmdefault}{ftm}}{}
    }{}
\fi
