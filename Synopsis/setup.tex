%%%%%%%%%%%%%%%%%%%%%%%%%%%%%%%%%%%%%%%%%%%%%%%%%%%%%%%
%%%% Fayl uproshchennykh nastroek shablona avtoreferata %%%%
%%%%%%%%%%%%%%%%%%%%%%%%%%%%%%%%%%%%%%%%%%%%%%%%%%%%%%%

%%% Initsializirovanie peremennykh, ne trogat!  %%%
\newcounter{showperssign}
\newcounter{showsecrsign}
\newcounter{showopplead}
%%%%%%%%%%%%%%%%%%%%%%%%%%%%%%%%%%%%%%%%%%%%%%%%%%%%%%%

%%% Spisok publikatsiy %%%
\makeatletter
\@ifundefined{c@usefootcite}{
  \newcounter{usefootcite}
  \setcounter{usefootcite}{0} % 0 --- dva (ili bolee) spiska literatury;
                              % 1 --- spisok publikatsiy avtora + tsitirovanie
                              %       drugikh rabot v snoskakh
}{}
\makeatother

\makeatletter
\@ifundefined{c@bibgrouped}{
  \newcounter{bibgrouped}
  \setcounter{bibgrouped}{0}  % 0 --- edinyy spisok rabot avtora;
                              % 1 --- sgruppirovannye raboty avtora
}{}
\makeatother

%%% Oblast uproshchennogo upravleniya oformleniem %%%

%% Upravlenie zazorom mezhdu podrisunochnoy podpisyu i osnovnym tekstom %%
\setlength{\belowcaptionskip}{10pt plus 20pt minus 2pt}


%% Podpis tablits %%

% smeshchenie strok podpisi posle pervoy
\newcommand{\tabindent}{0cm}

% tip formatirovaniya tablitsy
% plain --- nazvanie i tekst v odnoy stroke
% split --- nazvanie i tekst v raznykh strokakh
\newcommand{\tabformat}{plain}

%%% nastroyki formatirovaniya tablitsy `plain'

% vyravnivanie po tsentru podpisi, sostoyashchey iz odnoy stroki
% true  --- vyravnivat
% false --- ne vyravnivat
\newcommand{\tabsinglecenter}{false}

% vyravnivanie podpisi tablits
% justified   --- vyravnivat kak obychnyy tekst
% centering   --- vyravnivat po tsentru
% centerlast  --- vyravnivat po tsentru tolko poslednyuyu stroku
% centerfirst --- vyravnivat po tsentru tolko pervuyu stroku
% raggedleft  --- vyravnivat po pravomu krayu
% raggedright --- vyravnivat po levomu krayu
\newcommand{\tabjust}{justified}

% Razdelitel zapisi «Tablitsa #» i nazvaniya tablitsy
\newcommand{\tablabelsep}{~\cyrdash\ }

%%% nastroyki formatirovaniya tablitsy `split'

% polozhenie nazvaniya tablitsy
% \centering   --- vyravnivat po tsentru
% \raggedleft  --- vyravnivat po pravomu krayu
% \raggedright --- vyravnivat po levomu krayu
\newcommand{\splitformatlabel}{\raggedleft}

% polozhenie teksta podpisi
% \centering   --- vyravnivat po tsentru
% \raggedleft  --- vyravnivat po pravomu krayu
% \raggedright --- vyravnivat po levomu krayu
\newcommand{\splitformattext}{\raggedright}

%% Podpis risunkov %%
%Razdelitel zapisi «Risunok #» i nazvaniya risunka
\newcommand{\figlabelsep}{~\cyrdash\ }  % (GOST 2.105, 4.3.1)
                                        % "--- zdes ne rabotaet

%Demonstratsiya podpisi dissertanta na avtoreferate
\setcounter{showperssign}{1}  % 0 --- ne pokazyvat;
                              % 1 --- pokazyvat
%Demonstratsiya podpisi uchenogo sekretarya na avtoreferate
\setcounter{showsecrsign}{1}  % 0 --- ne pokazyvat;
                              % 1 --- pokazyvat
%Demonstratsiya informatsii ob opponentakh i vedushchey organizatsii na avtoreferate
\setcounter{showopplead}{1}   % 0 --- ne pokazyvat;
                              % 1 --- pokazyvat

%%% Tsveta giperssylok %%%
% Latex color definitions: http://latexcolor.com/
\definecolor{linkcolor}{rgb}{0.9,0,0}
\definecolor{citecolor}{rgb}{0,0.6,0}
\definecolor{urlcolor}{rgb}{0,0,1}
%\definecolor{linkcolor}{rgb}{0,0,0} %black
%\definecolor{citecolor}{rgb}{0,0,0} %black
%\definecolor{urlcolor}{rgb}{0,0,0} %black
