%%% Izobrazheniya %%%
\graphicspath{{images/}{Synopsis/images/}}         % Puti k izobrazheniyam

%%% Maket stranitsy %%%
\geometry{a5paper, top=14mm, bottom=14mm, inner=18mm, outer=10mm, footskip=5mm, nomarginpar}%, showframe
\setlength{\topskip}{0pt}   %razmer dopolnitelnogo verkhnego polya

%%% Intervaly %%%
%% Realizatsiya sredstvami klassa (na osnove setspace) blizhe k tipografskoy klassike.
%% I pravit srazu i v tablitsakh (esli so zvezdochkoy)
%\DoubleSpacing*     % Dvoynoy interval
%\OnehalfSpacing*    % Polutornyy interval
\SingleSpacing      % Odinarnyy interval
%\setSpacing{1.42}   % Polutornyy interval, podobnyy Vordu (vozmozhno, stoit vklyuchat vmeste s predydushchey strokoy)

%%% Vyravnivanie i perenosy %%%
%% http://tex.stackexchange.com/questions/241343/what-is-the-meaning-of-fussy-sloppy-emergencystretch-tolerance-hbadness
%% http://www.latex-community.org/forum/viewtopic.php?p=70342#p70342
\tolerance 1414
\hbadness 1414
\emergencystretch 1.5em % V sluchae problem regulirovat v pervuyu ochered
\hfuzz 0.3pt
\vfuzz \hfuzz
%\raggedbottom
%\sloppy                 % Izbavlyaemsya ot perepolneniy
\clubpenalty=10000      % Zapreshchaem razryv stranitsy posle pervoy stroki abzatsa
\widowpenalty=10000     % Zapreshchaem razryv stranitsy posle posledney stroki abzatsa

%%% Kolontituly %%%
\makeevenhead{plain}{}{}{}
\makeoddhead{plain}{}{}{}
\makeevenfoot{plain}{}{\thepage}{}
\makeoddfoot{plain}{}{\thepage}{}
\pagestyle{plain}

%%% Razmery zagolovkov %%%
\setsecheadstyle{\normalfont\large\bfseries}
\renewcommand*{\chaptitlefont}{\normalfont\large\bfseries}

%%% Podpisi %%%
\setfloatadjustment{table}{%
    \setlength{\abovecaptionskip}{0pt}   % Otbivka nad podpisyu
    \setlength{\belowcaptionskip}{0pt}   % Otbivka pod podpisyu
}

%%% Otstupy u plavayushchikh blokov %%%
\setlength\textfloatsep{1ex}
