\thispagestyle{empty}

\noindent%
\begin{tabularx}{\textwidth}{@{}lXr@{}}%
    & & \large{Na pravakh rukopisi}\\
    \IfFileExists{images/logo.pdf}{\includegraphics[height=2.5cm]{logo}}{\rule[0pt]{0pt}{2.5cm}}  & &
    \ifnumequal{\value{showperssign}}{0}{%
        \rule[0pt]{0pt}{1.5cm}
    }{
        \includegraphics[height=1.5cm]{personal-signature.png}
    }\\
\end{tabularx}

\vspace{0pt plus1fill} %chislo pered fill = kratnost otnositelno nekotorogo rasstoyaniya fill, kuskami kotorogo zapolneny pustye mesta
\begin{center}
\textbf {\large \thesisAuthor}
\end{center}

\vspace{0pt plus3fill} %chislo pered fill = kratnost otnositelno nekotorogo rasstoyaniya fill, kuskami kotorogo zapolneny pustye mesta
\begin{center}
\textbf {\Large %\MakeUppercase
\thesisTitle}

\vspace{0pt plus3fill} %chislo pered fill = kratnost otnositelno nekotorogo rasstoyaniya fill, kuskami kotorogo zapolneny pustye mesta
{\large Spetsialnost \thesisSpecialtyNumber\ "---\par <<\thesisSpecialtyTitle>>}

\ifdefined\thesisSpecialtyTwoNumber
{\large Spetsialnost \thesisSpecialtyTwoNumber\ "---\par <<\thesisSpecialtyTwoTitle>>}
\fi

\vspace{0pt plus1.5fill} %chislo pered fill = kratnost otnositelno nekotorogo rasstoyaniya fill, kuskami kotorogo zapolneny pustye mesta
\Large{Avtoreferat}\par
\large{dissertatsii na soiskanie uchenoy stepeni\par \thesisDegree}
\end{center}

\vspace{0pt plus4fill} %chislo pered fill = kratnost otnositelno nekotorogo rasstoyaniya fill, kuskami kotorogo zapolneny pustye mesta
{\centering\thesisCity~--- \thesisYear\par}

\newpage
% oborotnaya storona oblozhki
\thispagestyle{empty}
\noindent Rabota vypolnena v {\thesisInOrganization}.

\vspace{0.008\paperheight plus1fill}
\noindent%
\begin{tabularx}{\textwidth}{@{}lX@{}}
    \ifdefined\supervisorTwoFio
    Nauchnye rukovoditeli:   & \supervisorRegalia\par
                              \ifdefined\supervisorDead
                              \framebox{\textbf{\supervisorFio}}
                              \else
                              \textbf{\supervisorFio}
                              \fi
                              \par
                              \vspace{0.013\paperheight}
                              \supervisorRegaliaTwo\par
                              \ifdefined\supervisorTwoDead
                              \framebox{\textbf{\supervisorTwoFio}}
                              \else
                              \textbf{\supervisorTwoFio}
                              \fi
                              \vspace{0.013\paperheight}\\
    \else
    Nauchnyy rukovoditel:   & \supervisorRegalia\par
                              \ifdefined\supervisorDead
                              \framebox{\textbf{\supervisorFio}}
                              \else
                              \textbf{\supervisorFio}
                              \fi
                              \vspace{0.013\paperheight}\\
    \fi
    Ofitsialnye opponenty:  &
    \ifnumequal{\value{showopplead}}{0}{\vspace{13\onelineskip plus1fill}}{%
        \textbf{\opponentOneFio,}\par
        \opponentOneRegalia,\par
        \opponentOneJobPlace,\par
        \opponentOneJobPost\par
        \vspace{0.01\paperheight}
        \textbf{\opponentTwoFio,}\par
        \opponentTwoRegalia,\par
        \opponentTwoJobPlace,\par
        \opponentTwoJobPost
    \ifdefined\opponentThreeFio
        \par
        \vspace{0.01\paperheight}
        \textbf{\opponentThreeFio,}\par
        \opponentThreeRegalia,\par
        \opponentThreeJobPlace,\par
        \opponentThreeJobPost
    \fi
    }%
    \vspace{0.013\paperheight} \\
    \ifdefined\leadingOrganizationTitle
    Vedushchaya organizatsiya:    &
    \ifnumequal{\value{showopplead}}{0}{\vspace{6\onelineskip plus1fill}}{%
        \leadingOrganizationTitle
    }%
    \fi
\end{tabularx}
\vspace{0.008\paperheight plus1fill}

\noindent Zashchita sostoitsya \defenseDate~na~zasedanii dissertatsionnogo soveta \defenseCouncilNumber~pri \defenseCouncilTitle~po adresu: \defenseCouncilAddress.

\vspace{0.008\paperheight plus1fill}
\noindent S dissertatsiey mozhno oznakomitsya v biblioteke \synopsisLibrary.

\vspace{0.008\paperheight plus1fill}
\noindent Otzyvy na avtoreferat v dvukh ekzemplyarakh, zaverennye pechatyu uchrezhdeniya, prosba napravlyat po adresu: \defenseCouncilAddress, uchenomu sekretaryu dissertatsionnogo soveta~\defenseCouncilNumber.

\vspace{0.008\paperheight plus1fill}
\noindent{Avtoreferat razoslan \synopsisDate.}

\noindent Telefon dlya spravok: \defenseCouncilPhone.

\vspace{0.008\paperheight plus1fill}
\noindent%
\begin{tabularx}{\textwidth}{@{}%
>{\raggedright\arraybackslash}b{18em}@{}
>{\centering\arraybackslash}X
r
@{}}
    Uchenyy sekretar\par
    dissertatsionnogo soveta\par
    \defenseCouncilNumber,\par
    \defenseSecretaryRegalia
    &
    \ifnumequal{\value{showsecrsign}}{0}{}{%
        \includegraphics[width=2cm]{secretary-signature.png}%
    }%
    &
    \defenseSecretaryFio
\end{tabularx}
