\pdfbookmark{Obshchaya kharakteristika raboty}{characteristic}             % Zakladka pdf
\section*{Obshchaya kharakteristika raboty}

\newcommand{\actuality}{\pdfbookmark[1]{Aktualnost}{actuality}\underline{\textbf{\actualityTXT}}}
\newcommand{\progress}{\pdfbookmark[1]{Razrabotannost temy}{progress}\underline{\textbf{\progressTXT}}}
\newcommand{\aim}{\pdfbookmark[1]{Tseli}{aim}\underline{{\textbf\aimTXT}}}
\newcommand{\tasks}{\pdfbookmark[1]{Zadachi}{tasks}\underline{\textbf{\tasksTXT}}}
\newcommand{\aimtasks}{\pdfbookmark[1]{Tseli i zadachi}{aimtasks}\aimtasksTXT}
\newcommand{\novelty}{\pdfbookmark[1]{Nauchnaya novizna}{novelty}\underline{\textbf{\noveltyTXT}}}
\newcommand{\influence}{\pdfbookmark[1]{Prakticheskaya znachimost}{influence}\underline{\textbf{\influenceTXT}}}
\newcommand{\methods}{\pdfbookmark[1]{Metodologiya i metody issledovaniya}{methods}\underline{\textbf{\methodsTXT}}}
\newcommand{\defpositions}{\pdfbookmark[1]{Polozheniya, vynosimye na zashchitu}{defpositions}\underline{\textbf{\defpositionsTXT}}}
\newcommand{\reliability}{\pdfbookmark[1]{Dostovernost}{reliability}\underline{\textbf{\reliabilityTXT}}}
\newcommand{\probation}{\pdfbookmark[1]{Aprobatsiya}{probation}\underline{\textbf{\probationTXT}}}
\newcommand{\contribution}{\pdfbookmark[1]{Lichnyy vklad}{contribution}\underline{\textbf{\contributionTXT}}}
\newcommand{\publications}{\pdfbookmark[1]{Publikatsii}{publications}\underline{\textbf{\publicationsTXT}}}


{\actuality} Obzor, vvedenie v temu, oboznachenie mesta dannoy raboty v
mirovykh issledovaniyakh i~t.\:p., mozhno ispolzovat ssylki na~drugie
raboty~\autocite{Gosele1999161,Lermontov}
(esli ikh~net, to~v~avtoreferate
avtomaticheski propadet razdel <<Spisok literatury>>). Vnimanie! Ssylki
na~drugie raboty v~razdele obshchey kharakteristiki raboty mozhno
ispolzovat tolko pri ispolzovanii \verb!biblatex! (iz-za tekhnicheskikh
ogranicheniy \verb!bibtex8!. Eto svyazano s tem, chto odna
i~ta~zhe~kharakteristika ispolzuyutsya i~v~tekste dissertatsii, i v
avtoreferate. V~poslednem, soglasno GOST, dolzhen prisutstvovat spisok
rabot avtora po~teme dissertatsii, a~\verb!bibtex8! ne~umeet vyvodit v~odnom
fayle dva spiska literatury).
Pri ispolzovanii \verb!biblatex! vozmozhno ispolzovanie isklyuchitelno
v~avtoreferate podstrochnykh ssylok
dlya drugikh rabot komandoy \verb!\autocite!~\autocite{Marketing}, a~takzhe tsitirovanie
sobstvennykh rabot komandoy \verb!\cite!. Dlya etogo v~fayle
\verb!common/setup.tex! neobkhodimo prisvoit polozhitelnoe znachenie
schetchiku \verb!\setcounter{usefootcite}{1}!.

Dlya generatsii soderzhimogo titulnogo lista avtoreferata, dissertatsii
i~prezentatsii ispolzuyutsya dannye iz fayla \verb!common/data.tex!. Esli,
naprimer, vy menyaete nazvanie dissertatsii, to ono avtomaticheski
poyavitsya v~itogovykh faylakh posle ocherednogo zapuska \LaTeX. Soglasno
GOST 7.0.11-2011 <<5.1.1 Titulnyy list yavlyaetsya pervoy stranitsey
dissertatsii, sluzhit istochnikom informatsii, neobkhodimoy dlya obrabotki i
poiska dokumenta>>. Nalichie logotipa organizatsii na~titulnom liste
uproshchaet obrabotku i~poisk, dlya etogo razmetite logotip vashey
organizatsii v papke images v~formate PDF (luchshe nayti ego v vektornom
variante, chtoby on khorosho smotrelsya pri pechati) pod imenem
\verb!logo.pdf!. Nastroit razmer izobrazheniya s logotipom mozhno
v~sootvetstvuyushchikh mestakh faylov \verb!title.tex!  otdelno dlya
dissertatsii i avtoreferata. Esli vam logotip ne~nuzhen, to prosto
udalite fayl s~logotipom.

\ifsynopsis
Etot abzats poyavlyaetsya tolko v~avtoreferate.
Dlya formirovaniya blokov, kotorye budut obrabatyvatsya tolko v~avtoreferate,
zavedena proverka usloviya \verb!\!\verb!ifsynopsis!.
Znachenie usloviya zadaetsya v~osnovnom fayle dokumenta (\verb!synopsis.tex! dlya
avtoreferata).
\else
Etot abzats poyavlyaetsya tolko v~dissertatsii.
Cherez proverku usloviya \verb!\!\verb!ifsynopsis!, zadavaemogo v~osnovnom fayle
dokumenta (\verb!dissertation.tex! dlya dissertatsii), mozhno sdelat novuyu
komandu, obespechivayushchuyu poyavlenie tsitaty v~dissertatsii, no~ne~v~avtoreferate.
\fi

% {\progress}
% Etot razdel dolzhen byt otdelnym strukturnym elementom po
% GOST, no on, kak pravilo, vklyuchaetsya v opisanie aktualnosti
% temy. Nuzhen on otdelnym strukturynm elemementom ili net ---
% smotrite drugie dissertatsii vashego soveta, skoree vsego ne nuzhen.

{\aim} dannoy raboty yavlyaetsya \ldots

Dlya~dostizheniya postavlennoy tseli neobkhodimo bylo reshit sleduyushchie {\tasks}:
\begin{enumerate}[beginpenalty=10000] % https://tex.stackexchange.com/a/476052/104425
  \item Issledovat, razrabotat, vychislit i~t.\:d. i~t.\:p.
  \item Issledovat, razrabotat, vychislit i~t.\:d. i~t.\:p.
  \item Issledovat, razrabotat, vychislit i~t.\:d. i~t.\:p.
  \item Issledovat, razrabotat, vychislit i~t.\:d. i~t.\:p.
\end{enumerate}


{\novelty}
\begin{enumerate}[beginpenalty=10000] % https://tex.stackexchange.com/a/476052/104425
  \item Vpervye \ldots
  \item Vpervye \ldots
  \item Bylo vypolneno originalnoe issledovanie \ldots
\end{enumerate}

{\influence} \ldots

{\methods} \ldots

{\defpositions}
\begin{enumerate}[beginpenalty=10000] % https://tex.stackexchange.com/a/476052/104425
  \item Pervoe polozhenie
  \item Vtoroe polozhenie
  \item Trete polozhenie
  \item Chetvertoe polozhenie
\end{enumerate}
V papke Documents mozhno oznakomitsya s resheniem soveta iz Tomskogo~GU
(v~fayle \verb+Def_positions.pdf+), gde obosnovanno dayutsya rekomendatsii
po~formulirovkam zashchishchaemykh polozheniy.

{\reliability} poluchennykh rezultatov obespechivaetsya \ldots \ Rezultaty nakhodyatsya v sootvetstvii s rezultatami, poluchennymi drugimi avtorami.


{\probation}
Osnovnye rezultaty raboty dokladyvalis~na:
perechislenie osnovnykh konferentsiy, simpoziumov i~t.\:p.

{\contribution} Avtor prinimal aktivnoe uchastie \ldots

\ifnumequal{\value{bibliosel}}{0}
{%%% Vstroennaya realizatsiya s zagruzkoy fayla cherez dvizhok bibtex8. (Pri zhelanii, vnutri mozhno ispolzovat obychnye ssylki, napodobie `\cite{vakbib1,vakbib2}`).
    {\publications} Osnovnye rezultaty po teme dissertatsii izlozheny
    v~XX~pechatnykh izdaniyakh,
    X iz kotorykh izdany v zhurnalakh, rekomendovannykh VAK,
    X "--- v tezisakh dokladov.
}%
{%%% Realizatsiya paketom biblatex cherez dvizhok biber
    \begin{refsection}[bl-author, bl-registered]
        % Eto refsection=1.
        % Protsitirovannye zdes raboty:
        %  * podschityvayutsya, dlya avtomaticheskogo sostavleniya frazy "Osnovnye rezultaty ..."
        %  * popadayut v avtorskuyu bibliografiyu, pri usefootcite==0 i stile `\insertbiblioauthor` ili `\insertbiblioauthorgrouped`
        %  * numeruyutsya tam v zavisimosti ot poryadka komand `\printbibliography` v etom razdele.
        %  * pri ispolzovanii `\insertbiblioauthorgrouped`, poryadok komand `\printbibliography` v nem dolzhen byt tem zhe (sm. biblio/biblatex.tex)
        %
        % Nevidimyy bibliograficheskiy spisok dlya podscheta kolichestva publikatsiy:
        \phantom{\printbibliography[heading=nobibheading, section=1, env=countauthorvak,          keyword=biblioauthorvak]%
        \printbibliography[heading=nobibheading, section=1, env=countauthorwos,          keyword=biblioauthorwos]%
        \printbibliography[heading=nobibheading, section=1, env=countauthorscopus,       keyword=biblioauthorscopus]%
        \printbibliography[heading=nobibheading, section=1, env=countauthorconf,         keyword=biblioauthorconf]%
        \printbibliography[heading=nobibheading, section=1, env=countauthorother,        keyword=biblioauthorother]%
        \printbibliography[heading=nobibheading, section=1, env=countregistered,         keyword=biblioregistered]%
        \printbibliography[heading=nobibheading, section=1, env=countauthorpatent,       keyword=biblioauthorpatent]%
        \printbibliography[heading=nobibheading, section=1, env=countauthorprogram,      keyword=biblioauthorprogram]%
        \printbibliography[heading=nobibheading, section=1, env=countauthor,             keyword=biblioauthor]%
        \printbibliography[heading=nobibheading, section=1, env=countauthorvakscopuswos, filter=vakscopuswos]%
        \printbibliography[heading=nobibheading, section=1, env=countauthorscopuswos,    filter=scopuswos]}%
        %
        \nocite{*}%
        %
        {\publications} Osnovnye rezultaty po teme dissertatsii izlozheny v~\arabic{citeauthor}~pechatnykh izdaniyakh,
        \arabic{citeauthorvak} iz kotorykh izdany v zhurnalakh, rekomendovannykh VAK%
        \ifnum \value{citeauthorscopuswos}>0%
            , \arabic{citeauthorscopuswos} "--- v~periodicheskikh nauchnykh zhurnalakh, indeksiruemykh Web of~Science i Scopus%
        \fi%
        \ifnum \value{citeauthorconf}>0%
            , \arabic{citeauthorconf} "--- v~tezisakh dokladov.
        \else%
            .
        \fi%
        \ifnum \value{citeregistered}=1%
            \ifnum \value{citeauthorpatent}=1%
                Zaregistrirovan \arabic{citeauthorpatent} patent.
            \fi%
            \ifnum \value{citeauthorprogram}=1%
                Zaregistrirovana \arabic{citeauthorprogram} programma dlya EVM.
            \fi%
        \fi%
        \ifnum \value{citeregistered}>1%
            Zaregistrirovany\ %
            \ifnum \value{citeauthorpatent}>0%
            \formbytotal{citeauthorpatent}{patent}{}{a}{}%
            \ifnum \value{citeauthorprogram}=0 . \else \ i~\fi%
            \fi%
            \ifnum \value{citeauthorprogram}>0%
            \formbytotal{citeauthorprogram}{programm}{a}{y}{} dlya EVM.
            \fi%
        \fi%
        % K publikatsiyam, v kotorykh izlagayutsya osnovnye nauchnye rezultaty dissertatsii na soiskanie uchenoy
        % stepeni, v retsenziruemykh izdaniyakh priravnivayutsya patenty na izobreteniya, patenty (svidetelstva) na
        % poleznuyu model, patenty na promyshlennyy obrazets, patenty na selektsionnye dostizheniya, svidetelstva
        % na programmu dlya elektronnykh vychislitelnykh mashin, bazu dannykh, topologiyu integralnykh mikroskhem,
        % zaregistrirovannye v ustanovlennom poryadke.(v red. Postanovleniya Pravitelstva RF ot 21.04.2016 N 335)
    \end{refsection}%
    \begin{refsection}[bl-author, bl-registered]
        % Eto refsection=2.
        % Protsitirovannye zdes raboty:
        %  * popadayut v avtorskuyu bibliografiyu, pri usefootcite==0 i stile `\insertbiblioauthorimportant`.
        %  * ni na chto ne vliyayut v protivnom sluchae
        \nocite{vakbib2}%vak
        \nocite{patbib1}%patent
        \nocite{progbib1}%program
        \nocite{bib1}%other
        \nocite{confbib1}%conf
    \end{refsection}%
        %
        % Vse, chto vne etikh dvukh refsection, eto refsection=0,
        %  * dlya dissertatsii - eto normalnye ssylki, popadayushchie v obychnuyu bibliografiyu
        %  * dlya avtoreferata:
        %     * pri usefootcite==0, ssylka korrektno srabotaet tolko dlya istochnika iz `external.bib`. Dlya svoikh rabot --- napechataet "[0]" (i dazhe Warning ne vylezet).
        %     * pri usefootcite==1, ssylka srabotaet normalno. V avtorskoy bibliografii budut tolko protsitirovannye v refsection=0 raboty.
}

Pri ispolzovanii paketa \verb!biblatex! budut podschitany vse raboty, dobavlennye
v fayl \verb!biblio/author.bib!. Dlya pravilnogo podscheta rabot v~razlichnykh
sistemakh tsitirovaniya trebuetsya ispolzovat polya:
\begin{itemize}
        \item \texttt{authorvak} esli publikatsiya indeksirovana VAK,
        \item \texttt{authorscopus} esli publikatsiya indeksirovana Scopus,
        \item \texttt{authorwos} esli publikatsiya indeksirovana Web of Science,
        \item \texttt{authorconf} dlya dokladov konferentsiy,
        \item \texttt{authorpatent} dlya patentov,
        \item \texttt{authorprogram} dlya zaregistrirovannykh programm dlya EVM,
        \item \texttt{authorother} dlya drugikh publikatsiy.
\end{itemize}
Dlya podscheta ispolzuyutsya schetchiki:
\begin{itemize}
        \item \texttt{citeauthorvak} dlya rabot, indeksiruemykh VAK,
        \item \texttt{citeauthorscopus} dlya rabot, indeksiruemykh Scopus,
        \item \texttt{citeauthorwos} dlya rabot, indeksiruemykh Web of Science,
        \item \texttt{citeauthorvakscopuswos} dlya rabot, indeksiruemykh odnoy iz trekh baz,
        \item \texttt{citeauthorscopuswos} dlya rabot, indeksiruemykh Scopus ili Web of~Science,
        \item \texttt{citeauthorconf} dlya dokladov na konferentsiyakh,
        \item \texttt{citeauthorother} dlya ostalnykh rabot,
        \item \texttt{citeauthorpatent} dlya patentov,
        \item \texttt{citeauthorprogram} dlya zaregistrirovannykh programm dlya EVM,
        \item \texttt{citeauthor} dlya summarnogo kolichestva rabot.
\end{itemize}
% Schetchik \texttt{citeexternal} ispolzuetsya dlya podscheta protsitirovannykh publikatsiy;
% \texttt{citeregistered} "--- dlya podscheta summarnogo kolichestva patentov i programm dlya EVM.

Dlya dobavleniya v spisok publikatsiy avtora rabot, kotorye ne byli protsitirovany v
avtoreferate, trebuetsya ikh~perechislit s ispolzovaniem komandy \verb!\nocite! v
\verb!Synopsis/content.tex!.
 % Kharakteristika raboty po strukture vo vvedenii i v avtoreferate ne otlichaetsya (GOST R 7.0.11, punkty 5.3.1 i 9.2.1), potomu ee zagruzhaem iz odnogo i togo zhe vneshnego fayla, predvaritelno zadav formu vydeleniya nekotorym parametram

%Dissertatsionnaya rabota byla vypolnena pri podderzhke grantov \dots

%\underline{\textbf{Obem i struktura raboty.}} Dissertatsiya sostoit iz~vvedeniya,
%chetyrekh glav, zaklyucheniya i~prilozheniya. Polnyy obem dissertatsii
%\textbf{KhKhKh}~stranits teksta s~\textbf{KhKh}~risunkami i~5~tablitsami. Spisok
%literatury soderzhit \textbf{KhKhX}~naimenovanie.

\pdfbookmark{Soderzhanie raboty}{description}                          % Zakladka pdf
\section*{Soderzhanie raboty}
Vo \underline{\textbf{vvedenii}} obosnovyvaetsya aktualnost
issledovaniy, provodimykh v~ramkakh dannoy dissertatsionnoy raboty,
privoditsya obzor nauchnoy literatury po~izuchaemoy probleme,
formuliruetsya tsel, stavyatsya zadachi raboty, izlagaetsya nauchnaya novizna
i prakticheskaya znachimost predstavlyaemoy raboty. V~posleduyushchikh glavakh
snachala opisyvaetsya obshchiy printsip, pozvolyayushchiy \dots, a~potom idet
aprobatsiya na chastnykh primerakh: \dots  i~\dots.


\underline{\textbf{Pervaya glava}} posvyashchena \dots

kartinku mozhno dobavit tak:
\begin{figure}[ht]
    \centerfloat{
        \hfill
        \subcaptionbox{\LaTeX}{%
            \includegraphics[scale=0.27]{latex}}
        \hfill
        \subcaptionbox{Knuth}{%
            \includegraphics[width=0.25\linewidth]{knuth1}}
        \hfill
    }
    \caption{Podpis k kartinke.}\label{fig:latex}
\end{figure}

Formuly v stroku bez nomera dobavlyayutsya tak:
\[
    \lambda_{T_s} = K_x\frac{d{x}}{d{T_s}}, \qquad
    \lambda_{q_s} = K_x\frac{d{x}}{d{q_s}},
\]

\underline{\textbf{Vtoraya glava}} posvyashchena issledovaniyu

\underline{\textbf{Tretya glava}} posvyashchena issledovaniyu

Mozhno soslatsya na svoi raboty v avtoreferate. Dlya etogo v fayle
\verb!Synopsis/setup.tex! neobkhodimo prisvoit polozhitelnoe znachenie
schetchiku \verb!\setcounter{usefootcite}{1}!. V takom sluchae ssylki na
raboty drugikh avtorov budut podstrochnymi.
Izlozhennye v tretey glave rezultaty opublikovany v~\cite{vakbib1, vakbib2}.
Ispolzovanie podstrochnykh ssylok vnutri tablits mozhet vyzyvat problemy.

V \underline{\textbf{chetvertoy glave}} privedeno opisanie

\FloatBarrier
\pdfbookmark{Zaklyuchenie}{conclusion}                                  % Zakladka pdf
V \underline{\textbf{zaklyuchenii}} privedeny osnovnye rezultaty raboty, kotorye zaklyuchayutsya v sleduyushchem:
%% Soglasno GOST R 7.0.11-2011:
%% 5.3.3 V zaklyuchenii dissertatsii izlagayut itogi vypolnennogo issledovaniya, rekomendatsii, perspektivy dalneyshey razrabotki temy.
%% 9.2.3 V zaklyuchenii avtoreferata dissertatsii izlagayut itogi dannogo issledovaniya, rekomendatsii i perspektivy dalneyshey razrabotki temy.
\begin{enumerate}
  \item Na osnove analiza \ldots
  \item Chislennye issledovaniya pokazali, chto \ldots
  \item Matematicheskoe modelirovanie pokazalo \ldots
  \item Dlya vypolneniya postavlennykh zadach byl sozdan \ldots
\end{enumerate}


\pdfbookmark{Literatura}{bibliography}                                % Zakladka pdf
Pri ispolzovanii paketa \verb!biblatex! spisok publikatsiy avtora po teme
dissertatsii formiruetsya v razdele <<\publications>>\ fayla
\verb!common/characteristic.tex!  pri pomoshchi komandy \verb!\nocite!

\ifdefmacro{\microtypesetup}{\microtypesetup{protrusion=false}}{} % ne rekomenduetsya primenyat paket mikrotipografiki k avtomaticheski generiruemomu spisku literatury
\urlstyle{rm}                               % ssylki URL obychnym shriftom
\ifnumequal{\value{bibliosel}}{0}{% Vstroennaya realizatsiya s zagruzkoy fayla cherez dvizhok bibtex8
    \renewcommand{\bibname}{\large \bibtitleauthor}
    \nocite{*}
    \insertbiblioauthor           % Podklyuchaem Bib-bazy
    %\insertbiblioexternal   % !!! bibtex ne umeet rabotat s neskolkimi bibliografiyami !!!
}{% Realizatsiya paketom biblatex cherez dvizhok biber
    % Tsitirovaniya.
    %  * Poryadok perechisleniya opredelyaet poryadok v bibliografii (tolko vnutri podrazdela, esli `\insertbiblioauthorgrouped`).
    %  * Esli ne soblyudat poryadok "kak dlya \printbibliography", numeratsiya v `\insertbiblioauthor` budet krivoy.
    %  * Esli tsitirovat kazhdyy istochnik otdelnoy komandoy --- nayti nekotorye oshibki budet proshche.
    %
    %% authorvak
    \nocite{vakbib1}%
    \nocite{vakbib2}%
    \nocite{vakbib3}%
    \nocite{vakbib4}%
    \nocite{vakbib5}%
    \nocite{vakbib6}%
    \nocite{vakbib7}%
    \nocite{vakbib8}%
    \nocite{vakbib9}%
    \nocite{vakbib10}%
    \nocite{vakbib11}%
    \nocite{vakbib12}%
    %
    %% authorwos
    \nocite{wosbib1}%
    %
    %% authorscopus
    \nocite{scbib1}%
    %
    %% authorpatent
    \nocite{patbib1}%
    %
    %% authorprogram
    \nocite{progbib1}%
    %
    %% authorconf
    \nocite{confbib1}%
    \nocite{confbib2}%
    %
    %% authorother
    \nocite{bib1}%
    \nocite{bib2}%

    \ifnumgreater{\value{usefootcite}}{0}{
        \begin{refcontext}[labelprefix={}]
            \ifnum \value{bibgrouped}>0
                \insertbiblioauthorgrouped    % Vyvod vsekh rabot avtora, sgruppirovannykh po istochnikam
            \else
                \insertbiblioauthor      % Vyvod vsekh rabot avtora
            \fi
        \end{refcontext}
    }{
        \ifnum \totvalue{citeexternal}>0
            \begin{refcontext}[labelprefix=A]
                \ifnum \value{bibgrouped}>0
                    \insertbiblioauthorgrouped    % Vyvod vsekh rabot avtora, sgruppirovannykh po istochnikam
                \else
                    \insertbiblioauthor      % Vyvod vsekh rabot avtora
                \fi
            \end{refcontext}
        \else
            \ifnum \value{bibgrouped}>0
                \insertbiblioauthorgrouped    % Vyvod vsekh rabot avtora, sgruppirovannykh po istochnikam
            \else
                \insertbiblioauthor      % Vyvod vsekh rabot avtora
            \fi
        \fi
        %  \insertbiblioauthorimportant  % Vyvod naibolee znachimykh rabot avtora (opredelyaetsya v fayle characteristic vo vtoroy section)
        \begin{refcontext}[labelprefix={}]
            \insertbiblioexternal            % Vyvod spiska literatury, na kotoruyu ssylalis v tekste avtoreferata
        \end{refcontext}
        % Nevidimyy bibliograficheskiy spisok dlya podscheta kolichestva vneshnikh publikatsiy
        % Ispolzuetsya, chtoby ubrat pristavku "A" u rabot avtora, esli v avtoreferate net
        % tsitirovaniy vneshnikh istochnikov.
        \printbibliography[heading=nobibheading, section=0, env=countexternal, keyword=biblioexternal, resetnumbers=true]%
    }
}
\ifdefmacro{\microtypesetup}{\microtypesetup{protrusion=true}}{}
\urlstyle{tt}                               % vozvrashchaem ustanovki shrifta ssylok URL
