%%% Pereopredelenie imenovaniy, esli inache ne srabotaet %%%
%\gappto\captionsrussian{
%    \renewcommand{\chaptername}{Glava}
%    \renewcommand{\appendixname}{Prilozhenie} % (GOST R 7.0.11-2011, 5.7)
%}

%%% Izobrazheniya %%%
\graphicspath{{images/}{Dissertation/images/}}         % Puti k izobrazheniyam

%%% Intervaly %%%
%% Po GOST R 7.0.11-2011, punktu 5.3.6 trebuetsya polutornyy interval
%% Realizatsiya sredstvami klassa (na osnove setspace) blizhe k tipografskoy klassike.
%% I pravit srazu i v tablitsakh (esli so zvezdochkoy)
%\DoubleSpacing*     % Dvoynoy interval
\OnehalfSpacing*    % Polutornyy interval
%\setSpacing{1.42}   % Polutornyy interval, podobnyy Vordu (vozmozhno, stoit vklyuchat vmeste s predydushchey strokoy)

%%% Maket stranitsy %%%
% Vystavlyaem znacheniya poley (GOST 7.0.11-2011, 5.3.7)
\geometry{a4paper, top=2cm, bottom=2cm, left=2.5cm, right=1cm, nofoot, nomarginpar} %, heightrounded, showframe
\setlength{\topskip}{0pt}   %razmer dopolnitelnogo verkhnego polya
\setlength{\footskip}{12.3pt} % snimet warning, soglasno https://tex.stackexchange.com/a/334346

%%% Vyravnivanie i perenosy %%%
%% http://tex.stackexchange.com/questions/241343/what-is-the-meaning-of-fussy-sloppy-emergencystretch-tolerance-hbadness
%% http://www.latex-community.org/forum/viewtopic.php?p=70342#p70342
\tolerance 1414
\hbadness 1414
\emergencystretch 1.5em % V sluchae problem regulirovat v pervuyu ochered
\hfuzz 0.3pt
\vfuzz \hfuzz
%\raggedbottom
%\sloppy                 % Izbavlyaemsya ot perepolneniy
\clubpenalty=10000      % Zapreshchaem razryv stranitsy posle pervoy stroki abzatsa
\widowpenalty=10000     % Zapreshchaem razryv stranitsy posle posledney stroki abzatsa
\brokenpenalty=4991     % Ogranichenie na razryv stranitsy, esli stroka zakanchivaetsya perenosom

%%% Blok upravleniya parametrami dlya vyravnivaniya zagolovkov v tekste %%%
\newlength{\otstuplen}
\setlength{\otstuplen}{\theotstup\parindent}
\ifnumequal{\value{headingalign}}{0}{% vyravnivanie zagolovkov v tekste
    \newcommand{\hdngalign}{\centering}                % po tsentru
    \newcommand{\hdngaligni}{}% po tsentru
    \setlength{\otstuplen}{0pt}
}{%
    \newcommand{\hdngalign}{}                 % po levomu krayu
    \newcommand{\hdngaligni}{\hspace{\otstuplen}}      % po levomu krayu
} % V oboikh sluchayakh vrode by bez perenosa, kak i nado (GOST R 7.0.11-2011, 5.3.5)

%%% Oglavlenie %%%
\renewcommand{\cftchapterdotsep}{\cftdotsep}                % otbivka tochkami do nomera stranitsy nachala glavy/razdela

%% Perenosit slova v zagolovke ne dopuskaetsya (GOST R 7.0.11-2011, 5.3.5). Zagolovki v oglavlenii dolzhny tochno povtoryat zagolovki v tekste (GOST R 7.0.11-2011, 5.2.3). Pryamogo ukazaniya na zapret perenosov v oglavlenii net, no po toy zhe logike nevneseniya iskazheniy v smysl, luchshe v oglavlenii ne perenosit:
\setrmarg{2.55em plus1fil}                             %To have the (sectional) titles in the ToC, etc., typeset ragged right with no hyphenation
\renewcommand{\cftchapterpagefont}{\normalfont}        % nezhirnye nomera stranits u glav v oglavlenii
\renewcommand{\cftchapterleader}{\cftdotfill{\cftchapterdotsep}}% nezhirnye tochki do nomerov stranits u glav v oglavlenii
%\renewcommand{\cftchapterfont}{}                       % nezhirnye nazvaniya glav v oglavlenii

\ifnumgreater{\value{headingdelim}}{0}{%
    \renewcommand\cftchapteraftersnum{.\space}       % dobavlyaet tochku s probelom posle nomera razdela v oglavlenii
}{}
\ifnumgreater{\value{headingdelim}}{1}{%
    \renewcommand\cftsectionaftersnum{.\space}       % dobavlyaet tochku s probelom posle nomera podrazdela v oglavlenii
    \renewcommand\cftsubsectionaftersnum{.\space}    % dobavlyaet tochku s probelom posle nomera podpodrazdela v oglavlenii
    \renewcommand\cftsubsubsectionaftersnum{.\space} % dobavlyaet tochku s probelom posle nomera podpodpodrazdela v oglavlenii
    \AfterEndPreamble{% bez etogo polyglossia sama vse pereopredelyaet
        \setsecnumformat{\csname the#1\endcsname.\space}
    }
}{%
    \AfterEndPreamble{% bez etogo polyglossia sama vse pereopredelyaet
        \setsecnumformat{\csname the#1\endcsname\quad}
    }
}

\renewcommand*{\cftappendixname}{\appendixname\space} % Slovo Prilozhenie v oglavlenii

%%% Kolontituly %%%
% Poryadkovyy nomer stranitsy pechatayut na seredine verkhnego polya stranitsy (GOST R 7.0.11-2011, 5.3.8)
\makeevenhead{plain}{}{\rmfamily\thepage}{}
\makeoddhead{plain}{}{\rmfamily\thepage}{}
\makeevenfoot{plain}{}{}{}
\makeoddfoot{plain}{}{}{}
\pagestyle{plain}

%%% dobavit Str. nad nomerami stranits v oglavlenii
%%% http://tex.stackexchange.com/a/306950
\newif\ifendTOC

\newcommand*{\tocheader}{
\ifnumequal{\value{pgnum}}{1}{%
    \ifendTOC\else\hbox to \linewidth%
      {\noindent{}~\hfill{Str.}}\par%
      \ifnumless{\value{page}}{3}{}{%
        \vspace{0.5\onelineskip}
      }
      \afterpage{\tocheader}
    \fi%
}{}%
}%

%%% Oformlenie zagolovkov glav, razdelov, podrazdelov %%%
%% Rabota dolzhna byt vypolnena ... razmerom shrifta 12-14 punktov (GOST R 7.0.11-2011, 5.3.8). To est ne dolzhno byt nadpisey shriftom bolee 14. Tak i postavim.
%% Eti ustanovki budut davat odinakovyy rezultat nezavisimo ot vybora bazovym shriftom 12 pt ili 14 pt
\newcommand{\basegostsectionfont}{\fontsize{14pt}{16pt}\selectfont\bfseries}

\makechapterstyle{thesisgost}{%
    \chapterstyle{default}
    \setlength{\beforechapskip}{0pt}
    \setlength{\midchapskip}{0pt}
    \setlength{\afterchapskip}{\theintvl\curtextsize}
    \renewcommand*{\chapnamefont}{\basegostsectionfont}
    \renewcommand*{\chapnumfont}{\basegostsectionfont}
    \renewcommand*{\chaptitlefont}{\basegostsectionfont}
    \renewcommand*{\chapterheadstart}{}
    \ifnumgreater{\value{headingdelim}}{0}{%
        \renewcommand*{\afterchapternum}{.\space}   % dobavlyaet tochku s probelom posle nomera razdela
    }{%
        \renewcommand*{\afterchapternum}{\quad}     % dobavlyaet \quad posle nomera razdela
    }
    \renewcommand*{\printchapternum}{\hdngaligni\hdngalign\chapnumfont \thechapter}
    \renewcommand*{\printchaptername}{}
    \renewcommand*{\printchapternonum}{\hdngaligni\hdngalign}
}

\makeatletter
\makechapterstyle{thesisgostchapname}{%
    \chapterstyle{thesisgost}
    \renewcommand*{\printchapternum}{\chapnumfont \thechapter}
    \renewcommand*{\printchaptername}{\hdngaligni\hdngalign\chapnamefont \@chapapp} %
}
\makeatother

\chapterstyle{thesisgost}

\setsecheadstyle{\basegostsectionfont\hdngalign}
\setsecindent{\otstuplen}

\setsubsecheadstyle{\basegostsectionfont\hdngalign}
\setsubsecindent{\otstuplen}

\setsubsubsecheadstyle{\basegostsectionfont\hdngalign}
\setsubsubsecindent{\otstuplen}

\sethangfrom{\noindent #1} %vse zagolovki podrazdelov tsentriruyutsya s uchetom nomera, kak block

\ifnumequal{\value{chapstyle}}{1}{%
    \chapterstyle{thesisgostchapname}
    \renewcommand*{\cftchaptername}{\chaptername\space} % budet vpisano slovo Glava pered kazhdym nomerom razdela v oglavlenii
}{}%

%%% Intervaly mezhdu zagolovkami
\setbeforesecskip{\theintvl\curtextsize}% Zagolovki otdelyayut ot teksta sverkhu i snizu tremya intervalami (GOST R 7.0.11-2011, 5.3.5).
\setaftersecskip{\theintvl\curtextsize}
\setbeforesubsecskip{\theintvl\curtextsize}
\setaftersubsecskip{\theintvl\curtextsize}
\setbeforesubsubsecskip{\theintvl\curtextsize}
\setaftersubsubsecskip{\theintvl\curtextsize}

%%% Vertikalnye intervaly glav (\chapter) v oglavlenii kak i u zagolovkov
% raskommentirovat sleduyushchie 2
% \setlength{\cftbeforechapterskip}{0pt plus 0pt}   % ILI eti 2 stroki iz uchebnika
% \renewcommand*{\insertchapterspace}{}
% ili etu
% \renewcommand*{\cftbeforechapterskip}{0em}


%%% Blok dopolnitelnogo upravleniya razmerami zagolovkov
\ifnumequal{\value{headingsize}}{1}{% Proportsionalnye zagolovki i bazovyy shrift 14 pt
    \renewcommand{\basegostsectionfont}{\large\bfseries}
    \renewcommand*{\chapnamefont}{\Large\bfseries}
    \renewcommand*{\chapnumfont}{\Large\bfseries}
    \renewcommand*{\chaptitlefont}{\Large\bfseries}
}{}

%%% Schetchiki %%%

%% Uproshchennye nastroyki shablona dissertatsii: numeratsiya formul, tablits, risunkov
\ifnumequal{\value{contnumeq}}{1}{%
    \counterwithout{equation}{chapter} % Ubiraem svyazannost nomera formuly s nomerom glavy/razdela
}{}
\ifnumequal{\value{contnumfig}}{1}{%
    \counterwithout{figure}{chapter}   % Ubiraem svyazannost nomera risunka s nomerom glavy/razdela
}{}
\ifnumequal{\value{contnumtab}}{1}{%
    \counterwithout{table}{chapter}    % Ubiraem svyazannost nomera tablitsy s nomerom glavy/razdela
}{}

\AfterEndPreamble{
%% registriruem schetchiki v sisteme totcounter
    \regtotcounter{totalcount@figure}
    \regtotcounter{totalcount@table}       % Esli inym sposobom postavit v preambule to oshibka v chisle tablits
    \regtotcounter{TotPages}               % Esli inym sposobom postavit v preambule to oshibka v chisle stranits
    \newtotcounter{totalappendix}
    \newtotcounter{totalchapter}
}
