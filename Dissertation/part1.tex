\chapter{Oformlenie razlichnykh elementov}\label{ch:ch1}

\section{Formatirovanie teksta}\label{sec:ch1/sec1}

My mozhem sdelat \textbf{zhirnyy tekst} i \textit{kursiv}.

\section{Ssylki}\label{sec:ch1/sec2}

Soshlemsya na bibliografiyu.
Odna ssylka: \cite[s.~54]{Sokolov}\cite[s.~36]{Gaidaenko}.
Dve ssylki: \cite{Sokolov,Gaidaenko}.
Ssylka na sobstvennye raboty: \cite{vakbib1, confbib2}.
Mnogo ssylok: %\cite[s.~54]{Lermontov,Management,Borozda} % takoy «fokus»
%vyzyvaet biblatex warning otnositelno optsii sortcites, potomu chto neyasno, k
%kakomu istochniku otnositsya utochnenie o stranitsakh, a bibtex ob etoy probleme
%dazhe ne preduprezhdaet
\cite{Lermontov, Management, Borozda, Marketing, Constitution, FamilyCode,
    Gost.7.0.53, Razumovski, Lagkueva, Pokrovski, Methodology, Berestova,
    Kriger}%
\ifnumequal{\value{bibliosel}}{0}{% Primery dlya bibtex8
    \cite{Sirotko, Lukina, Encyclopedia, Nasirova}%
}{% Primery dlya biblatex cherez dvizhok biber
    \cite{Sirotko2, Lukina2, Encyclopedia2, Nasirova2}%
}%
.
I~eshche nemnogo ssylok:~\cite{Article,Book,Booklet,Conference,Inbook,Incollection,Manual,Mastersthesis,
    Misc,Phdthesis,Proceedings,Techreport,Unpublished}
% Sleduet obratit vnimanie, chto probel posle zapyatoy vnutri \cite{}
% obrabatyvaetsya ozhidaemo, a probel pered zapyatoy, mozhet vyzyvat problemy pri
% obrabotke ssylok.
\cite{medvedev2006jelektronnye, CEAT:CEAT581, doi:10.1080/01932691.2010.513279,
    Gosele1999161,Li2007StressAnalysis, Shoji199895, test:eisner-sample,
    test:eisner-sample-shorted, AB_patent_Pomerantz_1968, iofis_patent1960}%
\ifnumequal{\value{bibliosel}}{0}{% Primery dlya bibtex8
}{% Primery dlya biblatex cherez dvizhok biber
    \cite{patent2h, patent3h, patent2}%
}%
.

\ifnumequal{\value{bibliosel}}{0}{% Primery dlya bibtex8
Popytka realizovat neskolko ssylok na konkretnye stranitsy
dlya \texttt{bibtex} realizatsii bibliografii:
[\citenum{Sokolov}, s.~54; \citenum{Gaidaenko}, s.~36].
}{% Primery dlya biblatex cherez dvizhok biber
Neskolko istochnikov (multitsitata):
% Tut spetsialno napisano po-raznomu tire, dlya demonstratsii, chto
% primenenie spetsialnykh tire v nastoyashchiy moment v biblatex privodit k nepokazu
% "s.".
\cites[vii--x, 5, 7]{Sokolov}[v"--~x, 25, 526]{Gaidaenko}[vii--x, 5, 7]{Techreport},
rabotaet tolko v \texttt{biblatex} realizatsii bibliografii.
}%

Ssylki na sobstvennye raboty:~\cite{vakbib1, confbib1}.

Soshlemsya na prilozheniya: Prilozhenie~\cref{app:A}, Prilozhenie~\cref{app:B2}.

Soshlemsya na formulu: formula~\cref{eq:equation1}.

Soshlemsya na izobrazhenie: risunok~\cref{fig:knuth}.

Standartnoy praktikoy yavlyaetsya dobavlenie k ssylkam prefiksa, kharakterizuyushchego tip elementa.
Eto ne yavlyaetsya strogim trebovaniem, no~pozvolyaet luchshe orientirovatsya v dokumentakh bolshogo razmera.
Naprimer, dlya ssylok na~risunki ispolzuetsya prefiks \textit{fig},
dlya ssylki na~tablitsu "--- \textit{tab}.

V tablitse \cref{tab:tab_pref} prilozheniya~\cref{app:B4} priveden spisok rekomenduemykh
k ispolzovaniyu standartnykh prefiksov.

V nekotorykh situatsiyakh voznikaet neobkhodimost otoyti ot trebovaniy GOST po oformleniyu ssylok na
literaturu.
V takom sluchae mozhno vospolzovatsya dopolnitelnymi optsiyami paketa \verb+biblatex+.

Naprimer, v ssylke na knigu~\cite{sobenin_kdv} ispolzovanie optsii \verb+maxnames=4+ pozvolyaet
vyvesti imena vsekh chetyrekh avtorov.
Po GOST imena poslednikh trekh avtorov opuskayutsya.

Krome togo, chasto voznikayut problemy s transliterovannymi initsialami. Nekotorye bukvy russkogo
alfavita po pravilam transliteratsii zapisyvayutsya dvumya bukvami latinskogo alfavita (yu-yu, e-yo i
t.d.).
Takie initsialy \verb+biblatex+ budet sokrashchat do odnoy bukvy, chto neverno.
Popravit ego rabotu mozhno ispolzovav optsiyu \verb+giveninits=false+.
Primer ispolzovaniya etoy optsii mozhno videt v ssylke~\cite{initials}.

\section{Formuly}\label{sec:ch1/sec3}

Blagodarya paketu \textit{icomma}, \LaTeX~odinakovo khorosho vosprinimaet
v~kachestve desyatichnogo razdelitelya i zapyatuyu (\(3,1415\)), i tochku (\(3.1415\)).

\subsection{Nenumerovannye odinochnye formuly}\label{subsec:ch1/sec3/sub1}

Vot tak mozhet vyglyadet formula, kotoruyu neobkhodimo vstavit v~stroku
po~tekstu: \(x \approx \sin x\) pri \(x \to 0\).

A vot tak vyglyadit nenumerovannaya otdelnostoyashchaya formula c podstrochnymi
i nadstrochnymi indeksami:
\[
    (x_1+x_2)^2 = x_1^2 + 2 x_1 x_2 + x_2^2
\]

Formula s neopredelennym integralom:
\[
    \int f(\alpha+x)=\sum\beta
\]

Pri ispolzovanii drobey formuly mogut poluchatsya ochen vysokie:
\[
    \frac{1}{\sqrt{2}+
        \displaystyle\frac{1}{\sqrt{2}+
            \displaystyle\frac{1}{\sqrt{2}+\cdots}}}
\]

V formulakh mozhno ispolzovat grecheskie bukvy:
%Vse \original... komandy zaranee, radi etogo primera, opredeleny v Dissertation\userstyles.tex
\[
    \alpha\beta\gamma\delta\originalepsilon\epsilon\zeta\eta\theta%
    \vartheta\iota\kappa\varkappa\lambda\mu\nu\xi\pi\varpi\rho\varrho%
    \sigma\varsigma\tau\upsilon\originalphi\phi\chi\psi\omega\Gamma\Delta%
    \Theta\Lambda\Xi\Pi\Sigma\Upsilon\Phi\Psi\Omega
\]
\[%https://texfaq.org/FAQ-boldgreek
    \boldsymbol{\alpha\beta\gamma\delta\originalepsilon\epsilon\zeta\eta%
        \theta\vartheta\iota\kappa\varkappa\lambda\mu\nu\xi\pi\varpi\rho%
        \varrho\sigma\varsigma\tau\upsilon\originalphi\phi\chi\psi\omega\Gamma%
        \Delta\Theta\Lambda\Xi\Pi\Sigma\Upsilon\Phi\Psi\Omega}
\]

Dlya dobavleniya formul mozhno ispolzovat pary \verb+$+\dots\verb+$+ i \verb+$$+\dots\verb+$$+,
no~oni schitayutsya ustarevshimi.
Luchshe ispolzovat ikh funktsionalnye analogi \verb+\(+\dots\verb+\)+ i \verb+\[+\dots\verb+\]+.

\subsection{Nenumerovannye mnogostrochnye formuly}\label{subsec:ch1/sec3/sub2}

Vot tak mozhno napisat dve formuly, ne numeruya ikh, chtoby znaki <<ravno>> byli
strogo drug pod drugom:
\begin{align}
    f_W & =  \min \left( 1, \max \left( 0, \frac{W_{soil} / W_{max}}{W_{crit}} \right)  \right), \nonumber \\
    f_T & =  \min \left( 1, \max \left( 0, \frac{T_s / T_{melt}}{T_{crit}} \right)  \right), \nonumber
\end{align}

Vyrovnyat sistemu eshche i po peremennoy \( x \) mozhno, ispolzuya okruzhenie
\verb|alignedat| iz paketa \verb|amsmath|. Vot tak:
\[
|x| = \left\{
\begin{alignedat}{2}
     &   & x, \quad & \text{esli } x\geqslant 0 \\
     & - & x, \quad & \text{esli } x<0
\end{alignedat}
\right.
\]
Zdes pervyy ampersand (v iskhodnom \LaTeX\ opisanii formuly) oznachaet
vyravnivanie po~levomu krayu, vtoroy "--- po~\( x \), a~tretiy "--- po~slovu
<<esli>>. Komanda \verb|\quad| delaet bolshoy gorizontalnyy probel.

Eshche variant:
\[
    |x|=
    \begin{cases}
        \phantom{-}x, \text{esli } x \geqslant 0 \\
        -x, \text{esli } x<0
    \end{cases}
\]

Krome togo, dlya  numerovannykh formul \verb|alignedat| delaet vertikalnoe
vyravnivanie nomera formuly po tsentru formuly. Naprimer, vyravnivanie
komponent vektora:
\begin{equation}
    \label{eq:2p3}
    \begin{alignedat}{2}
        {\mathbf{N}}_{o1n}^{(j)} = \,{\sin} \phi\,n\!\left(n+1\right)
        {\sin}\theta\,
        \pi_n\!\left({\cos} \theta\right)
        \frac{
        z_n^{(j)}\!\left( \rho \right)
        }{\rho}\,
         & {\boldsymbol{\hat{\mathrm e}}}_{r}\,+      \\
        +\,
        {\sin} \phi\,
        \tau_n\!\left({\cos} \theta\right)
        \frac{
        \left[\rho z_n^{(j)}\!\left( \rho \right)\right]^{\prime}
        }{\rho}\,
         & {\boldsymbol{\hat{\mathrm e}}}_{\theta}\,+ \\
        +\,
        {\cos} \phi\,
        \pi_n\!\left({\cos} \theta\right)
        \frac{
        \left[\rho z_n^{(j)}\!\left( \rho \right)\right]^{\prime}
        }{\rho}\,
         & {\boldsymbol{\hat{\mathrm e}}}_{\phi}\:.
    \end{alignedat}
\end{equation}

Eshche ob otstupakh. Inogda dlya luchshey <<chitaemosti>> formul polezno
nemnogo ispravit standartnye intervaly \LaTeX\ s uchetom logicheskoy
struktury samoy formuly. Naprimer v formule~\cref{eq:2p3} dobavlen
nebolshoy otstup \verb+\,+ mezhdu osnovnymi somnozhitelyami, nizhe
rezultat primeneniya vsekh variantov otstupa:
\begin{align*}
    \backslash!             & \quad f(x) = x^2\! +3x\! +2         \\
    \mbox{po-umolchaniyu}     & \quad f(x) = x^2+3x+2               \\
    \backslash,             & \quad f(x) = x^2\, +3x\, +2         \\
    \backslash{:}           & \quad f(x) = x^2\: +3x\: +2         \\
    \backslash;             & \quad f(x) = x^2\; +3x\; +2         \\
    \backslash \mbox{space} & \quad f(x) = x^2\ +3x\ +2           \\
    \backslash \mbox{quad}  & \quad f(x) = x^2\quad +3x\quad +2   \\
    \backslash \mbox{qquad} & \quad f(x) = x^2\qquad +3x\qquad +2
\end{align*}

Mozhno ispolzovat raznye matematicheskie alfavity:
\begin{align}
    \mathcal{ABCDEFGHIJKLMNOPQRSTUVWXYZ} \nonumber  \\
    \mathfrak{ABCDEFGHIJKLMNOPQRSTUVWXYZ} \nonumber \\
    \mathbb{ABCDEFGHIJKLMNOPQRSTUVWXYZ} \nonumber
\end{align}

Posmotrim na sistemu uravneniy na primere attraktora Lorentsa:

\[
\left\{
\begin{array}{rl}
    \dot x = & \sigma (y-x)  \\
    \dot y = & x (r - z) - y \\
    \dot z = & xy - bz
\end{array}
\right.
\]

A dlya verstki matrits udobno ispolzovat mnogotochiya:
\[
    \left(
        \begin{array}{ccc}
            a_{11} & \ldots & a_{1n} \\
            \vdots & \ddots & \vdots \\
            a_{n1} & \ldots & a_{nn} \\
        \end{array}
    \right)
\]

\subsection{Numerovannye formuly}\label{subsec:ch1/sec3/sub3}

A vot tak pishetsya numerovannaya formula:
\begin{equation}
    \label{eq:equation1}
    e = \lim_{n \to \infty} \left( 1+\frac{1}{n} \right) ^n
\end{equation}

Numerovannykh formul mozhet byt neskolko:
\begin{equation}
    \label{eq:equation2}
    \lim_{n \to \infty} \sum_{k=1}^n \frac{1}{k^2} = \frac{\pi^2}{6}
\end{equation}

Vposledstvii na formuly~\cref{eq:equation1, eq:equation2} mozhno ssylatsya.

Sdelat tak, chtoby nomer formuly stoyal naprotiv sredney stroki, mozhno,
ispolzuya okruzhenie \verb|multlined| (paket \verb|mathtools|) vmesto
\verb|multline| vnutri okruzheniya \verb|equation|. Vot tak:
\begin{equation} % \tag{S} % tag - vpisyvaet svoy tekst
    \label{eq:equation3}
    \begin{multlined}
        1+ 2+3+4+5+6+7+\dots + \\
        + 50+51+52+53+54+55+56+57 + \dots + \\
        + 96+97+98+99+100=5050
    \end{multlined}
\end{equation}

Uravneniya~\cref{eq:subeq_1,eq:subeq_2} demonstriruyut vozmozhnosti
okruzheniya \verb|subequations| (paket \verb|amsmath|).
\begin{subequations}
    \label{eq:subeq_1}
    \begin{gather}
        y = x^2 + 1 \label{eq:subeq_1-1} \\
        y = 2 x^2 - x + 1 \label{eq:subeq_1-2}
    \end{gather}
\end{subequations}
Ssylki na otdelnye uravneniya~\cref{eq:subeq_1-1,eq:subeq_1-2,eq:subeq_2-1}.
\begin{subequations}
    \label{eq:subeq_2}
    \begin{align}
        y & = x^3 + x^2 + x + 1 \label{eq:subeq_2-1} \\
        y & = x^2
    \end{align}
\end{subequations}

\subsection{Formatirovanie chisel i razmernostey velichin}\label{sec:units}

Chisla formatiruyutsya pri pomoshchi komandy \verb|\num|:
\num{5,3};
\num{2,3e8};
\num{12345,67890};
\num{2,6 d4};
\num{1+-2i};
\num{.3e45};
\num[exponent-base=2]{5 e64};
\num[exponent-base=2,exponent-to-prefix]{5 e64};
\num{1.654 x 2.34 x 3.430}
\num{1 2 x 3 / 4}.
Dlya napisaniya posledovatelnosti chisel mozhno ispolzovat komandy \verb|\numlist| i \verb|\numrange|:
\numlist{10;30;50;70}; \numrange{10}{30}.
Znacheniya uglov mozhno formatirovat pri pomoshchi komandy \verb|\ang|:
\ang{2.67};
\ang{30,3};
\ang{-1;;};
\ang{;-2;};
\ang{;;-3};
\ang{300;10;1}.

Obratite vnimanie, chto GOST zapreshchaet ispolzovanie znaka <<->> dlya oboznacheniya otritsatelnykh chisel
za isklyucheniem formul, tablits i~risunkov.
Vmesto nego sleduet ispolzovat slovo <<minus>>.

Razmernosti mozhno zapisyvat pri pomoshchi komand \verb|\si| i \verb|\SI|:
\si{\farad\squared\lumen\candela};
\si{\joule\per\mole\per\kelvin};
\si[per-mode = symbol-or-fraction]{\joule\per\mole\per\kelvin};
\si{\metre\per\second\squared};
\SI{0.10(5)}{\neper};
\SI{1.2-3i e5}{\joule\per\mole\per\kelvin};
\SIlist{1;2;3;4}{\tesla};
\SIrange{50}{100}{\volt}.
Spisok edinits izmereniy priveden v tablitsakh~\cref{tab:unit:base,
    tab:unit:derived,tab:unit:accepted,tab:unit:physical,tab:unit:other}.
Pristavki edinits privedeny v~tablitse~\cref{tab:unit:prefix}.

S dopolnitelnymi optsiyami formatirovaniya mozhno oznakomitsya v~opisanii paketa \texttt{siunitx};
izmenit ili dobavit edinitsy izmereniy mozhno v~fayle \texttt{siunitx.cfg}.

\begin{table}
    \centering
    \captionsetup{justification=centering} % vyravnivanie podpisi po-tsentru
    \caption{Osnovnye velichiny SI}\label{tab:unit:base}
    \begin{tabular}{llc}
        \toprule
        Nazvanie  & Komanda          & Simvol         \\
        \midrule
        Amper     & \verb|\ampere|   & \si{\ampere}   \\
        Kandela   & \verb|\candela|  & \si{\candela}  \\
        Kelvin   & \verb|\kelvin|   & \si{\kelvin}   \\
        Kilogramm & \verb|\kilogram| & \si{\kilogram} \\
        Metr      & \verb|\metre|    & \si{\metre}    \\
        Mol      & \verb|\mole|     & \si{\mole}     \\
        Sekunda   & \verb|\second|   & \si{\second}   \\
        \bottomrule
    \end{tabular}
\end{table}

\begin{table}
    \small
    \centering
    \begin{threeparttable}% vyravnivanie podpisi po granitsam tablitsy
        \caption{Proizvodnye edinitsy SI}\label{tab:unit:derived}
        \begin{tabular}{llc|llc}
            \toprule
            Nazvanie       & Komanda               & Simvol              & Nazvanie & Komanda & Simvol \\
            \midrule
            Bekkerel      & \verb|\becquerel|     & \si{\becquerel}     &
            Nyuton         & \verb|\newton|        & \si{\newton}                                      \\
            Gradus Tselsiya & \verb|\degreeCelsius| & \si{\degreeCelsius} &
            Om             & \verb|\ohm|           & \si{\ohm}                                         \\
            Kulon          & \verb|\coulomb|       & \si{\coulomb}       &
            Paskal        & \verb|\pascal|        & \si{\pascal}                                      \\
            Farad          & \verb|\farad|         & \si{\farad}         &
            Radian         & \verb|\radian|        & \si{\radian}                                      \\
            Grey           & \verb|\gray|          & \si{\gray}          &
            Simens         & \verb|\siemens|       & \si{\siemens}                                     \\
            Gerts           & \verb|\hertz|         & \si{\hertz}         &
            Zivert         & \verb|\sievert|       & \si{\sievert}                                     \\
            Genri          & \verb|\henry|         & \si{\henry}         &
            Steradian      & \verb|\steradian|     & \si{\steradian}                                   \\
            Dzhoul         & \verb|\joule|         & \si{\joule}         &
            Tesla          & \verb|\tesla|         & \si{\tesla}                                       \\
            Katal          & \verb|\katal|         & \si{\katal}         &
            Volt          & \verb|\volt|          & \si{\volt}                                        \\
            Lyumen          & \verb|\lumen|         & \si{\lumen}         &
            Vatt           & \verb|\watt|          & \si{\watt}                                        \\
            Lyuks           & \verb|\lux|           & \si{\lux}           &
            Veber          & \verb|\weber|         & \si{\weber}                                       \\
            \bottomrule
        \end{tabular}
    \end{threeparttable}
\end{table}

\begin{table}
    \centering
    \begin{threeparttable}% vyravnivanie podpisi po granitsam tablitsy
        \caption{Vnesistemnye edinitsy}\label{tab:unit:accepted}

        \begin{tabular}{llc}
            \toprule
            Nazvanie        & Komanda           & Simvol          \\
            \midrule
            Den            & \verb|\day|       & \si{\day}       \\
            Gradus          & \verb|\degree|    & \si{\degree}    \\
            Gektar          & \verb|\hectare|   & \si{\hectare}   \\
            Chas             & \verb|\hour|      & \si{\hour}      \\
            Litr            & \verb|\litre|     & \si{\litre}     \\
            Uglovaya minuta  & \verb|\arcminute| & \si{\arcminute} \\
            Uglovaya sekunda & \verb|\arcsecond| & \si{\arcsecond} \\ %
            Minuta          & \verb|\minute|    & \si{\minute}    \\
            Tonna           & \verb|\tonne|     & \si{\tonne}     \\
            \bottomrule
        \end{tabular}
    \end{threeparttable}
\end{table}

\begin{table}
    \centering
    \captionsetup{justification=centering}
    \caption{Vnesistemnye edinitsy, poluchaemye iz eksperimenta}\label{tab:unit:physical}
    \begin{tabular}{llc}
        \toprule
        Nazvanie                & Komanda                  & Simvol                 \\
        \midrule
        Astronomicheskaya edinitsa & \verb|\astronomicalunit| & \si{\astronomicalunit} \\
        Atomnaya edinitsa massy   & \verb|\atomicmassunit|   & \si{\atomicmassunit}   \\
        Borovskiy radius        & \verb|\bohr|             & \si{\bohr}             \\
        Skorost sveta          & \verb|\clight|           & \si{\clight}           \\
        Dalton                 & \verb|\dalton|           & \si{\dalton}           \\
        Massa elektrona         & \verb|\electronmass|     & \si{\electronmass}     \\
        Elektron Volt          & \verb|\electronvolt|     & \si{\electronvolt}     \\
        Elementarnyy zaryad      & \verb|\elementarycharge| & \si{\elementarycharge} \\
        Energiya Khartri          & \verb|\hartree|          & \si{\hartree}          \\
        Postoyannaya Planka       & \verb|\planckbar|        & \si{\planckbar}        \\
        \bottomrule
    \end{tabular}
\end{table}

\begin{table}
    \centering
    \begin{threeparttable}% vyravnivanie podpisi po granitsam tablitsy
        \caption{Drugie vnesistemnye edinitsy}\label{tab:unit:other}
        \begin{tabular}{llc}
            \toprule
            Nazvanie                  & Komanda              & Simvol             \\
            \midrule
            Angstrem                  & \verb|\angstrom|     & \si{\angstrom}     \\
            Bar                       & \verb|\bar|          & \si{\bar}          \\
            Barn                      & \verb|\barn|         & \si{\barn}         \\
            Bel                       & \verb|\bel|          & \si{\bel}          \\
            Detsibel                   & \verb|\decibel|      & \si{\decibel}      \\
            Uzel                      & \verb|\knot|         & \si{\knot}         \\
            Millimetr rtutnogo stolba & \verb|\mmHg|         & \si{\mmHg}         \\
            Morskaya milya              & \verb|\nauticalmile| & \si{\nauticalmile} \\
            Neper                     & \verb|\neper|        & \si{\neper}        \\
            \bottomrule
        \end{tabular}
    \end{threeparttable}
\end{table}

\begin{table}
    \small
    \centering
    \begin{threeparttable}% vyravnivanie podpisi po granitsam tablitsy
        \caption{Pristavki SI}\label{tab:unit:prefix}
        \begin{tabular}{llcc|llcc}
            \toprule
            Pristavka & Komanda       & Simvol      & Stepen      &
            Pristavka & Komanda       & Simvol      & Stepen        \\
            \midrule
            Iokto     & \verb|\yocto| & \si{\yocto} & \textminus24 &
            Deka      & \verb|\deca|  & \si{\deca}  & 1              \\
            Zepto     & \verb|\zepto| & \si{\zepto} & \textminus21 &
            Gekto     & \verb|\hecto| & \si{\hecto} & 2              \\
            Atto      & \verb|\atto|  & \si{\atto}  & \textminus18 &
            Kilo      & \verb|\kilo|  & \si{\kilo}  & 3              \\
            Femto     & \verb|\femto| & \si{\femto} & \textminus15 &
            Mega      & \verb|\mega|  & \si{\mega}  & 6              \\
            Piko      & \verb|\pico|  & \si{\pico}  & \textminus12 &
            Giga      & \verb|\giga|  & \si{\giga}  & 9              \\
            Nano      & \verb|\nano|  & \si{\nano}  & \textminus9  &
            Terra     & \verb|\tera|  & \si{\tera}  & 12             \\
            Mikro     & \verb|\micro| & \si{\micro} & \textminus6  &
            Peta      & \verb|\peta|  & \si{\peta}  & 15             \\
            Milli     & \verb|\milli| & \si{\milli} & \textminus3  &
            Eksa      & \verb|\exa|   & \si{\exa}   & 18             \\
            Santi     & \verb|\centi| & \si{\centi} & \textminus2  &
            Zetta     & \verb|\zetta| & \si{\zetta} & 21             \\
            Detsi      & \verb|\deci|  & \si{\deci}  & \textminus1  &
            Iotta     & \verb|\yotta| & \si{\yotta} & 24             \\
            \bottomrule
        \end{tabular}
    \end{threeparttable}
\end{table}

\subsection{Zagolovki s formulami: \texorpdfstring{\(a^2 + b^2 = c^2\)}{%
        a\texttwosuperior\ + b\texttwosuperior\ = c\texttwosuperior},
    \texorpdfstring{\(\left\vert\textrm{{Im}}\Sigma\left(
            \protect\varepsilon\right)\right\vert\approx const\)}{|ImΣ (ε)| ≈ const},
    \texorpdfstring{\(\sigma_{xx}^{(1)}\)}{σ\_\{xx\}\textasciicircum\{(1)\}}
}\label{subsec:with_math}

Paket \texttt{hyperref} beret tekst dlya zakladok v pdf-fayle iz~argumentov
komand tipa \verb|\section|, kotorye mogut soderzhat matematicheskie formuly,
a~takzhe izmeneniya tsveta teksta ili shrifta, kotorye ne otobrazhayutsya v~zakladkakh.
Chtoby ispolzovanie formul v zagolovkakh ne vyzyvalo v~loge kompilyatsii poyavlenie
preduprezhdeniy tipa <<\texttt{Token not allowed in~a~PDF string
    (Unicode):(hyperref) removing...}>>, sleduet ispolzovat konstruktsiyu
\verb|\texorpdfstring{}{}|, gde v~pervykh figurnykh skobkakh ukazyvaetsya
formula, a~vo~vtorykh "--- zapis formuly dlya zakladok.

\section{Retsenzirovanie teksta}\label{sec:markup}

V shablone dlya dissertatsii i avtoreferata zadany komandy retsenzirovaniya.
Oni vidny pri kompilyatsii shablona v rezhime chernovika ili pri ustanovke
sootvetstvuyushchey nastroyki (\verb+showmarkup+) v~fayle \verb+common/setup.tex+.

Komanda \verb+\todo+ otmechaet tekst krasnym tsvetom.
\todo{Naprimer, tak.}

Komanda \verb+\note+ pozvolyaet vybrat tsvet teksta.
\note{Chernyy, } \note[red]{krasnyy, } \note[green]{zelenyy, }
\note[blue]{siniy.} \note[orange]{Obratite vnimanie na shirinu i rasstanovku
    formiruyushchikhsya probelov, v~rezultate privedennoy zapisi (zavisit takzhe
    ot~primenyaemogo kompilyatora).}

Okruzhenie \verb+commentbox+ takzhe pozvolyaet vybrat tsvet.

\begin{commentbox}[red]
    Krasnyy tekst.

    Neskolko paragrafov krasnogo teksta.
\end{commentbox}

\begin{commentbox}[blue]
    Sinyaya formula.

    \begin{equation}
        \alpha + \beta = \gamma
    \end{equation}
\end{commentbox}

\verb+commentbox+ pozvolyaet zakommentirovat uchastok koda v~rezhime chistovika.
Chtoby ubrat kusok koda dlya vsekh rezhimov, mozhno ispolzovat okruzhenie
\verb+comment+.

\begin{comment}
Etot tekst vsegda skryt.
\end{comment}

\section{Rabota so spiskom sokrashcheniy i~uslovnykh oboznacheniy}\label{sec:acronyms}

S pomoshchyu paketa \texttt{nomencl} mozhno sozdavat udobnyy sortirovannyy spisok
sokrashcheniy i uslovnykh oboznacheniy vo vremya napisaniya teksta. Vyzov
\verb+\nomenclature+ dobavlyaet nuzhnyy simvol ili sokrashchenie s~opisaniem
v~spisok, kotoryy zatem pechataetsya vyzovom \verb+\printnomenclature+
v~sootvetstvuyushchem razdele.
Dlya togo, chtoby eti operatsii proshli, potrebuetsya dopolnitelnyy vyzov
\verb+makeindex -s nomencl.ist -o %.nls %.nlo+ v~komandnoy stroke, gde vmesto
\verb+%+ sleduet podstavit imya glavnogo fayla proekta (\verb+dissertation+
dlya etogo shablona).
Zatem potrebuetsya odin ili dva dopolnitelnykh vyzova kompilyatora proekta.
\begin{equation}
    \omega = c k,
\end{equation}
gde \( \omega \) "--- chastota sveta, \( c \) "--- skorost sveta, \( k \) "---
modul volnovogo vektora.
\nomenclature{\(\omega\)}{chastota sveta\nomrefeq}
\nomenclature{\(c\)}{skorost sveta\nomrefpage}
\nomenclature{\(k\)}{modul volnovogo vektora\nomrefeqpage}
Ispolzovanie
\begin{verbatim}
\nomenclature{\(\omega\)}{chastota sveta\nomrefeq}
\nomenclature{\(c\)}{skorost sveta\nomrefpage}
\nomenclature{\(k\)}{modul volnovogo vektora\nomrefeqpage}
\end{verbatim}
posle uravneniya dobavit v spisok uslovnykh oboznacheniy tri zapisi.
Ssylki \verb+\nomrefeq+ na poslednee uravnenie, \verb+\nomrefpage+ "--- na
stranitsu, \verb+\nomrefeqpage+ "--- srazu na~poslednee uravnenie i~na~stranitsu,
mozhno opuskat i~ne~ispolzovat.

Gruppirovkoy i sortirovkoy punktov v spiske mozhno upravlyat s~pomoshchyu ukazaniya
dopolnitelnykh argumentov k komande \verb+nomenclature+.
Naprimer, pri vyzove
\begin{verbatim}
\nomenclature[03]{\( \hbar \)}{postoyannaya Planka}
\nomenclature[01]{\( G \)}{gravitatsionnaya postoyannaya}
\end{verbatim}
\( G \) budet stoyat v spiske vyshe, chem \( \hbar \).
Dlya korrektnykh vertikalnykh otstupov mezhdu strokami v opisanii luchshe
ne~ispolzovat mnogostrochnye formuly v~spiske oboznacheniy.

\nomenclature{%
    \( \begin{rcases}
        a_n \\
        b_n
    \end{rcases} \)%
}{koeffitsienty razlozheniya Mi v dalnem pole sootvetstvuyushchie elektricheskim i
    magnitnym multipolyam}
\nomenclature[a\( e \)]{\( {\boldsymbol{\hat{\mathrm e}}} \)}{edinichnyy vektor}
\nomenclature{\( E_0 \)}{amplituda padayushchego polya}
\nomenclature{\( j \)}{tip funktsii Besselya}
\nomenclature{\( k \)}{volnovoy vektor padayushchey volny}
\nomenclature{%
    \( \begin{rcases}
        a_n \\
        b_n
    \end{rcases} \)%
}{i snova koeffitsienty razlozheniya Mi v dalnem pole sootvetstvuyushchie
    elektricheskim i magnitnym multipolyam. Dobavleno mnogo teksta, tak chto
    opisanie gruppy uslovnykh oboznacheniy znachitelno prevysilo vysotu etoy
    gruppy...}
\nomenclature{\( L \)}{obshchee chislo sloev}
\nomenclature{\( l \)}{nomer sloya vnutri stratifitsirovannoy sfery}
\nomenclature{\( \lambda \)}{dlina volny elektromagnitnogo izlucheniya v vakuume}
\nomenclature{\( n \)}{poryadok multipolya}
\nomenclature{%
    \( \begin{rcases}
        {\mathbf{N}}_{e1n}^{(j)} & {\mathbf{N}}_{o1n}^{(j)} \\
        {\mathbf{M}_{o1n}^{(j)}} & {\mathbf{M}_{e1n}^{(j)}}
    \end{rcases} \)%
}{sfericheskie vektornye garmoniki}
\nomenclature{\( \mu \)}{magnitnaya pronitsaemost v vakuume}
\nomenclature{\( r, \theta, \phi \)}{polyarnye koordinaty}
\nomenclature{\( \omega \)}{chastota padayushchey volny}

S pomoshchyu \verb+nomenclature+ mozhno vklyuchat v~spisok sokrashcheniya,
ne~ispolzuya ikh~v~tekste.
% zapis sokrashcheniya v spisok proiskhodit komandoy \nomenclature,
% a ne upotrebleniem samogo sokrashcheniya
\nomenclature{FEM}{finite element method, metod konechnykh elementov}
\nomenclature{FIT}{finite integration technique, metod konechnykh integralov}
\nomenclature{FMM}{fast multipole method, bystryy metod mnogopolyusnika}
\nomenclature{FVTD}{finite volume time-domain, metod konechnykh obemov
    vo~vremennoy oblasti}
\nomenclature{MLFMA}{multilevel fast multipole algorithm, mnogourovnevyy
    bystryy algoritm mnogopolyusnika}
\nomenclature{BEM}{boundary element method, metod granichnykh elementov}
\nomenclature{CST MWS}{Computer Simulation Technology Microwave Studio
    programma dlya kompyuternogo modelirovaniya uravnen Maksvella}
\nomenclature{DDA}{discrete dipole approximation, priblizhenie diskretinykh
    dipoley}
\nomenclature{FDFD}{finite difference frequency domain, metod konechnykh
    raznostey v~chastotnoy oblasti}
\nomenclature{FDTD}{finite difference time domain, metod konechnykh raznostey
    vo~vremennoy oblasti}
\nomenclature{MoM}{method of moments, metod momentov}
\nomenclature{MSTM}{multiple sphere T-Matrix, metod T-matrits dlya mnozhestva
    sfer}
\nomenclature{PSTD}{pseudospectral time domain method, psevdospektralnyy metod
    vo~vremennoy oblasti}
\nomenclature{TLM}{transmission line matrix method, metod matrits liniy peredach}

\FloatBarrier
