\chapter{Dlinnoe nazvanie glavy, v kotoroy my smotrim na~primery togo, kak budut verstatsya izobrazheniya i~spiski}\label{ch:ch2}

\section{Odinochnoe izobrazhenie}\label{sec:ch2/sec1}

\begin{figure}[ht]
    \centerfloat{
        \includegraphics[scale=0.27]{latex}
    }
    \caption{TeX.}\label{fig:latex}
\end{figure}

Dlya vyravnivaniya izobrazheniya po-tsentru ispolzuetsya komanda \verb+\centerfloat+, kotoraya yavlyaetsya vo
mnogom uluchshennoy versiey vstroennoy komandy \verb+\centering+.

\section{Dlinnoe nazvanie paragrafa, v kotorom my uznaem kak sdelat dve kartinki s~obshchim nomerom i nazvaniem}\label{sec:ch2/sect2}

A eto dve kartinki pod obshchim nomerom i nazvaniem:
\begin{figure}[ht]
    \begin{minipage}[b][][b]{0.49\linewidth}\centering
        \includegraphics[width=0.5\linewidth]{knuth1} \\ a)
    \end{minipage}
    \hfill
    \begin{minipage}[b][][b]{0.49\linewidth}\centering
        \includegraphics[width=0.5\linewidth]{knuth2} \\ b)
    \end{minipage}
    \caption{Ochen dlinnaya podpis k izobrazheniyu,
        na kotorom predstavleny dve fotografii Donalda Knuta}
    \label{fig:knuth}
\end{figure}

Te~zhe~dve kartinki pod~obshchim nomerom i~nazvaniem,
no s avtomatizirovannoy numeratsiey podrisunkov:
\begin{figure}[ht]
    \centerfloat{
        \hfill
        \subcaptionbox[List-of-Figures entry]{Pervyy podrisunok\label{fig:knuth_2-1}}{%
            \includegraphics[width=0.25\linewidth]{knuth1}}
        \hfill
        \subcaptionbox{\label{fig:knuth_2-2}}{%
            \includegraphics[width=0.25\linewidth]{knuth2}}
        \hfill
        \subcaptionbox{Tretiy podrisunok, podpis k kotoromu
            ne~pomeshchaetsya na~odnoy stroke}{%
            \includegraphics[width=0.3\linewidth]{example-image-c}}
        \hfill
    }
    \legend{Podrisunochnyy tekst, opisyvayushchiy oboznacheniya, naprimer. Soglasno
        GOST 2.105, punkt 4.3.1, raspolagaetsya pered naimenovaniem risunka.}
    \caption[Etot tekst popadaet v nazvaniya risunkov v spiske risunkov]{Ochen
        dlinnaya podpis k vtoromu izobrazheniyu, na~kotorom predstavleny dve
        fotografii Donalda Knuta}\label{fig:knuth_2}
\end{figure}

Na risunke~\cref{fig:knuth_2-1} pokazan Donald Knut bez golovnogo ubora.
Na risunke~\cref{fig:knuth_2}\subcaptionref*{fig:knuth_2-2}
pokazan Donald Knut v golovnom ubore.

\section{Vektornaya grafika}\label{sec:ch2/vector}

Vozmozhno vstavlyat vektornye kartinki, rasschityvaemye \LaTeX\ <<na~letu>>
s~ikh~predvaritelnoy kompilyatsiey. Nadpisi v takikh risunkakh budut vypolneny
tem zhe~shriftom, kotoryy ukazan dlya dokumenta v tselom.
Na~risunke~\cref{fig:tikz_example} na~stranitse~\pageref{fig:tikz_example}
predstavlen primer skhemy, rasschityvaemoy paketom \verb|tikz| <<na~letu>>.
Dlya uskoreniya kompilyatsii, podobnye risunki mogut byt <<keshirovany>>, chto
opredelyaetsya nastroykami v~\verb|common/setup.tex|.
Prichem imya predkompilirovannogo
fayla i~papka raspolozheniya takikh faylov mogut byt otdelno zadany,
chto udobno, esli ne~dlya podgotovki dissertatsii,
to~dlya podgotovki nauchnykh publikatsiy.
\begin{figure}[ht]
    \centerfloat{
        \ifdefmacro{\tikzsetnextfilename}{\tikzsetnextfilename{tikz_example_compiled}}{}% prisvaivaemoe predkompilirovannomu pdf imya fayla (ne obyazatelno)
        \input{Dissertation/images/tikz_scheme.tikz}

    }
    \legend{}
    \caption[Primer \texttt{tikz} skhemy]{Primer risunka, rasschityvaemogo
        \texttt{tikz}, kotoryy mozhet byt predkompilirovan}\label{fig:tikz_example}
\end{figure}

Mnozhestvo programm imeyut libo vstroennuyu vozmozhnost eksportirovat vektornuyu
grafiku kodom \verb|tikz|, libo sootvetstvuyushchiy paket rasshireniya.
Naprimer, v GeoGebra est vstroennyy eksport,
dlya Inkscape est paket svg2tikz,
dlya Python est paket tikzplotlib,
dlya R est paket tikzdevice.

\begin{figure}[htbp]
    \centerfloat{
        \ifdefmacro{\tikzsetnextfilename}{\tikzsetnextfilename{pic2}}{}%
        \input{Dissertation/images/scheme.tikz}
    }
    \legend{%
        \textbf{1} "--- kruzhok s zagogulinoy;
        \textbf{2} "--- kamertony;
        \textbf{3} "--- kresty;
        \textbf{4} "--- volny;
        \textbf{5} "--- pryamougolniki;
        \textbf{5} "--- pronzennyy streloy pryamougolnik.%
    }
    \caption{Sostavnaya skhema \textit{tikz}}\label{fig:scheme-tikz}
\end{figure}

Na risunke~\cref{fig:scheme-tikz} predstavlena sostavnaya skhema \textit{tikz}.
Kazhdyy ee element narisovan v otdelnom fayle v edinichnom masshtabe.
Rasstanovka elementov na~risunke proizvoditsya pri pomoshchi argumentov \texttt{xshift},
\texttt{yshift}, \texttt{rotate} i~\texttt{scale} okruzheniya \texttt{scope}.

Primer ispolzovaniya biblioteki \textit{circuitikz} izobrazhen na risunke~\cref{fig:circuitikz}.

\begin{figure}[htbp]
    \centerfloat{
        \input{Dissertation/images/circuit.tikz}
    }
    \caption{Skhema \textit{circuitikz}}\label{fig:circuitikz}
\end{figure}

Krasivye grafiki takzhe mozhno dobavlyat pri pomoshchi paketa \textit{pgfplot}~(risunok~\cref{fig:pgfplot}).
Zamechatelnoy osobennostyu etogo sposoba yavlyaetsya sootvetstvie shriftov na grafike obshchemu
stilyu dokumenta.

\begin{figure}[htbp]
    \centerfloat{
        \input{Dissertation/images/plot_csv.tikz}
    }
    \caption{Grafik \textit{pgfplot} na osnove dannykh iz \texttt{csv} fayla}\label{fig:pgfplot}
\end{figure}


\section{Primer verstki spiskov}\label{sec:ch2/sec3}

\noindent Numerovannyy spisok:
\begin{enumerate}
    \item Pervyy punkt.
    \item Vtoroy punkt.
    \item Tretiy punkt.
\end{enumerate}

\noindent Markirovannyy spisok:
\begin{itemize}
    \item Pervyy punkt.
    \item Vtoroy punkt.
    \item Tretiy punkt.
\end{itemize}

\noindent Vlozhennye spiski:
\begin{itemize}
    \item Imeetsya markirovannyy spisok.
          \begin{enumerate}
              \item V nem lezhit numerovannyy spisok,
              \item v kotorom
                    \begin{itemize}
                        \item lezhit eshche odin markirovannyy spisok.
                    \end{itemize}
          \end{enumerate}
\end{itemize}

\noindent Numerovannye vlozhennye spiski:
\begin{enumerate}
    \item Pervyy punkt.
    \item Vtoroy punkt.
    \item Voobshche, po GOST 2.105 pervyy uroven numeratsii
          (pri neobkhodimosti ssylki v tekste dokumenta na odno iz perechisleniy)
          idet bukvami russkogo ili latinskogo alfavitov,
          a vtoroy "--- tsiframi so~skobkami.
          Zdes otkhodim ot GOST.
          \begin{enumerate}
              \item v nem lezhit numerovannyy spisok,
              \item v kotorom
                    \begin{enumerate}
                        \item eshche odin numerovannyy spisok,
                        \item tretiy uroven numeratsii ne normirovan GOST 2.105;
                        \item obrashchaem vnimanie na strochnost bukv,
                        \item v etom spiske
                              \begin{itemize}
                                  \item lezhit eshche odin markirovannyy spisok.
                              \end{itemize}
                    \end{enumerate}

          \end{enumerate}

    \item Chetvertyy punkt.
\end{enumerate}

\section{Traditsii russkogo nabora}

Mnogo poleznykh sovetov privedeno v materiale
<<\href{https://kostyrka.ru/main/ru/typesetting-and-typography-crash-course-by-kostyrka/}{Kratkiy kurs blagorodnogo nabora}>>
(avtor A.\:V.~Kostyrka).
Dalee my kosnemsya lish nekotorykh naibolee rasprostranennykh osobennostey.

\subsection{Probely}

V~russkom nabore prinyato:
\begin{itemize}
    \item edinitsy izmereniya, znak protsenta otdelyat probelami ot~chisla:
          10~kVt, 15~\% (soglasno GOST 8.417, razdel 8);
    \item \(\tg 20\text{\textdegree}\), no: 20~{\textdegree}C
          (soglasno GOST 8.417, razdel 8);
    \item znak nomera, paragrafa otdelyat ot~chisla: №~5, \S~8;
    \item standartnye sokrashcheniya: t.\:e., i~t.\:d., i~t.\:p.;
    \item nerazryvnye probely v~predlozheniyakh.
\end{itemize}

\subsection{Matematicheskie znaki i simvoly}

Russkaya traditsiya nachertaniya grecheskikh bukv i nekotorykh matematicheskikh
funktsiy otlichaetsya ot~zapadnoy. Eto ispravlyaetsya seriey
\verb|\renewcommand|.
\begin{itemize}
    %Vse \original... komandy zaranee, radi etogo primera, opredeleny v Dissertation\userstyles.tex
    \item[Do:] \( \originalepsilon \originalge \originalphi\),
          \(\originalphi \originalleq \originalepsilon\),
          \(\originalkappa \in \originalemptyset\),
          \(\originaltan\),
          \(\originalcot\),
          \(\originalcsc\).
    \item[Posle:] \( \epsilon \ge \phi\),
          \(\phi \leq \epsilon\),
          \(\kappa \in \emptyset\),
          \(\tan\),
          \(\cot\),
          \(\csc\).
\end{itemize}

Krome togo, prinyato nabirat grecheskie bukvy vertikalnymi, chto
reshaetsya podklyucheniem paketa \verb|upgreek| (sm. zakommentirovannyy
blok v~\verb|userpackages.tex|) i~analogichnym pereopredeleniem v
preambule (sm.~zakommentirovannyy blok v~\verb|userstyles.tex|). V
etom shablone takie pereopredeleniya uzhe vklyucheny.

Znaki matematicheskikh operatsiy prinyato perenosit. Primer perenosa
v~formule~\eqref{eq:equation3}.

\subsection{Kavychki}
V angliyskom yazyke prinyaty odinarnye i dvoynye kavychki v~vide ‘...’ i~“...”.
V~Rossii prinyaty frantsuzskie («...») i~nemetskie („...“) kavychki (oni nazyvayutsya
«elochki» i~«lapki», sootvetstvenno). ,,Lapki`` obychno ispolzuyutsya vnutri
<<elochek>>, naprimer, <<... nash gordyy ,,Varyag``...>>.

Frantsuzkie levye i pravye kavychki nabirayutsya
kak ligatury \verb|<<| i~\verb|>>|, a~nemetskie levye
i pravye kavychki nabirayutsya kak ligatury \verb|,,| i~\verb|‘‘| (\verb|``|).

Vmesto ligatur ili komand s~aktivnym simvolom "\ mozhno ispolzovat komandy
\verb|\glqq| i \verb|\grqq| dlya nabora nemetskikh kavychek i komandy \verb|\flqq|
i~\verb|\frqq| dlya nabora frantsuzskikh kavychek. Oni opredeleny v pakete
\verb|babel|.

\subsection{Tire}
%  babel+pdflatex po umolchaniyu, v polyglossia nado vklyuchat optsiey (i perekompilirovat s udaleniem vremennykh faylov)
Komanda \verb|"---| ispolzuetsya dlya pechati tire v tekste. Ono mozhet byt
neskolko koroche angliyskogo dlinnogo tire (podrobnosti v~dokumentatsii
rusifikatsii babel). Krome togo, komanda zadaet nebolshuyu zhestkuyu otbivku
ot~slova, stoyashchego pered tire. Pri etom, samo tire ne~otryvaetsya ot~slova.
Posle tire sleduet takaya zhe otbivka ot teksta, kak i~pered tire. Pri nabore
teksta mezhdu slovom i komandoy, za kotorym ona sleduet, dolzhen stoyat probel.

V sostavnykh slovakh, takikh, kak <<Zakon Mendeleeva"--~Klapeyrona>>, dlya pechati
tire nado ispolzovat komandu \verb|"--~|. Ona stavit bolee korotkoe,
po~sravneniyu s~angliyskim, tire i pozvolyaet delat perenosy vo vtorom slove.
Pri~nabore teksta komanda \verb|"--~| ne otdelyaetsya probelom ot slova,
za~kotorym ona sleduet (\verb|Mendeleeva"--~|). Sleduyushchee za komandoy slovo
mozhet byt  otdeleno ot~nee probelom ili pereneseno na druguyu stroku.

Esli pryamaya rech nachinaetsya s~abzatsa, to pered nachalom ee pechataetsya tire
komandoy \verb|"--*|. Ona pechataet russkoe tire i zhestkuyu otbivku nuzhnoy
velichiny pered tekstom.

\subsection{Defisy i perenosy slov}
%  babel+pdflatex po umolchaniyu, v polyglossia nado vklyuchat optsiey (i perekompilirovat s udaleniem vremennykh faylov)
Dlya pechati defisa v~sostavnykh slovakh vvedeny dve komandy. Komanda~\verb|"~|
pechataet defis i~zapreshchaet delat perenosy v~samikh slovakh, a~komanda \verb|"=|
pechataet defis, ostavlyaya \TeX ’u pravo delat perenosy v~samikh slovakh.

V otlichie ot komandy \verb|\-|, komanda \verb|"-| zadaet mesto v~slove, gde
mozhno delat perenos, ne~zapreshchaya perenosy i~v~drugikh mestakh slova.

Komanda \verb|""| zadaet mesto v~slove, gde mozhno delat perenos, prichem defis
pri~perenose v~etom meste ne~stavitsya.

Komanda \verb|",| vstavlyaet nebolshoy probel posle initsialov s~pravom perenosa
v~familii.

\section{Tekst iz pangramm i formul}

Lyubya, sesh shchiptsy, "--- vzdokhnet mer, "--- kayf zhguch. Shef vzyaren tchk shchiptsy
s~ekhom gudbay Zhyul. Ey, zhlob! Gde tuz? Pryach yunykh semshchits v~shkaf. Eks-graf?
Plyush izyat. Bem chuzhdyy tsen khvoshch! Ekh, chuzhak! Obshchiy sem tsen shlyap (yuft) "---
vdryzg! Lyubya, sesh shchiptsy, "--- vzdokhnet mer, "--- kayf zhguch. Shef vzyaren tchk
shchiptsy s~ekhom gudbay Zhyul. Ey, zhlob! Gde tuz? Pryach yunykh semshchits v~shkaf.
Eks-graf? Plyush izyat. Bem chuzhdyy tsen khvoshch! Ekh, chuzhak! Obshchiy sem tsen shlyap
(yuft) "--- vdryzg! Lyubya, sesh shchiptsy, "--- vzdokhnet mer, "--- kayf zhguch. Shef
vzyaren tchk shchiptsy s~ekhom gudbay Zhyul. Ey, zhlob! Gde tuz? Pryach yunykh semshchits
v~shkaf. Eks-graf? Plyush izyat. Bem chuzhdyy tsen khvoshch! Ekh, chuzhak! Obshchiy sem tsen
shlyap (yuft) "--- vdryzg! Lyubya, sesh shchiptsy, "--- vzdokhnet mer, "--- kayf zhguch.
Shef vzyaren tchk shchiptsy s~ekhom gudbay Zhyul. Ey, zhlob! Gde tuz? Pryach yunykh semshchits
v~shkaf. Eks-graf? Plyush izyat. Bem chuzhdyy tsen khvoshch! Ekh, chuzhak! Obshchiy sem tsen
shlyap (yuft) "--- vdryzg! Lyubya, sesh shchiptsy, "--- vzdokhnet mer, "--- kayf zhguch.
Shef vzyaren tchk shchiptsy s~ekhom gudbay Zhyul. Ey, zhlob! Gde tuz? Pryach yunykh semshchits
v~shkaf. Eks-graf? Plyush izyat. Bem chuzhdyy tsen khvoshch! Ekh, chuzhak! Obshchiy sem tsen
shlyap (yuft) "--- vdryzg! Lyubya, sesh shchiptsy, "--- vzdokhnet mer, "--- kayf zhguch.
Shef vzyaren tchk shchiptsy s~ekhom gudbay Zhyul. Ey, zhlob! Gde tuz? Pryach yunykh semshchits
v~shkaf. Eks-graf? Plyush izyat. Bem chuzhdyy tsen khvoshch! Ekh, chuzhak! Obshchiy sem tsen
shlyap (yuft) "--- vdryzg! Lyubya, sesh shchiptsy, "--- vzdokhnet mer, "--- kayf zhguch.
Shef vzyaren tchk shchiptsy s~ekhom gudbay Zhyul. Ey, zhlob! Gde tuz? Pryach yunykh semshchits
v~shkaf. Eks-graf? Plyush izyat. Bem chuzhdyy tsen khvoshch! Ekh, chuzhak! Obshchiy sem tsen
shlyap (yuft) "--- vdryzg! Lyubya, sesh shchiptsy, "--- vzdokhnet mer, "--- kayf zhguch.
Shef vzyaren tchk shchiptsy s~ekhom gudbay Zhyul. Ey, zhlob! Gde tuz? Pryach yunykh semshchits
v~shkaf. Eks-graf? Plyush izyat. Bem chuzhdyy tsen khvoshch! Ekh, chuzhak! Obshchiy sem tsen
shlyap (yuft) "--- vdryzg! Lyubya, sesh shchiptsy, "--- vzdokhnet mer, "--- kayf zhguch.
Shef vzyaren tchk shchiptsy s~ekhom gudbay Zhyul. Ey, zhlob! Gde tuz? Pryach yunykh semshchits
v~shkaf. Eks-graf? Plyush izyat. Bem chuzhdyy tsen khvoshch! Ekh, chuzhak! Obshchiy sem tsen
shlyap (yuft) "--- vdryzg! Lyubya, sesh shchiptsy, "--- vzdokhnet mer, "--- kayf zhguch.
Shef vzyaren tchk shchiptsy s~ekhom gudbay Zhyul. Ey, zhlob! Gde tuz? Pryach yunykh semshchits
v~shkaf. Eks-graf? Plyush izyat. Bem chuzhdyy tsen khvoshch! Ekh, chuzhak! Obshchiy sem tsen
shlyap (yuft) "--- vdryzg! Lyubya, sesh shchiptsy, "--- vzdokhnet mer, "--- kayf zhguch.
Shef vzyaren tchk shchiptsy s~ekhom gudbay Zhyul. Ey, zhlob! Gde tuz? Pryach yunykh semshchits
v~shkaf. Eks-graf? Plyush izyat. Bem chuzhdyy tsen khvoshch! Ekh, chuzhak! Obshchiy sem tsen
shlyap (yuft) "--- vdryzg!Lyubya, sesh shchiptsy, "--- vzdokhnet mer, "--- kayf zhguch.
Shef vzyaren tchk shchiptsy s~ekhom gudbay Zhyul. Ey, zhlob! Gde tuz? Pryach yunykh semshchits
v~shkaf. Eks-graf? Plyush izyat. Bem chuzhdyy tsen khvoshch! Ekh, chuzhak! Obshchiy sem tsen

Ku kkhoro adolezhkens voluptaria khazh, vim graeko ykchpetynda ty. Graeky zhemper
lyukyaliyuch kvuy ku, aekvyuy prodyzhshchet khazh ne. Vim ku magna pyrikula, no kvyuando
pozhydoneyum pro. Kvuy at rykvyuy enermyshch. Vyro akkuzata vim ne.
\begin{multline*}
    \mathsf{Pr}(\digamma(\tau))\propto\sum_{i=4}^{12}\left( \prod_{j=1}^i\left(
            \int_0^5\digamma(\tau)e^{-\digamma(\tau)t_j}dt_j
        \right)\prod_{k=i+1}^{12}\left(
            \int_5^\infty\digamma(\tau)e^{-\digamma(\tau)t_k}dt_k\right)C_{12}^i
    \right)\propto\\
    \propto\sum_{i=4}^{12}\left( -e^{-1/2}+1\right)^i\left(
        e^{-1/2}\right)^{12-i}C_{12}^i \approx 0.7605,\quad
    \forall\tau\neq\overline{\tau}
\end{multline*}
Kvuy yeyuz omniyum yn. Ekz alekvyuam konchyulatu kvuy, ty alyakvyuam envidyunt per.
Zyd ne kommodo probatuzh. Zhyat doktyuzh dyzhpyutando ut, ku zalutande yurbanytazh
dezsenteash zhyat, vim zhyumo dolorezh rationebyuzh ea.

Ad entegry korpora zhplendide khazh. Ezht at fakete dycherunt perzhykyuti. Ne nam
doming percheus. Ku kvyuo euzhto errem zyuchkepit. Pro khabeo albyukiyus ne.
\[
    \begin{pmatrix}
        a_{11} & a_{12} & a_{13} \\
        a_{21} & a_{22} & a_{23}
    \end{pmatrix}
\]

\[
    \begin{vmatrix}
        a_{11} & a_{12} & a_{13} \\
        a_{21} & a_{22} & a_{23}
    \end{vmatrix}
\]

\[
    \begin{bmatrix}
        a_{11} & a_{12} & a_{13} \\
        a_{21} & a_{22} & a_{23}
    \end{bmatrix}
\]
Pro ea graeki kvyuaykvue dyzhpyutando. Yt vel tebikvyue defyanyatyonys, nam zholyum
kvyuando mandamyuch ea. Eozh paulo laudym inkedyrint ne, perpetyua forynchybyuzh per
eyu. Modyratiyuz dytyrryuizshchet duo ad, viryz feugyaat dytrakzhyt nyk ed, duo alie
kayuchae lygendoch no. Ea molliz yurbanytazh zigneferumkvyuy ezht.

Pro mandamyuch konchetytyur ed. Tretane prenkipyz zigneferumkvyuy vyash an. At khez
ekvyuedym shchuavyatate. Aleenyum zentyntiae ad pro, ea yuchyu myunyre graeki demokritum,
ku pro chent voluptaria. Ylit dykory alyakvyuid eyuzh yt. Ku rybyum myundy yutenam
duo.
\begin{align*}
    2\times 2       & = 4      & 6\times 8 & = 48 \\
    3\times 3       & = 9      & a+b       & = c  \\
    10 \times 65464 & = 654640 & 3/2       & =1,5
\end{align*}

\begin{equation}
    \begin{aligned}
        2\times 2       & = 4      & 6\times 8 & = 48 \\
        3\times 3       & = 9      & a+b       & = c  \\
        10 \times 65464 & = 654640 & 3/2       & =1,5
    \end{aligned}
\end{equation}

Per yn tale pozhtea, mya ed popyulo debetiz zhkribentur. Yn kvuy appetyre
menandrya, zyd alyakvyuid khabymuch korpora yn. Omniyum perkepityur shea eyu, shea
appetyre akkuzata reformydanch yt, ty yrror vertyuty nyumkvuam \(10 \times 65464 =
654640\quad  3/2=1,5\) meya. Ipzum euezhmod \(a+b = c\) malyuizchyt ad duo. Ad
feyugayat pytynteyum advyrzharyayum vyash. Modo erepyuyat detrakto ty nyk, eyuzh mentetyum
pyrikula appellantyur ea.

Mel ty delynete takematysh. Zentyntiae konklyuzhionemkvue an meya. Vezhi lebyr
kvyuaykvue kvuy ne, duo zymyul delikata ku. Yam ku alie putynt.

%Bolshaya figurnaya skobka tolko sprava
\[\left. %VAZhNO: tochka posle slova left delaet skobku neotobrazhaemoy
    \begin{aligned}
        2 \times x      & = 4 \\
        3 \times y      & = 9 \\
        10 \times 65464 & = z
    \end{aligned}\right\}
\]


Konvynery vityupyrata no nam, tebikvyue mentetyum poztyulant ed pro. Duo ea laudym
kopiozhay, nyk movet veniam leberavichsy eyu, nam epikyure detrakto rykyuchabo yt.
Verytyuzh akkyuzhamyuz ty shea, debetiz forynchybyuzh zhkryapsherit yt pre. An eyuzh tympor
ryferrentur, yuchyu dolor kotediekvyue yn. Zyd ipzum dytrakzhyt nyglegentur ne,
partym ykzhplikari dezhzhentiyunt ad per. Mel ty kyterozh molyzhtyay, nam no yrror
zhkripta appareat.

\[ \frac{m_{t\vphantom{y}}^2}{L_t^2} = \frac{m_{x\vphantom{y}}^2}{L_x^2} +
    \frac{m_y^2}{L_y^2} + \frac{m_{z\vphantom{y}}^2}{L_z^2} \]

Vere laborezh tebikvyue khazh ut. An paulo torkvyuatoz khazh, ne probo feugyaat
takematysh shea. Meleuz pertinakea yullamkorper pre ad, no mya rykvyuy konkyptam.
Khez kvyuot pertinakea ei, ellyud traktatoz per ad. Zyd ed anemal laborezh
nominavi, zhyat ad konguy labyatyur. Labore tamkvyuam vekzh yn, per ne deko diam
shaperet, ekz vyash tebikvyue eleefend mediokretatym.

Ne pro natyum fyuyzchyt kvyualizkvyue, aekvyuy zhkayvola mel ku. Ad graekyzh
platonem advyrzharyayum kvuy, vim empydit kommyuny at, at shea odeo kvyuayrendum.
Vertyuty azhzhyntior effikeendi eozh ne, doming laboramyuz ei yam. Chenzeret
mnyzharkkhyum ekz eozh, ylit tamkvyuam fakilizizh nyk ei. Kvuy an elyktram
tinkidyunt entyrprytaryash. Yn yanvynyary traktatoz zentyntiae zyd. Dyuizh zalyutatuzh
yam no, pro yt anemal mnyzharkkhyum, ei yyum ponderyum mayyzhtatyzh.

\FloatBarrier
