\chapter{Verstka tablits}\label{ch:ch3}

\section{Tablitsa obyknovennaya}\label{sec:ch3/sect1}

Tak razmeshchaetsya tablitsa:

\begin{table} [htbp]
    \centering
    \begin{threeparttable}% vyravnivanie podpisi po granitsam tablitsy
        \caption{Nazvanie tablitsy}\label{tab:Ts0Sib}%
        \begin{tabular}{| p{3cm} || p{3cm} | p{3cm} | p{4cm}l |}
            \hline
            \hline
            Mesyats   & \centering \(T_{min}\), K & \centering \(T_{max}\), K & \centering  \((T_{max} - T_{min})\), K & \\
            \hline
            Dekabr & \centering  253.575       & \centering  257.778       & \centering      4.203                  & \\
            Yanvar  & \centering  262.431       & \centering  263.214       & \centering      0.783                  & \\
            Fevral & \centering  261.184       & \centering  260.381       & \centering     \(-\)0.803              & \\
            \hline
            \hline
        \end{tabular}
    \end{threeparttable}
\end{table}

\begin{table} [htbp]% Primer zapisi tablitsy s nomerom, no bez otobrazhaemogo naimenovaniya
    \centering
    \begin{threeparttable}% vyravnivanie podpisi po granitsam tablitsy
        \caption{}%
        \label{tab:test1}%
        \begin{SingleSpace}
            \begin{tabular}{| c | c | c | c |}
                \hline
                Okonnaya funktsiya & \({2N}\) & \({4N}\) & \({8N}\) \\ \hline
                Pryamougolnoe   & 8.72     & 8.77     & 8.77     \\ \hline
                Khanna           & 7.96     & 7.93     & 7.93     \\ \hline
                Khemminga        & 8.72     & 8.77     & 8.77     \\ \hline
                Blekmana        & 8.72     & 8.77     & 8.77     \\ \hline
            \end{tabular}%
        \end{SingleSpace}
    \end{threeparttable}
\end{table}

Tablitsa~\cref{tab:test2} "--- primer tablitsy, oformlennoy v~klassicheskom knizhnom
variante ili~ochen blizko k~nemu. \mbox{GOSTu} po~suti ne~protivorechit. Mozhno
eshche~uluchshit predstavlenie, s~pomoshchyu paketa \verb|siunitx| ili~podobnogo.

\begin{table} [htbp]%
    \centering
    \caption{Naimenovanie tablitsy, ochen dlinnoe naimenovanie tablitsy, chtoby posmotret kak ono budet raspolagatsya na~neskolkikh strokakh i~perenositsya}%
    \label{tab:test2}% label vsegda zhelatelno idti posle caption
    \renewcommand{\arraystretch}{1.5}%% Uvelichenie rasstoyaniya mezhdu ryadami, dlya uluchsheniya vospriyatiya.
    \begin{SingleSpace}
        \begin{tabular}{@{}@{\extracolsep{20pt}}llll@{}} %Vertikalnye polosy ne ispolzuyutsya printsipialno, kak i lishnie gorizontalnye (dopuskaetsya po GOST 2.105 punkt 4.4.5) % @{} pozvolyaet prizhimatsya k krayam
            \toprule     %%% verkhnyaya lineyka
            Okonnaya funktsiya & \({2N}\) & \({4N}\) & \({8N}\) \\
            \midrule %%% tonkiy razdelitel. Otdelyaet nazvaniya stolbtsov. Obyazatelen po GOST 2.105 punkt 4.4.5
            Pryamougolnoe   & 8.72     & 8.77     & 8.77     \\
            Khanna           & 7.96     & 7.93     & 7.93     \\
            Khemminga        & 8.72     & 8.77     & 8.77     \\
            Blekmana        & 8.72     & 8.77     & 8.77     \\
            \bottomrule %%% nizhnyaya lineyka
        \end{tabular}%
    \end{SingleSpace}
\end{table}

\section{Tablitsa s mnogostrochnymi yacheykami i primechaniem}

V tablitse \cref{tab:makecell} priveden primer ispolzovaniya komandy
\verb+\multicolumn+ dlya obedineniya gorizontalnykh yacheek tablitsy,
i komand paketa \textit{makecell} dlya dobavleniya razryva stroki vnutri yacheek.
Pri formatirovanii tablitsy \cref{tab:makecell} ispolzovan stil podpisey \verb+split+.
Globalno etot stil mozhet byt vklyuchen v fayle \verb+Dissertation/setup.tex+ dlya dissertatsii i v
fayle \verb+Synopsis/setup.tex+ dlya avtoreferata.
Odnako takoe oformlenie ne~sootvetstvuet GOST.

\begin{table} [htbp]
    \captionsetup[table]{format=split}
    \centering
    \begin{threeparttable}% vyravnivanie podpisi po granitsam tablitsy
        \caption{Primer ispolzovaniya funktsiy paketa \textit{makecell}}%
        \label{tab:makecell}%
        \begin{tabular}{| c | c | c | c |}
            \hline
            Kolonka 1                      & Kolonka 2 &
            \thead{Nazvanie kolonki 3,                                                 \\
            ne pomeshchayushcheesya v odnu stroku} & Kolonka 4                                 \\
            \hline
            \multicolumn{4}{|c|}{Vyravnivanie po tsentru}                               \\
            \hline
            \multicolumn{2}{|r|}{\makecell{Vyravnivanie                                \\ k~pravomu krayu}} &
            \multicolumn{2}{l|}{Vyravnivanie k levomu krayu}                            \\
            \hline
            \makecell{V etoy yacheyke                                                    \\
            mnogo informatsii}              & 8.72      & 8.55                   & 8.44 \\
            \cline{3-4}
            A v etoy malo                  & 8.22      & \multicolumn{2}{c|}{5}        \\
            \hline
        \end{tabular}%
    \end{threeparttable}
\end{table}

Tablitsy~\cref{tab:test3,tab:test4} "--- primer realizatsii raspolozheniya
primechaniya v~sootvetstvii s GOST 2.105. Kazhdyy variant so svoimi dostoinstvami
i~nedostatkami. Variant cherez \verb|tabulary| khorosho podbiraet shirinu stolbtsov,
no~slozhno upravlyat vertikalnym vyravnivaniem, \verb|tabularx| "--- naoborot.
\begin{table}[ht]%
    \caption{Ne pro natyum fyuyzchyt kvyualizkvyue}\label{tab:test3}% label vsegda zhelatelno idti posle caption
    \begin{SingleSpace}
        \setlength\extrarowheight{6pt} %vot etim upravlyaem rasstoyaniem mezhdu ryadami, \arraystretch daet neudachnyy rezultat
        \setlength{\tymin}{1.9cm}% minimalnaya shirina stolbtsa
        \begin{tabulary}{\textwidth}{@{}>{\zz}L >{\zz}C >{\zz}C >{\zz}C >{\zz}C@{}}% Vertikalnye polosy ne ispolzuyutsya printsipialno, kak i lishnie gorizontalnye (dopuskaetsya po GOST 2.105 punkt 4.4.5) % @{} pozvolyaet prizhimatsya k krayam
            \toprule     %%% verkhnyaya lineyka
            doming laboramyuz ei yam (Obshchiy sem tsen shlyap (yuft)) & Shef vzyaren &
            advyrzharyayum &
            tebikvyue eleefend mediokretatym &
            Chenzeret mnyzharkkhyum         \\
            \midrule %%% tonkiy razdelitel. Otdelyaet nazvaniya stolbtsov. Obyazatelen po GOST 2.105 punkt 4.4.5
            Ey, zhlob! Gde tuz? Pryach yunykh semshchits v~shkaf Plyush izyat. Bem chuzhdyy tsen khvoshch! &
            \({\approx}\) &
            \({\approx}\) &
            \({\approx}\) &
            \( + \) \\
            Ekh, chuzhak! Obshchiy sem tsen &
            \( + \) &
            \( + \) &
            \( + \) &
            \( - \) \\
            Ne pro natyum fyuyzchyt kvyualizkvyue, aekvyuy zhkayvola mel ku. Ad
            graekyzh platonem advyrzharyayum kvuy, vim empydit kommyuny at, at shea
            odeo &
            \({\approx}\) &
            \( - \) &
            \( - \) &
            \( - \) \\
            Lyubya, sesh shchiptsy, "--- vzdokhnet mer, "--- kayf zhguch. &
            \( - \) &
            \( + \) &
            \( + \) &
            \({\approx}\) \\
            Ne pro natyum fyuyzchyt kvyualizkvyue, aekvyuy zhkayvola mel ku. Ad
            graekyzh platonem advyrzharyayum kvuy, vim empydit kommyuny at, at shea
            odeo kvyuayrendum. Vertyuty azhzhyntior effikeendi eozh ne. &
            \( + \) &
            \( - \) &
            \({\approx}\) &
            \( - \) \\
            \midrule%%% tonkiy razdelitel
            \multicolumn{5}{@{}p{\textwidth}@{}}{%
            \vspace*{-4ex}% etim podtyagivaem povyshe
            \hspace*{2.5em}% abzatsnyy otstup - trebovanie GOST 2.105
            Primechanie "---  Plyush izyat: <<\(+\)>> "--- advyrzharyayum kvuy, vim
            empydit; <<\(-\)>> "--- empydit kommyuny at; <<\({\approx}\)>> "---
            Shef vzyaren tchk shchiptsy s~ekhom gudbay Zhyul. Ey, zhlob! Gde tuz?
            Pryach yunykh semshchits v~shkaf. Eks-graf?
            }
            \\
            \bottomrule %%% nizhnyaya lineyka
        \end{tabulary}%
    \end{SingleSpace}
\end{table}

Esli tablitsa~\cref{tab:test3} ne pomeshchaetsya na toy zhe stranitse, vse
ee~soderzhimoe perenositsya na~sleduyushchuyu, blizhayshuyu, a~etot tekst idet pered ney.
\begin{table}[ht]%
    \caption{Lyubya, sesh shchiptsy, "--- vzdokhnet mer, "--- kayf zhguch}%
    \label{tab:test4}% label vsegda zhelatelno idti posle caption
    \renewcommand{\arraystretch}{1.6}%% Uvelichenie rasstoyaniya mezhdu ryadami, dlya uluchsheniya vospriyatiya.
    \def\tabularxcolumn#1{m{#1}}
    \begin{tabularx}{\textwidth}{@{}>{\raggedright}X>{\centering}m{1.9cm} >{\centering}m{1.9cm} >{\centering}m{1.9cm} >{\centering\arraybackslash}m{1.9cm}@{}}% Vertikalnye polosy ne ispolzuyutsya printsipialno, kak i lishnie gorizontalnye (dopuskaetsya po GOST 2.105 punkt 4.4.5) % @{} pozvolyaet prizhimatsya k krayam
        \toprule     %%% verkhnyaya lineyka
        doming laboramyuz ei yam (Obshchiy sem tsen shlyap (yuft))  & Shef vzyaren &
        advyr\-zharyayum                                         &
        tebikvyue eleefend mediokretatym                     &
        Chenze\-ret mnyzharkkhyum                                                 \\
        \midrule %%% tonkiy razdelitel. Otdelyaet nazvaniya stolbtsov. Obyazatelen po GOST 2.105 punkt 4.4.5
        Ey, zhlob! Gde tuz? Pryach yunykh semshchits v~shkaf Plyush izyat.
        Bem chuzhdyy tsen khvoshch!                                 &
        \({\approx}\)                                         &
        \({\approx}\)                                         &
        \({\approx}\)                                         &
        \( + \)                                                               \\
        Ekh, chuzhak! Obshchiy sem tsen                             &
        \( + \)                                               &
        \( + \)                                               &
        \( + \)                                               &
        \( - \)                                                               \\
        Ne pro natyum fyuyzchyt kvyualizkvyue, aekvyuy zhkayvola mel ku.
        Ad graekyzh platonem advyrzharyayum kvuy, vim empydit kommyuny at,
        at shea odeo                                           &
        \({\approx}\)                                         &
        \( - \)                                               &
        \( - \)                                               &
        \( - \)                                                               \\
        Lyubya, sesh shchiptsy, "--- vzdokhnet mer, "--- kayf zhguch. &
        \( - \)                                               &
        \( + \)                                               &
        \( + \)                                               &
        \({\approx}\)                                                         \\
        Ne pro natyum fyuyzchyt kvyualizkvyue, aekvyuy zhkayvola mel ku. Ad graekyzh
        platonem advyrzharyayum kvuy, vim empydit kommyuny at, at shea odeo
        kvyuayrendum. Vertyuty azhzhyntior effikeendi eozh ne.     &
        \( + \)                                               &
        \( - \)                                               &
        \({\approx}\)                                         &
        \( - \)                                                               \\
        \midrule%%% tonkiy razdelitel
        \multicolumn{5}{@{}p{\textwidth}@{}}{%
        \vspace*{-4ex}% etim podtyagivaem povyshe
        \hspace*{2.5em}% abzatsnyy otstup - trebovanie GOST 2.105
        Primechanie "---  Plyush izyat: <<\(+\)>> "--- advyrzharyayum kvuy, vim
        empydit; <<\(-\)>> "--- empydit kommyuny at; <<\({\approx}\)>> "--- Shef
        vzyaren tchk shchiptsy s~ekhom gudbay Zhyul. Ey, zhlob! Gde tuz? Pryach yunykh
        semshchits v~shkaf. Eks-graf?
        }
        \\
        \bottomrule %%% nizhnyaya lineyka
    \end{tabularx}%
\end{table}

\section{Tablitsy s formatirovannymi chislami}\label{sec:ch3/formatted-numbers}

V tablitsakh \cref{tab:S:parse,tab:S:align} predstavleny primery ispolzovaniya optsii
formatirovaniya chisel \texttt{S}, predostavlyaemoy paketom \texttt{siunitx}.

\begin{table}
    \centering
    \begin{threeparttable}% vyravnivanie podpisi po granitsam tablitsy
        \caption{Vyravnivanie stolbtsov}\label{tab:S:parse}
        \begin{tabular}{SS[table-parse-only]}
            \toprule
            {Vyravnivanie po razdelitelyu} & {Obychnoe vyravnivanie} \\
            \midrule
            12.345                        & 12.345                 \\
            6,78                          & 6,78                   \\
            -88.8(9)                      & -88.8(9)               \\
            4.5e3                         & 4.5e3                  \\
            \bottomrule
        \end{tabular}
    \end{threeparttable}
\end{table}

\begin{table}
    \centering
    \begin{threeparttable}% vyravnivanie podpisi po granitsam tablitsy
        \caption{Vyravnivanie s ispolzovaniem optsii \texttt{S}}\label{tab:S:align}
        \sisetup{
            table-figures-integer = 2,
            table-figures-decimal = 4
        }
        \begin{tabular}
            {SS[table-number-alignment = center]S[table-number-alignment = left]S[table-number-alignment = right]}
            \toprule
            {Kolonka 1} & {Kolonka 2} & {Kolonka 3} & {Kolonka 4} \\
            \midrule
            2.3456      & 2.3456      & 2.3456      & 2.3456      \\
            34.2345     & 34.2345     & 34.2345     & 34.2345     \\
            56.7835     & 56.7835     & 56.7835     & 56.7835     \\
            90.473      & 90.473      & 90.473      & 90.473      \\
            \bottomrule
        \end{tabular}
    \end{threeparttable}
\end{table}

\section{Paragraf \texorpdfstring{\cyrdash{}}{---} dva}\label{sec:ch3/sect2}
% Ne vse (xe|lua)latex sovmestimye shrifty umeyut rabotat s russkim tire "---

Nekotoryy tekst.

\section{Paragraf s podparagrafami}\label{sec:ch3/sect3}

\subsection{Podparagraf \texorpdfstring{\cyrdash{}}{---} odin}\label{subsec:ch3/sect3/sub1}

Nekotoryy tekst.

\subsection{Podparagraf \texorpdfstring{\cyrdash{}}{---} dva}\label{subsec:ch3/sect3/sub2}

Nekotoryy tekst.

\clearpage
