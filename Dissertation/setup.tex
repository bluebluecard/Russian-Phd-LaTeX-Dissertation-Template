%%%%%%%%%%%%%%%%%%%%%%%%%%%%%%%%%%%%%%%%%%%%%%%%%%%%%%
%%%% Fayl uproshchennykh nastroek shablona dissertatsii %%%%
%%%%%%%%%%%%%%%%%%%%%%%%%%%%%%%%%%%%%%%%%%%%%%%%%%%%%%

%%% Initsializirovanie peremennykh, ne trogat!  %%%
\newcounter{intvl}
\newcounter{otstup}
\newcounter{contnumeq}
\newcounter{contnumfig}
\newcounter{contnumtab}
\newcounter{pgnum}
\newcounter{chapstyle}
\newcounter{headingdelim}
\newcounter{headingalign}
\newcounter{headingsize}
%%%%%%%%%%%%%%%%%%%%%%%%%%%%%%%%%%%%%%%%%%%%%%%%%%%%%%

%%% Oblast uproshchennogo upravleniya oformleniem %%%

%% Interval mezhdu zagolovkami i mezhdu zagolovkom i tekstom %%
% Zagolovki otdelyayut ot teksta sverkhu i snizu
% tremya intervalami (GOST R 7.0.11-2011, 5.3.5)
\setcounter{intvl}{3}               % Koeffitsient kratnosti k razmeru shrifta

%% Otstupy u zagolovkov v tekste %%
\setcounter{otstup}{0}              % 0 --- bez otstupa; 1 --- abzatsnyy otstup

%% Numeratsiya formul, tablits i risunkov %%
% Numeratsiya formul
\setcounter{contnumeq}{0}   % 0 --- porazdelno (vo vvedenii podryad,
                            %       bez nomera razdela);
                            % 1 --- skvoznaya numeratsiya po vsey dissertatsii
% Numeratsiya risunkov
\setcounter{contnumfig}{0}  % 0 --- porazdelno (vo vvedenii podryad,
                            %       bez nomera razdela);
                            % 1 --- skvoznaya numeratsiya po vsey dissertatsii
% Numeratsiya tablits
\setcounter{contnumtab}{1}  % 0 --- porazdelno (vo vvedenii podryad,
                            %       bez nomera razdela);
                            % 1 --- skvoznaya numeratsiya po vsey dissertatsii

%% Oglavlenie %%
\setcounter{pgnum}{1}       % 0 --- nomera stranits nikak ne oboznacheny;
                            % 1 --- Str. nad nomerami stranits (dvazhdy
                            %       kompilirovat posle izmeneniya nastroyki)
\settocdepth{subsection}    % do kakogo urovnya podrazdelov vynosit v oglavlenie
\setsecnumdepth{subsection} % do kakogo urovnya numerovat podrazdely


%% Tekst i formatirovanie zagolovkov %%
\setcounter{chapstyle}{1}     % 0 --- razdely tolko pod nomerom;
                              % 1 --- razdely s nazvaniem "Glava" pered nomerom
\setcounter{headingdelim}{1}  % 0 --- nomer otdelen propuskom v 1em ili \quad;
                              % 1 --- nomera razdelov i prilozheniy otdeleny
                              %       tochkoy s probelom, podrazdely propuskom
                              %       bez tochki;
                              % 2 --- nomera razdelov, podrazdelov i prilozheniy
                              %       otdeleny tochkoy s probelom.

%% Vyravnivanie zagolovkov v tekste %%
\setcounter{headingalign}{0}  % 0 --- po tsentru;
                              % 1 --- po levomu krayu

%% Razmery zagolovkov v tekste %%
\setcounter{headingsize}{0}   % 0 --- po GOST, vse vsegda 14 pt;
                              % 1 --- proportsionalno izmenyayushchiysya razmer
                              %       v zavisimosti ot bazovogo shrifta

%% Podpis tablits %%

% Smeshchenie strok podpisi posle pervoy stroki
\newcommand{\tabindent}{0cm}

% Tip formatirovaniya zagolovka tablitsy:
% plain --- nazvanie i tekst v odnoy stroke
% split --- nazvanie i tekst v raznykh strokakh
\newcommand{\tabformat}{plain}

%%% Nastroyki formatirovaniya tablitsy `plain`

% Vyravnivanie po tsentru podpisi, sostoyashchey iz odnoy stroki:
% true  --- vyravnivat
% false --- ne vyravnivat
\newcommand{\tabsinglecenter}{false}

% Vyravnivanie podpisi tablits:
% justified   --- vyravnivat kak obychnyy tekst («po shirine»)
% centering   --- vyravnivat po tsentru
% centerlast  --- vyravnivat po tsentru tolko poslednyuyu stroku
% centerfirst --- vyravnivat po tsentru tolko pervuyu stroku (ne rekomenduetsya)
% raggedleft  --- vyravnivat po pravomu krayu
% raggedright --- vyravnivat po levomu krayu
\newcommand{\tabjust}{justified}

% Razdelitel zapisi «Tablitsa #» i nazvaniya tablitsy
\newcommand{\tablabelsep}{~\cyrdash\ }

%%% Nastroyki formatirovaniya tablitsy `split`

% Polozhenie nazvaniya tablitsy:
% \centering   --- vyravnivat po tsentru
% \raggedleft  --- vyravnivat po pravomu krayu
% \raggedright --- vyravnivat po levomu krayu
\newcommand{\splitformatlabel}{\raggedleft}

% Polozhenie teksta podpisi:
% \centering   --- vyravnivat po tsentru
% \raggedleft  --- vyravnivat po pravomu krayu
% \raggedright --- vyravnivat po levomu krayu
\newcommand{\splitformattext}{\raggedright}

%% Podpis risunkov %%
%Razdelitel zapisi «Risunok #» i nazvaniya risunka
\newcommand{\figlabelsep}{~\cyrdash\ }  % (GOST 2.105, 4.3.1)
                                        % "--- zdes ne rabotaet

%%% Tsveta giperssylok %%%
% Latex color definitions: http://latexcolor.com/
\definecolor{linkcolor}{rgb}{0.9,0,0}
\definecolor{citecolor}{rgb}{0,0.6,0}
\definecolor{urlcolor}{rgb}{0,0,1}
%\definecolor{linkcolor}{rgb}{0,0,0} %black
%\definecolor{citecolor}{rgb}{0,0,0} %black
%\definecolor{urlcolor}{rgb}{0,0,0} %black
